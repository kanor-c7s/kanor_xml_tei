Okay, here is the LaTeX code generated from the `Kanor.html` file.

I've used standard LaTeX packages and defined some custom commands to handle the specific HTML markup (like folio numbers, notes, corrections, etc.). I've opted for footnotes for the tooltips and notes, and used `\sout` for deletions and `\textsuperscript` for additions. Lettrines are included using the `lettrine` package.

**LaTeX Code (`Kanor.tex`):**

```latex
\documentclass[12pt]{article} % Using article class, consider 'book' for chapters
\usepackage[utf8]{inputenc}
\usepackage[T1]{fontenc}
\usepackage[french]{babel} % For French typography rules
\usepackage{geometry}
\geometry{a4paper, margin=2.5cm} % Adjust margins if needed
\usepackage{lettrine}
\usepackage{xcolor}
\usepackage{graphicx}
\usepackage{amsmath} % For \text
\usepackage{amssymb} % For symbols
\usepackage{ulem}    % For \sout (strikethrough)
\normalem           % Keep \emph as italic (ulem changes it otherwise)
\usepackage{hyperref} % For links
\usepackage{verse}    % For songs/rondeaux
\usepackage{marginnote} % For folio/column marks in the margin

% --- Define Colors ---
\definecolor{darkblue}{rgb}{0.0, 0.0, 0.55}
\definecolor{darkred}{rgb}{0.55, 0.0, 0.0}

% --- Custom Commands ---
% Folio/Column marks - places them in the margin
\newcommand{\foliomar}[2]{\marginnote{\href{#2}{[#1]}}}
\newcommand{\foliomarID}[1]{\textsuperscript{[#1]}} % Folio ID inline superscript
\newcommand{\colmar}[1]{\marginnote{[#1]}}          % Column break mark in margin

% Text Markup
\newcommand{\persName}[1]{\emph{#1}} % Person name
\newcommand{\placeName}[1]{\emph{#1}} % Place name
\newcommand{\num}[1]{\texttt{#1}}    % Numbers (e.g., Roman numerals) - using tt for distinctness
\newcommand{\rubric}[1]{\textit{#1}}  % Rubricated text
\newcommand{\corr}[2]{\emph{(#1)}\textbf{›#2‹}} % Correction command (orig -> corr)
\newcommand{\add}[1]{\textsuperscript{#1}}       % Added text (simple superscript)
\newcommand{\supplied}[1]{\textlangle#1\textrangle} % Supplied letter (requires textcomp or amssymb/amsmath)
\newcommand{\uncertain}[1]{\textit{[#1?]}} % Uncertain reading (italicized in brackets with ?)
\newcommand{\missing}{\texttt{[\ldots]}}     % Missing text marker [×] - using ellipsis in tt
\newcommand{\fntooltip}[1]{\footnote{#1}}       % Generic footnote for tooltip content
\newcommand{\fnnote}[1]{\footnote{\textbf{Note:} #1}} % Specific footnote for notes (💬)
\newcommand{\fnworknote}[1]{\footnote{\textbf{Travail:} #1}} % Specific footnote for work notes (❓)
\newcommand{\fnfpz}[1]{\footnote{\textbf{fpzanardi:} #1}} % Specific footnote for fpzanardi notes (SVG icon)
\newcommand{\del}[1]{\sout{#1}}      % Deleted text (strikethrough)
\newcommand{\surplus}[1]{\textit{[surplus: #1]}} % Surplus text (marked and italicized)

% Paragraph numbering - simple bold §Num
\newcounter{paranum}
\setcounter{paranum}{0} % Initialize
\newcommand{\pnum}{\stepcounter{paranum}\textbf{§\arabic{paranum}}\quad}

% Redefine lettrine default for potentially better spacing
\renewcommand{\DefaultLoversize}{0.1}
\renewcommand{\DefaultLraise}{0.1}


% --- Document Setup ---
\title{\textit{Li histoire de Kanor et de ses freres}}
\author{Anonyme (Transcrit depuis HTML)} % Or leave blank if not needed
\date{\today}

\hypersetup{
    colorlinks=true,
    linkcolor=blue,
    urlcolor=magenta,
}

\begin{document}
\maketitle

% --- Start of Content ---

\foliomar{1ra}{https://gallica.bnf.fr/ark:/12148/btv1b10023851v/f7.image} \foliomarID{1r}

\section*{[C1]}
\rubric{Ichi encomence li histoire de \persName{Kanor} et de ses freres, liqueil furent fil au noble \persName{Kassidorus}, empereor de \placeName{Costostinnoble} et de \placeName{Rome}, liqueil furent engenré en l'emperis \persName{Fastige}, ki fille fu a l'empereor \persName{Phiseus}.}


\pnum \lettrine[lines=3, findiff=1pt, nindent=0pt]{\color{darkblue}H}{a !} Diex, si sousfissanment ai esté requis de noble prince \persName{Huon de Casteillon}\fnnote{En fait, il semble que les conclusions de Lewis Thorpe, dans son article "Paulin Paris and the French sequels etc.", soient déjà présentes dans un ouvrage de Maur Dantine (18e siècle), L'Art de vérifier les dates historiques des chartes, des chroniques et autres monuments, depuis la naissance de J.-C. Le ms. 1446 y est bien rattaché à Hugues VI de Saint-Pol, le fils, et non Hugues I de Châtillon. C'est un témoignage qui date du temps où le ms. était chez monsieur de Thou, dont on trouve bien l'ex-libris.} et comte de Saint-Pol, pour lequeil je ne me porroie mie tenir que toute m'entente ne me covingne metre a ce que il premierement, et autres en apriés lui, sacent qui cil quatre frere furent, dont je, en la rebriche ci encoste, ai fait mension. Et por ce que li envieus\fnnote{topos de prologue que de rejeter les critiques à l'avance. Ce qui est intéressant, c'est l'utilisation d'un cas sujet alors que finalement le GN est CO2 : ou alors il y a une anacoluthe...}, qui nul avancement ne vauroient avoir sor nul home qui blasmer seuist aus, ne ceus de leur lignie, veil je le mien non del tout en tout celer, quant il me doit sousfire a ce que li miens sire le sace devant dis, sans cui confort je ne porroie mie teil ouvrage asovir legierement, si veil comencier en teil maniere que je en apiel celui qui en gré prent toutes ouevres qui sont faites et dites en colour de droiture, de raison et de verité. Çou est cil qui tout cria et fist por home mener et conduire el regne dou ciel.

\pnum Voirs fu, si come je truis en pluisors lius escrit, que jadis eut un empereour en la citei de \placeName{Coustantinoble}, ensi come je desus ai dit. Cil eut non \persName{Kassidorus} et fu nobles princes si come li histoire le devise, qui dist en teil maniere que il avoit eu \num{ii} femes, desqueles il avoit eut enfans ; liqueil avoient eut dissencion entr'iaus, si come il est contenut en l'istoire qui vient devant cesti. Mais por çou que je ne puis mie legiererement entrer en matere qui se puist acorder a cest dont je vous veil touchier, si m'estuet venir a çou que vous sachiés en cestui livre qui furent ces \num{ii} dames dont li emperere desus dis avoit eut ces enfans, por ce que li aucun qui mie n'ont oï de devant sacent qui cil furent dont je vorrai faire mension en mon conte.

\pnum La premiere de ces \num{ii} dames si fu nee de \placeName{Galilee}, d'une citei qui avoit non \placeName{Bethsaïde}, fille a un prince qui avoit non \persName{Hedipus}, et eut non la dame \persName{Helcana}. \persName{Kasidorus}, qui a celui tans ert jones damoisiaus ausi come de l'eage de \num{xii} ans, estoit emperere de \placeName{Costantinoble}. Et vinrent a lui li prince de l'empire, qui li dissent que bone chose seroit que il feme priist\fnnote{Première occurrence du double "i" typique de la scripta picarde : à voir si on conserve ou corrige.}, por ce que il ne voloient mie que la tiere demourast sans hoir de sa char qui la tiere tenist enapriés lui. Li damoisiaus avoit teil volenté que il lour respondi qu'il n'estoit encore mie li tans venus que volenté en euist. Dont avint que li baron qui ce oïrent vorent savoir par art d'ingremancie queus raisoins a ce l'amenoit. Dit lor fu, par ciaus qui\del{u} de ce savoient ovrer, qu'il covenoit que li damoisiaus euist une feme par coi li plus proisiet de l'empire fussent destruit, et ne mie par le cope de l'empereour ne de l'empereris mais de lor coupe \missing{} feme. Dont il avint apriés cestui sors que li damoisiaus \persName{Kasidorus} cierça mains diviers païs ou il se fist conoistre \missing{} et avint issi com par aventure que il s'enbati en la \missing{} que on apiel \missing{} \placeName{Betsaïde}. Il s'acointa a\missing{} en teil maniere que li damoisiaus \missing{} anchois repaira en son \colmar{1rb}\colmar{b} paiis. Et avint qu'il ne demoura mie mout que la damoisiel\corr{e}{e} qui avoit mis son cuer au damoisiel fist tant par force d'esperiment que il sambloit au damoisiel chascune nuit que cele li venoit de devant, et li disoit paroles necessaires ; par coi il, au matin, se leva et manda de ses p\supplied{r}inces, et dist par sa bone foi que or ert li tans venus qu'il voloit celi avoir qui li ert venue en son dormant. Quant ce entendirent li baron, si s'aviserent de lor sort dont j'ai desus dit, et missent le plus grant debat par paroles contraires, par coi li damoisiaus ne s'en meust. Mais ne lor valut, car si l'aloit l'amours a la puciele destraingnant que il s'en mist hors de son empire, et cierça tamaint divers paiis avant qu'il peuist asener a celi de cui je paroil.


\pnum \lettrine[lines=2]{\color{darkred}A}{vint} que, quant il eut tant alé come une merveille, il s'en vint en \placeName{Bahtsaïde} et trova que çou ert \persName{Helkana}, dont je devant ai fait mension. De coi il avint qu'il l'en amena en \placeName{Gresce} et l'espousa a la guise de son paiis. Il n'eurent mie esté mout ensamble quant nouviles vi\supplied{n}rent de \placeName{Rome} que sa cousine \persName{Fastige} avoient li Roumain demise et chacïe\fnnote{Pas d'accentuation sur le «e» final : participe passé picard : le radical se termine par un son palatal, la final «-ie» est une réduction de «-iee». On peut dire que le «e» final est le morphème grammatical de genre féminin.} de la citei et que il le vausist soucoure en auteil maniere come li siens pere \persName{Pheseus} avoit soucoru Laurun le sien pere. Que vous iroie devisant ? Li vaillans \persName{Kassidorus} s'avisa, et dist que <<li une bontés l'autre recuiert>>.\fnnote{L'une bonté l'autre requiert (Morawski 1146).} Il atorna sa voie et laissa sa feme en garde un sien ami. Il en ala a \placeName{Rome} et sa feme demoura enchainte. Et avint que li tiermes ala tant que la dame envoia a \placeName{Rome} a son signor por lui savoir coment il \supplied{e}n\fnnote{Ce "n" vaut souvent pour "en" : à mettre dans la syntaxe} estoit et savoir de lui son iestre. Li més que cestui mesage deut faire le fist en teil maniere ke sa letre fu fausee, et trova \persName{Kassidorus} escrit que cil en cui garde il avoit sa feme laissie li mandoit que elle despuis qu'il s'ert partis de li, elle ne s'estoit gardee mout autrement come se elle fust a tos conmune, et que il mandast de queil mort il voloit que on le feist morir.

\pnum Cil qui mout ert sages et aviseus remanda a celui que telle come il li avoit charcié, il li gardast de ci a son repairier. Et qu'avint de çou ? Li traitour\fnnote{Ici, il s'agit de "traitor" "négociateurs" qui se trouvent être "traïtor", traitres. On garde "traitor".} qui de çou furent porveu ont celi letre refausee et ont rescrit a celui qui la dame avoit en garde que li emperere avoit esté requis de ciaus de \placeName{Rome} qu'il voloient qu'il fust lor emperere, et ce ne povoit iestre tant conme sa feme vesquist et que il meist paine a ce que il porchaçast sa mort a l'enfanter, par coi en cuidast que ce fust par l'enfant dont tamainte autre est malmise. Ensi avint que ceste letre vint a celui qui mout fu esbahis, et cuida vraiement que ce fust volentés son signor. Et qu'en avint ? Cil qui fu decius cuida faire ausi come on li eut mandé. Mais cil qui tous ciaus garde cui il li plest le fist en teil maniere que, quant on le deut murdrir, elle s'escria, et dist : <<Aide moi, Sire Dius tous poissans !>> Et dont chaïrent toutes pasmees celes qui tenoient les coutiaus por li ochire et l'enfant qui ja ert nes, la tres plus biele creature c'onques Diex feist, et eut non cil de \persName{Helcanus} contre la mere. De coi il avint que la mere covint puis widier come povre chaitive son enfant entre les bras, et quant ce avint que fu hors de la contree, elle s'enbati en une grant foriest u ses fius li fu ravis ausi come d'aucuns esperites qui en fisent teil garde, come dit est en l'istoire.

\pnum La dame, qui en cestui afaire ert a une fontainne ou ele se reposoit, vit que elle eut son enfant pierdut. Si avint que Dius la conforta d'un saint hiermite enchiés cui elle demoura, tant come il est contenut devant, mais de li me covint ore ici endroit taire et venir a ce que \persName{Kassidorus} \foliomar{1va}{https://gallica.bnf.fr/ark:/12148/btv1b10023851v/f8.image} \foliomarID{1v} mena tant ciaus de \placeName{Rome} qu'il envoierent a lui et li manderent qu'il voloient pais a lui par maniere qu'il recheuist l'empire en sa main et lor dounast signor covignable. En cestui point vi\supplied{n}rent novieles de \placeName{Gresce} a l'empereour que li empereris estoit morte d'enfant, adont eut teil duel li emperere que nus ne pot a lui besoigner dedens \num{v} jors. Mais quant li \num{v} jor furent passé, si vinrent li prince a lui et li dissent : <<Sire, ``il n'est si male qui n'aïut, ne si bone qui ne griet''\fnnote{Phrase proverbiale : il n'y a si bonne chose qui ne puisse causer du mal et si mauvaise qui ne puisse venir en aide cf. FJ (paragraphe 406, page 485, Cassidorus) cf. Langlois 462}.>> Dont li misent avant qu'il covenoit qu'il recheuist l'empire de \placeName{Rome} et prist a feme \persName{Fastige}, qui mie ne li ert si pres qu'il ne le peuist bien avoir en non de concorde et de pais. Isi avint que cil mariages fu fais et le rechiurent Ronmain a signor. En ce que cis mariages fu fais, vint un des princes de \placeName{Costantinoble}, et dist tant a l'empereour d'un et d'el qu'il envoia en \placeName{Costatinoble}, por celui destruire que l'empereris avoit eut en garde, mais ausi come Dius, qui mie ne volt sousfrir la destruction dou loial siergant, le tensa, par ce \surplus{qu'}il fu mis en forte et orible prison de ci a dont que \persName{Kassidorus} revint en \placeName{Gresce}, mais ainch por ce ne fu delivrés, si sutil furent li traïtor, qui la bone dame cuidierent avoir mise a mort.\fnnote{La syntaxe et le sens de cette fin de pbrase ne fonctionnent pas...}


\pnum \lettrine[lines=3]{\color{darkblue}L}{i} emperere, qui en nule maniere ne povoit oublier la mort de l'empereris que li cuers ne li deist que morte avoit esté ausi come par aucune defaute, fu teus menés, que nulement il ne se peut prendre a nule bone besoigne de croistre son pris ne sa bone renomee. Dont il avint que novieles coururent en \placeName{Galilee} au boin prince \persName{Hedipus} que sa fille estoit morte par grant traïson, por coi il a quist tant de gent qu'il vint en \placeName{Gresce} si esforciement qu'il atorna teil l'empereour qu'il ne se povoit desfendre. Anchois envoia a \placeName{Rome} por soucors, qui gaires ne li valu. Mais ce veil je laissier ester et venir a ce que li empereris, qui ert issue a tel tort de son empire, demoura aveuc li ermite en guise de jovenciel, ausi come il est contenut en l'istoire. Et av\supplied{i}nt de çou une mout fiere aventure, car cil qui avoit le traïson porchacié, dont la dame estoit en essil, prist une maladie dont nus ne le peuist garir, ne fust par l'empereris qui le gari ausi come Dieus i vot mostrer de ses o\add{^ue^}vres. Si en eut la dame \corr{povre}{pou}\fnnote{povre me paraît curieux, je corrigerais en pou, en considérant que l'abrévation est une erreur.} de fierté, car une fille avoit li princes, qui mout fu plainne de pus ars quant elle fu enchainte d'un chevaliers et le mist seure l'empereris, por coi elle fu prise de la u elle demouroit aveuc li ermite et menee en une \add{^\missing{}^} foriest por li devorer, u il n'avoit fors liuons, ours, lupars et males biestes. La endroit le tensa cil qui \persName{Daniel} tensa en la fose u il fu mis por devorer, et avint de ce uns gratieus miracles, car cele par cui il ere la mise ne peut iestre delivré de l'enfant dont elle fu enchainte, de ci a dont qu'ele jehi son malisse a tout le parole. Por coi cil qui l'avoit corumpue fu envoiiés en l'ile ou li empereris avoit esté por savoir se cil ert mors, dont cele avoit faite fause coupe. Cil sans cui cis contes ne porroit mout ligierement iestre asouvis s'en vint en l'ille et trova la bone dame qui aoiroit a son Createur, et quant elle vit celui, si seut por coi et a coi il beoit et lors dist : <<Amis, or soiiés en vostre bone pais et vous remetés arriere, et dites hardiement a vostre damoisiele que devant ce que cil que je covoite me jetera de ci, elle n'iert delivree dou fruit que ele me mist seure. Et por ce que vous saciés que ce soit voirs, metés vous arier ensi come vous iestes venus et alés a mon bon ami \persName{Ydoi}\colmar{1vb}\colmar{b}ne en la \placeName{foriest de Volgan}, et li dites cest noviele isi come dit vous ai.>> Isi conme vous povés entendre, vint cil a l'hiermite et li dist ce que ¶ vos avés oï.


\pnum \lettrine[lines=3]{\color{darkred}L}{i} hiermites, quant il çou entendi, si loa son Creator, et dist : <<Amis, vous en irés en \placeName{Coustantinoble} et la troverés vous l'empereour, et li acointiés vostre afaire, car ce est cil qui vous puet delivrer.>> Cil se mist en son chemin et era tant qu'il trova un pavillon ou il avoit une roine, dames et damoisieles, et en si grant deduit que il cuida iestre en paradis quant vit leur maniere, car elles le fissent bienvingnant et li covint illuech demorer la nuit ; et vit illuech un damoisiel ausi come de l'eage de \num{vii} ans, qui mout grant compaignie li porta la nuit. Mais au matin ne seut cil u il fu. Anchois se trova gisans sor son escut armé de toutes armes et son cheval dejouste soi en ce qu'il se dreça, si vit venir de pres le damoisiel qu'il avoit veut la nuit devant, et cil vint a lui et li dist : <<\persName{Licorus}, montés tost et isniel, si me tenés compaignie a asovir ce dont je sai que vous ieste en la queste.>> Cil qui joians fu sailli el cheval, si se missent a la voie, si ont tant chevauchié par lor jornees qu'il vinrent a une liue de \placeName{Costantinoble}. Dont il avint que il ert mout matin, et virent que \num{vii} chevalier avoient un chevalier avironé et si feroient sor lui si aigrement come por lui metre a mort. Cil sor cui cil feroient se desfendoit tant noblement come il ne les prisast se pau non, mais en la fin li ochisent sor pau d'eure son arrabi desous lui. A cest point, \persName{Licorus} s'enbati sor aus et ne se traist mie devers la force. Anchois se mist contre le\supplied{s} \num{vii} chevaliers, par coi il n'eurent duree a l'aide d'un liuon que \persName{Licorus} avoit o soi ; par le coi\fnnote{Quelle analyse faire de ce « coi » ? Un pronom relatif, une forme de « lequel » ? Possibilité que « le » soit une erreur (habitude d'écrire « parle »)... C'est tout à fait inhabituel, mais cela me paraît possible : c'est en effet une forme du pronom relatif, ici le neutre « quoi » est en quelque sorte substantivé par l'article « le », pour un équivalent de pronom relatif composé (« lequel » en effet). Il faudra faire une note dans l'introduction linguistique. Décidément la langue du copiste est pleine de surprises !} cil furent tout desconfit et trovere\supplied{n}t que cil chevaliers qui seus se conbatoit as \num{vii}, que çou ert \persName{Kassidorus}, li emperere de \placeName{Costantinoble} qui, par destrainte de ciaus de \placeName{Galilee} qui a ce l'avoient amené, \surplus{qu'il} n'avoit povoir de lui desfendre. Anchois aloit soucors cuerre a \placeName{Rome}, quant li damoisiaus dist : <<En non Dieu, biaus pere, vees ici le soucors que je vous aport. Ma mere vous mande que vous a li venés, ausi come vous avés fait autrefois.>>

\pnum Quant li emperere a le damoisiel entendut, si fu si esbahis qu'il ne respondi mot en piece. Mais en la fin ne fust nus qui grant pitié n'en peuist avoir ; et que me vauroit ore avant faire ci plus lonc conte quant aillours est contenu mieus et en millor maniere ? Mais por le mius entendre avant vient ore li contes a çou que li emperere et li damoisiaus reparriere\supplied{n}t en l'ost \persName{Hedipu} et fu ceste chose acointïe a tous ciaus qui il le covint savoir. Mie ne demoura que li emperere o lui \persName{Hedipus} vinrent aveuch aus grant plenté de barons, si sont venut par lor jornees enchiés celui \persName{Polum}, et lor fu conté ce qu'il cuidierent que bon fu. Apriés ce, li emperere et \persName{Licorus} meime \persName{Hedipus} et li damoisiaus et encor et encor autre de lor privee amor, se missent en l'ille ou li empereris estoit. La i ont trovee de robes imperiaus aornee et a mout grant merveille reciut son signor sagement en souspirs et en larmes, et il li. Mais de nule rien plus ne veil faire mension fors tant que la damoisiele enfanta et parla li fruis qui de li se parti si tost que il fu nés. De la se sont parti et vi\supplied{n}rent par \persName{Ydoine}, et pus n'ariesterent de ci en \placeName{Costantinoble}, ou il n'atarga mie mout que vengance fu prise de ciaus qui la traïson porchacierent et firent. En apriés ces afaires tous asovis vint l'empereris de \placeName{Rome} a l'empereour et prist congiet a l'endemain en larmes et en plours, et dist a lui priveement : <<Chiers sire, il me covint de vous partir a ma confusion dolante et enchainte, \foliomar{2ra}{https://gallica.bnf.fr/ark:/12148/btv1b10023851v/f9.image} \foliomarID{2r} or me veille Dieus otroiier en aucun tans perseverer ma vie et \missing{} grant joie que je de vos ne me puisse partir.>> -- <<Ha ! Dieus, dist li emperere, et \missing{} qu'on ne puet contrester, ne covient fors que le cuer mener a \missing{} pais, et se vos ce ne faites, il ne puet mie legierement s\missing{} qu'en aucune maniere cil courous ne vos doit iestre convertis \missing{} Et si avient soi nient qu'il n'est si male qui n'aïut ne si bo\supplied{ne qui} ne griet, ne d'autre part nus ne doit iestre dolans d'autrui a\missing{}ment mais qu'il vigne de droiture.>> -- <<Ha ! Sire, dist la dame \missing{} diés mie que je soie dolante de çou que jou voie et sai qu'il\missing{} venit ma dame telle honors, que si me doist cil o \missing{} naistre me fist joie de cest fruit que jou ai senti en ni\missing{} Je ne vorroie mie demourer en ceste honor.>> \fnworknote{Le balise des dialogues n'est pas clair ici ; on comprend que l'empereur parle, mais je ne sais pas où placer ses premiers mots.} Et elle f\missing{} moree el blasme ou elle estoit sans sa desierte. Mai\missing{} jou voi droiture sormonter, traïson et torcenerie mes \missing{} ir parfaite joie de mon tres grant anui. Quant li emperere eut \missing{} me, si fu mout joians et en celi joie en eut trop grant \missing{} si dist : <<Dame, s'il est ensi come vos dites, si on loes \missing{}>> que ceste grasce li avoit dounee. Et il l'aseuroit \missing{} son envenroit a parfaite joie. En ceste maniere s\missing{} Rome de l'emperere, et s'en revint a Roume ou ele de\missing{} Si fait ore ci fin ceste histoire et reparte li \missing{} ce en teil maniere come vos ci aprés porés oïr \missing{} pereris de Roume se fu partie de son signor ou\missing{} ai faite rekapitulation de l'histoire, \missing{} tier au plus briement que je porai l'autre qui ci\missing{} le. Je ne porroie mie entrer en mon propo\missing{} fussent plus empeechié coment que j'en soi\missing{}


\section*{[C2]}


\pnum \lettrine[lines=3]{\color{darkred}A}{vint} aprés ce dit desus que grant\missing{} dit est furent venu\missing{} de Constantinoble demour\missing{} regne en pais et en concor\missing{} mie bien alé. Et en apr\missing{} ce que li emperere et l'empereris\missing{} nable. Lor vint n\missing{} \num{ii} damoisiaus\missing{} furent joiant\missing{} cil qui ceste\missing{} si s'en loa\missing{} li emperere\missing{} \colmar{2rb}\colmar{b} \missing{} \foliomar{2va}{https://gallica.bnf.fr/ark:/12148/btv1b10023851v/f10.image} \foliomarID{2v} \missing{} \colmar{2vb}\colmar{b} il eut guerpi ensi come jou ai dit desus. Novieles en sont venues a son frere qui maintenant seut quel part elle verti. Dont envoia a \missing{}mie et enquist de li noviele a l'un lés et a l'autre, et ne peut iestre celé que \missing{}ne le seuist, et fist tant par une piere que il avoi qui ert invisible \missing{} se parti de Roume et eut armes et cheval, et pus se mist en la queste en \missing{} maniere que ainch mais en nul conte de plus noble aventure n'oï \missing{}r, car uns murdreors avoit la puciele encoupee, que ile avoit \missing{}rd\add{^i^}é une pucelete que ele avoit en garde. Et il meisme avoit le \missing{}rdre fait por la raison de ce que elle ne se voloit acorder en vilon\supplied{ie f}aire de son cors. Et por ice li avoit fait cest encriemé que il ert \missing{}de prover soi contre un autre s'il le voloit escondire. Cis escondis \missing{}is mout a point ensi com il est conté aillors, car Helcanus vint si a point \missing{}puciele fust perïe. Quant il se presenta contre le traïtor et le conquist \missing{}nes com cil qui bien faire le seut. Et dist li contes que tout ce a point \missing{}la partie a l'emperere eurent bataille contre Peliarmenus et sa gent et furent \missing{}confi et mis a merci Fastidorus et li rois d'\placeName{Aragon} pris et Costantinoble \missing{}se deviers l'empereour en ateil maniere come se chascuns fust saisis ¶ d'iaus mil.


\pnum \missing{} emperere vit ceste besoigne ensi faite, si nus \missing{}eil de ses millors amis et il iront tuit conseillié. \missing{} avoir Kassidorus por sovoir l'ordenance de cesti besoigne, \missing{}chemin des plus proisiés, si n'ont finé l'un jor plus l'autre \missing{}t venut ausi conme dit lor fu sor le chemin. La \missing{}de lui. Beneïçon mere de Diu, dist chascuns. \missing{}aia peri ce que Dieus velt sauver. Ensi avint de \missing{}amoit le chevalier Helcanus com il est contenu en l'istoire \missing{}idi le ramenere\supplied{n}t a teil honor que cuers ne \missing{}boute d'escrire. Quant li emperere eut de cesti \missing{}de nule autre rien ne savoit fait si \missing{}t. eurent teil merveille de l'honor \missing{}qu'il disent que tout avoit pardoné \missing{}is. por coi li emperere dist oiant maint \missing{}que li cuers vos ensiece que je \missing{}et de mon chier fil arai \missing{}ci ele si fu mout joians, \missing{}is boo gré vousi \missing{}se mist ensanble, \missing{}. et dot pus apres \missing{}mout furent \missing{}furent a \missing{}s consau \missing{} \foliomar{3ra}{https://gallica.bnf.fr/ark:/12148/btv1b10023851v/f11.image} \foliomarID{3r} qui les enfans devoit avoir mal mis, ensi come il est contenu aillors, a \persName{Dorus} por faire toute sa plaine volenté. Dont il avint que \persName{Peliarmenus} ne vot tenir covent. Et qu'en avint ? \persName{Dorus} qui, au jor de dont, passoit tous les autres chevaliers de proece, fist tant qu'il se mist a aler viers \placeName{Rome} a \corr{t}{t}ot asés pau de gent, et prist \persName{Peliarmenum} en une foriest ou il ert alés chacier, et dist li contes que, <<en teil maniere que on porte poisson de la mer sor un cheval>>, en \num{ii} paniers furent mis lovet et tenchillié \persName{Peliarmenus} et \persName{Dyalogus}, dont je devant ai fait me\supplied{n}sion.


\pnum \lettrine[lines=2]{\color{darkred}E}{nsi} n'ont finé tant qu'il vinre\supplied{n}t el chastiel de Luxe\supplied{m}bours et furent illuec une piece em prison tant que \persName{Helcanus} le seut et manda son frere \persName{Dorus}, et li amenast ciaus dont il avoit oïe teil noviele. Il vint tost o lui le duch son signor, et fist ciaus amener en Grese. Et avint que nul autre conseil \persName{Dorus} ne veut croire, que il covint derechief que tuit li baron qui devant avoient esté a l'ordenance furent mandé por le confirmation et l'amende jugier, qui a ce apartenoit, qui issi avoit esté menee. Meime li emperere covint chierkier et cuerre de ci a dont qu'il fu trovés enchiés un hiermite o lui un liuon qui porte compaignie, ausi come contenu est el conte. Ensi vint li emperere en \placeName{Coustantinoble} a tout le liuon, ou on en fist mout grant joie de lui, et le tint on a grant merveille de çou que li liuons le sivoit en teil maniere que il a nului ne faisoit mal. Mais de ce ne me covient or mie tenir conte, mais a ce venir briement que li baron i furent venut de tamainte region, isi com il avoient fait devant, por faire aide a l'empereour et a son fil. Quant il furent tout ensamble, s'en i eut \num{xv} principaus qui furent esleu por le concorde et le pais asovir. Et por ce que je veil que vous mius sachiés qui il furent, por ce qu'il m'en covenra aidier en mon conte ci apriés, le mesestuet il nomer ci endroit, por plus covignablement puisier ma matere, qui de cesti doit movoir. Li premiers et li plus grans sire fu li rois d'\placeName{Aragon} et cil apriés, por cui on en feist plus d'une part et d'autre, ce fu li cuens de \placeName{Flandres}, li tiers \persName{Hedipus} de \placeName{Galylee}, li quars \persName{Japhus} de \placeName{Frige}, li quins d'\placeName{Espaigne} \persName{Josias}, li sisimes fu \persName{Daphus} li Gris, li sietiesmes \persName{Mirus} li Fiers, li witimes \persName{Gazaus} de \placeName{Rome}, li nuevimes \persName{Karus} de \placeName{Nisse}, li disimes \persName{Cliodorus}, li onsimes \persName{Nestor} d'Aquillé, li dousiemes \uncertain{Hecas} de \placeName{Frige}, li tresimes \persName{Mardocheus} li Grius, li quatorsimes \persName{Leus} et li quinsimes, que je deuisse avoir premiers nomei, fu li dus Lembourgis, qui le plus covignable voie trova, par coi li une partie et li autre furent grant tans ensamble bo\surplus{n}in ami. Si ne veil ore mie faire une longue devise, car aillors est contenu et jou el ai a entendre, si veil venir a ce que li emperere covint par sa volenté et a le requeste des damoisiaus deseure dis, que il repairast a \placeName{Roume} a l'empereris \persName{Fastige}, qui a merveille ert bone dame et de grant se\supplied{n}s aornee. Dont il avint que apriés toutes ces choses devant dites et faites, tuit li baron qui a ceste concorde furent s'en vinrent a Rome por plus honoreement faire l'asamblee. Por coi il n'avint onques si grant joie en \placeName{Rome} com il avint de cestui afaire, si me veil ore a ce metre que je ci endroit face fin de ma rechapitulation et entre en ma matere, dont jou ai fait mension \add{^en^} mon prologue devant, et comence ici endroit mon livre.


\section*{[C3]}


\pnum \lettrine[lines=4]{\color{darkred}D}{ieus} qui, par sa grant puissance, le monde establi, il doinst honour et joie parfaite a mon tres chier signor devant nomei, por lequeil j'ai enpris a traitier et metre en conte, apriés ce que je devant ai dit, de honorei \colmar{3rb}\colmar{b} empereour \persName{Kassidorus}, qui jadis fu sire des Roumains et des Gris, que, quant il furent repairiet a \placeName{Roume}, isi com il est contenu en l'istoire, li baron vinrent a lui, et li disent tot emsanble : <<Sire, vees ici ma dame l'empereris qui vient contre vos, et toute \placeName{Roume} s'esjoiist de ce qu'ele puet, si conme vos poés veoir.>> -- <<Biau signor, loés en soit cil, cil qui tout puet justicier et metre a son droit.>> Adont s'asamblerent li emperere et l'empereris, et conjoï li uns autre en teil maniere que, tuit comunalment, en orent grant joie, et mervielleusement furent loé li uns et li autres de lor noble contenances. Coment li emperere se maintint, qui a l'avis de chascun sor tous les autres il moustroit, et ert apierte chose a tous qu'il sourmontoit tous autres princes de sens, de biauté, de parfaite honor et de noble contenance. Dont il avint de lui çou qu'il n'avient mie souvent, que il ne covenoit mie demander a ciaus qui devant ne le conurent : <<Liques est ore li emperere ?>> Anchois le conurent tot, a ce que il ert la flors de tous, ausi come ¶ jou ai dit deseure.


\pnum \lettrine[lines=3]{\color{darkblue}L}{i} empereris, a l'autre lés, qui sormonta totes autres dames qui i furent, tout en auteil maniere come de noble contenance, estraite d'umilité, de parfaite ordenance, aesmee de biauté. N'estoit nus qui grant bonté ne tenist, que grant vaillance n'euist dame qui teus grasces avoit entre toutes, que l'en peuist dire : <<C'est la rose entre la flour de Kaneson.>> Et nonporquant i avoit des plus bieles dames et pucieles dou monde. Ensi furent sor toutes autres aorré de grasce a celui jour li emperere et li empereris. Si ne me plaist ore que je vous en fache \add{^plus^} autres devises que il vinrent a la grant eglise de \placeName{Roume}, et furent li emperere et li empereris reconcillié del pape et des chardenaus, en teil maniere com il cuidierent que raisons aportast. Apriés ce, repairient el Palais Majour. La endroit fu tous porpendus de dras d'or et de maint autre pluisor riche drap de soie ouvré diverseme\supplied{n}t em pieres presieuses, dont il i eut tant de chascune maniere que li aucun disent, qui a teil afaire se counurent, qu'il ne cuidoient mie qu'en tout le remanant dou monde en euist encore autant. Selonc ceste ordenance furent tout autre mestier establi, dont je ne veil ore mie faire de tous mension. Anchois pense chascuns selonc ce qu'il a d'avis, li uns plus et li autres mains, coment on peuist por nule painne ne nul cost ceste asamblee plus noblement asovir au greit de toute maniere de gent, autresi fu il fait, et encor plus selonc\del{e} ce que li escris en fait mention. Por coi chascuns puet bien savoir que li prince devant nomei furent de l'empereris porveu chascuns de diviers acesmemens de lor armes, por eus et por lor chevaliers, si que ce fu de la rien de coi li emperere fu plus joians et en seut grignor \add{^gré^} l'empereris. Li baron, d'autre part, furent priiet des damoisiaus de \placeName{Roume} \corr{principlament}{principalment} que por nul coust il ne laissaissent mie a faire une fieste qui a joie et a honor de chevalerie n'apartenist, car bien seuist chascuns que se por tresor ne por avoir peuist on avoir \corr{esvoiturer}{esvoituree} guerre, dont euissent il le lor esvoituree et asovie. Mais il ont bien veut, as amis que lor contre partie avoit, qu'il n'i euist mestier, et por ce fu dis cis proverbes premiers que <<mius vuit amis, qu'autres tresors.>> Dont \persName{Peliarmenus} dist a celi fois au noble prince le conte de \placeName{Flandres} : <<Sire, cest tresor que vostre chiere mere a asamblé, come feme qui mout n'a mie eut afaire, vos proions nos que vos nos aidiés a faire une noble fieste, par coi nus de nos puist entrer en nule male covoitise aprés ce que vous serés departis.>> Quant li cuens eut entendu \persName{Peliarmenus}, si le prisa mout dedens son cuer, et li dist : <<\persName{Peliarmenus}, encor voi jou bien que la fieste ne puet demorer sans grant coust.>> A cest mot \foliomar{3va}{https://gallica.bnf.fr/ark:/12148/btv1b10023851v/f12.image} \foliomarID{3v} fu l'iauwe\fnnote{Première occurrence de \textit{iauwe}. Nous favorisons la graphie \textit{iauwe} sur la base des occurrences actualisées par un déterminant qui n'est pas ambigu : \textit{d'iauwe}, §77 ; \textit{une iauwe}, §250 ; \textit{les iauwe}, §xxx. La graphie est picarde, voir GreubCollet, 2.5d, évolution de \textit{aqua}. Toutefois, §250 (f. 35vb), on rencontre "li auwe" avec la fin de ligne à "li", qui n'est pas suivie du trait de liaison caractéristique, alors qu'il est bien présent dans le reste du folio. Le \textit{li} est senti comme l'article.} cornee et il, selonc çou que jou ai dit desus, fu li ordenance de l'aseoir, chascuns selonc ce que il furent. A celui jor siervi li cuens de \placeName{Flandres} devant l'empereour \persName{Japhus} li Fris et maint autre pluisor prince, por plus honorer tous ciaus qui a ceste joie s'acorderent. De leur mes ne de lor entremes, meime del boire, n'est il mie mestier que je m'entremete dou raconter, car, ausi com jou ai dit desus, bien furent porveut a la volenté de chascun. Apriés cest mangier furent les napes traites, et menestreil apareillié, qui ne furent mie a aprendre de choise de coi il se vosissent meller. Le fisent en teil maniere que mout fisent les baron entendre a iaus. Apriés ce se misent li prince d'une part, et entendirent a çou dont il avoient esté requis des damoisiaus desus dis. <<Beneïçon aiie de Diu, dist chascuns, coment porroit on grignor coust faire de fieste come ceste est encomencïe ?>> -- <<Par foi, dist chascuns, voirement puet on bien savoir que, qui le chose veut faire de bone volenté, que mie ne cuert volentiers escusance dou laissier, et pus qu'il est ensi que faire l'estuet par droite raison, si en faisons la volenté de ciaus a cui li cous en plaist a avoir.>> Dont il avint que il encoumenciere\supplied{n}t a apareillier et a deviser lor afaire coment li mius faisans et cil qui plus avoient de proecce en iaus fusent couneut, et li autre s'entremisent de fieste faire, et de charoles ordener et faire.


\pnum \lettrine[lines=2]{\color{darkred}N}{era}, dont j'ai de devant traitiet ou conte de la recapitulation, n'avoit mie le cuer endormi, qui sour toutes le autres ne\fnnote{encore un 'ne' qui soit est explétif, soit est mis pour 'en' ?} fust honoree apriés l'empereris, si come de l'empereour meime de l'empereris ; et quant ce virent les autres siues serours qui avoient les damoisiaus de \placeName{Rome}, si lor fu avis que por li fussent deshonorees et lor en fesist on mains d'ounor, dont il en deut iestre avenus uns morteus encombriers, ausi come l'istoire en touce qui mie ne se puet taire selonc ce qu'Envie, si puet tout devourer, arreste bone renomee, et vos dirai en queil maniere, ausi come jou avoie encoumencié de devant. \persName{Nera}, qui la plus jone fu des \num{iiii} serours roiaus, ensi come il est contenu en l'istoire, avint a la susperior honor, par ce qu'il li fu porveu et seut que li dius d'Amors le mist de la basse roe de Fortune en la grignor amont, par ce qu'ele fu vrais amans\fnnote{Le mot 'amans', issu du participe présent de 'amare', est une forme épicène : le masculin et le féminin ont les mêmes formes (sauf au CS pluriel, ou le masculin est 'amant' et le féminin 'amans'). Quant à l'adjectif 'vrai', il a pu adopter cette forme de masculin soit par imitation de la forme épicène, soit parce qu'en picard on a souvent confusion entre formes masculines et féminines (voir tout de suite après le pronom 'il' ? En tout cas c'est bien ce que contient le ms : 'vrais amans'} en aviersité, por ce que il fu bien prove\supplied{u} a ceste noble fieste, ou elle enporta le pris de biauté et de jouvent. Por coi il ne fust nus qui en sen cuer ne le prisast, amast et covoitast. Que vous en feroie ore plus longe devise de lor charoles ne de chose qui a ce apartigne ? De celi jornee enporta l'onor et le pris en cui biautés fu enploié, car je ne truis mie que, por biauté qu'ele euist ne por grasce que on d'autre part i covoitast nus, en fust niente envis ce nul, dont reprise peuist iestre envers Dieu ne arme de sa partie. Ceste jornee passa et vint au soir que chascuns se traist a son repos, et ce fait que faire dut. Li emperere et li empereris se furent tost entracointiet come cil qui autrefois l'avoient fait, si n'est ore mie parole que mout covoitast a faire l'un çou qu'il cuidoit qu'il plaist a l'autre, si que celi nuit despendirent en joie et en solas, come cil qui poi dormirent de ci au jour, que tous li palais fu emplis des barons. Li emperere se leva et issi de ses chambres quant \persName{Daphus} li vint a l'encontre o lui un ge\supplied{n}til damoisiel de l'eage de \num{xi} ans, liqueus avoit en lui la flor de biauté. Et si tost come li emperere le vit, si dist : <<Biaus fius, a bien puissiés vous iestre venus.>> Adont s'est cil agenoilliés devant lui et il li mist sa main sor \colmar{3vb}\colmar{b} son cief et li dist qu'il se drechast, et il si fist ; et quant il fu dreciés, si dist \persName{Daphus} : <<Sire, conisteriés vous cest damoisiaus ?>> -- <<Sachiés, dist il, que li cuers me dist que jou li ai couneu et qu'il est fius a Celidoine dou Chastiel Joli.>> -- <<En non Diu, sire, bien l'avés coneu.>> Atant se misent a conseil et eurent avis que il ne le feroient ore mie conoistre. Si fu eure d'aler oïr messe et le Diu siervice. Cil qui faire le durent le fissent, si que apriés ce li pluisor qui se vorrent entremetre de chevalerie avoient as chans fait fichier grans estaces, ausi come en terrastres, et li autre avoient fait drecier quintainnes, et li auquant entreprisent jostes as chevaus pierdre et as chevaus gaignier.


\pnum \persName{Dorus}, qui mout avoit le cuer a ce qu'il peuist venir que on parlast de lui en aucune maniere de chevalerie, \fnfpz{Ce verbe est-il une forme de 'attiser' ou de 'aatier' ? Le sens est 'désirait ardemment (l'entreprise, l'occupation, de sortir en champ - c'est-à-dire de se battre en duel dans le tournoi)'}atissoit l'afaire d'issir as chans, si que tuit cil qui vorrent faire d'armes prisent une soupe en vin et apriés se sont tuit apresté d'issir as chans. Les dames de \placeName{Rome}, ou il en i avoit plus de bieles qu'il n'euist el remanant de l'empire, avoient esté requisses de l'empereris que elles issir devoient as chans en lor chars. Dont il i eut teil bruit et teil friente qu'il ne fust nus qui a grant merveille ne tenist le grant honor des noble chars de \placeName{Rome}. Dont il avint ausi come des autres tous que l'empereris avoit fais faire \surplus{qu'elle avait fait faire} \num{v} chars et tous garnis d'orfaverie, liqueil fure\supplied{n}t aorné de pieres presieuses et coviers de fins dras d'or finement doublés dehors et dedens. Li cheval qui atelé i furent, ne cuidiés mie que il fussent mains riche come a ce apartint, de coi il avint que a celi jornee li empereris \del{li empereris} de \placeName{Rome} ne vot mie issir as chans ausi come por paroles. Mais \persName{Nera}, qui dou tout faisoit quanqu'ele cuidoit que bon fust, le fist entrer o li en son char ausi come nus ne seut qui elle fust, et ses autres deus serours se missent en lor chars, et li autre \corr{t}{d}oi demourent ausi come por ce que la roine d'\placeName{Aragon} n'i fu mie qui encuidoit que elle i deuist iestre, et li autre demoura por ce que il n'aferoit mie que l'empereris i fust a celi jornee.


\pnum \lettrine[lines=3]{\color{darkblue}L}{ors} fu eure que toute \placeName{Rome} ausi come tout comunement isirent de dehors la citei as chans. Qui dont veist les princes deseure dis coment chascuns s'estoit mis d'une part por lui mius mostrer, dire peuist que nule autre chose ne fust plus biele a veoir. \persName{Peliarmenus}, qui mie ne se volt metre arriere, se mist premierement hors de la citei, o lui maint noble chevalier. Apriés issi \persName{Dorus}, qui mie mains n'avoit de suite. Li tiers fu \persName{Japhus} li Fris, li quars \persName{Josias} d'\placeName{Espaigne}, li \corr{quisis}{quins} \persName{Hedipus} de \placeName{Galylee}, li sisimes \persName{Bourleus} de \placeName{Lenborch}, li sietimes \persName{Karus}, li dus de \placeName{Nisse}, li witimes \persName{Mirus} li Fiers, li nuevimes \persName{Daphus} li Gris, li disiemes \persName{Rebiers} de \placeName{Flandres}, li onsimes \persName{Diomarkes}, li rois d'\placeName{Aragon} et li dousimes fu li emperere et li dus \persName{Bourleus}, qui maint noble baron eure\supplied{n}t aveuch aus. Au dos les sivoit li grans bruis des dames nomees devant. Si ne covient ore mie de toutes faire mension, car trop i aroit a dire, si ne veil d'autres faire mension que des \num{iiii} serours et de la fille a l'empereour \persName{Kassidorus} que on deuist avoir nomee devant. Mais cele s'en maintint si simplement por ce que il li menbroit de la mort sa bone mere, que tous jors eut le larme a l'uel, mais de çou qu'ele peut se confortoit si con il avint que feme a tost trové joie quant aucuns le conforte enviers cui elle a parfaite amor si come ceste eut enviers son signor \persName{Leum} qui a li vint a la requeste d'aucune ame qui le volist metre en joie, et dist : <<Dame, \foliomar{4ra}{https://gallica.bnf.fr/ark:/12148/btv1b10023851v/f13.image} \foliomarID{4r} queil chire faites vous ! Donés moi vostre guimple, car por l'amor de vous, vorai chevalerie faire.>> Quant elle entendi son signor, si mua coulor et dist : <<Sire, encore me povés vous bien metre en grignor pensee que je ne sui.>> Adont tendi \persName{Leus} son brach et prist sa guimple desor son chief et le a fait metre en som son hiaume. Lors envint a sa gent qui l'atendoient. \persName{Peliarmenus} et \persName{Dorus} avoient fait pluisor rens aprester, par coi li uns se devoit asaiier a ferir en la quintainne, li autre a lanchier de dars en une estace et li tierc a joster au droit d'armes par le cheval. Mais sor tous ces embatimens, \persName{Peliarmenus} et \persName{Dorus} s'acorderent d'un tournoiement a l'endemain que li \num{iiii} frere \corr{tornoiement}{tournoieroient} par acort contre toute l'autre partie. Ceste noviele vint a l'empereour, qui mout s'en esjoï, et dist entre ses dens : <<J'aie mal dehé se je ne voroie volentiers\fnnote{Signe qui ressemble à un tilde sur le premier «o», mais C ne contient jamais «vonlont-». Ce n'est pourtant pas impossible. Confusion avec “voulont-” ?}, mais que je ne m'en doutasse d'aucun mescief que avenir i poroit.>> En çou qu'il pensoit a çou, li dist \persName{Borleus} li dus : <<Sire, que dites vos des damoisiaus qui a ce se sont osfiert ?>> -- <<Sire, dist il, je me douteroie que il n'en avenist autre chose que bien, car s\supplied{i} sai l'une partie et l'autre plainne de chevalerie que mout \supplied{e}n aroit fait d'armes avant que li uns ne li autres i presist se pau non.>> -- <<Verité avés dit>>, fait \persName{Borleus} li dus.


\pnum \lettrine[lines=2]{\color{darkred}Q}{ue} vous iroie ore disant de cesti chose ? N'i avoit nul qui ne s'acordast a tornoiement mais çou qu'il n'estoit encore mie ordené, et il veoient devant aus les grans esbatemens dont jou ai desus parlé, et d'autre part les dames s'estoient mises es rens a veoir les mius faisans. Dont il i avoit ja maint cop ferut d'uns et d'autres dont li contes ne fait mie mension, fors d'aucuns por le plus biel raconter lor avenues. Si comence a \persName{Leum} qui plus savoit dou cheual que li auqin ne cuidassent. Il coisi une estace qui ert enmi un camp fichié qui bien avoit deus piés d'esquarie, et fu de caisne dur et tenant. Il s'adreça cele part ferant des esporons, dont maint p\supplied{l}uisor misent lor entente a lui veoir, se n'i eut nul qui mout ne prisast sa contenance, car de teil force et de si grant vertu l'enporta li diestriers sor coi il seoit, que d'un dart que il tenoit, il a l'aprochier de l'estace entrepassant le lança de teil vertu que il le trespierça parmi en teil maniere que li fiers parut une paume au dehors a l'autre lés. Cest cop virent maint baron qui ce tinrent a mervilleuse proeche. Apriés cestui si essaiierent pluisor qui ce ne peurent ataindre que \persName{Leus} fist. De la se parti \persName{Leus} et vint a une quintainne ou li pluisor avoient si malement falli que l'en disoit que nus preudom, se il ne vosist que on se mochast de lui, ne se deuist asaiier, car li contes dist que cele quintainne estoit en teil manier fremee que elle toudis tornoit, par coi il covint, qui faillir n'i voloit, que on venist sor li si a point autour, par coi on fresist en la clef si a droit que la forche dou cop et li venue del cheval mesist en pieches le frasne de l'espiel. Li auquant ou plus avoit de proecce ne s'i vorent esaiier, por ce qu'il virent et seurent que nus nel peuist brisier par proecce se il en soi n'en euist la mesure ; et cil \persName{Leus} i vint, qui mout avoit de s'entente aministree et mise a sifaite chose, dont s'avisa coment il i peuist ferir, par coi li pluisor ne l'en peuissent mochier. Il prist un espiel fort et trenchant et puis prist tiere a son chois, et il avoit cheval a son voloir ; et tuit li plus grant signor vinrent a cest cope, qui grant entente missent a lui veoir. Et cil se mist a cheval poindre et vint les menus saus cele part. A l'aprochier qu'il fist hasta le diestrier, si se joinst en ses armes mout ameneviement, si q\supplied{u}'a l'aprochier qu'il fist feri si adroit en la clef que li fiers rompi et li frasnes \colmar{4rb}\colmar{b} vola em pieches, si s'enpenssa outre mout afaitiement sans lui desordener. <<Beneïçon aiie de Diu, dist chascuns, come ci a grant mervelle que cis a fait çou que tant bon chevalier i ont failli !>>


\pnum \lettrine[lines=3]{\color{darkblue}P}{eliearmenus}, qui bien avoit mise s'entente a ce que il avoit veü \persName{Leum} faire, s'est abandounés au ferir en la quintaine, mais tout ausi come li aucun i avoient failli li covint faire faute, si en fu trop iriés, car tout ausi come li autre en fu escharnis. apriés \persName{Peliarmenus} mist \persName{Japhus} s'entente, mais ce fu por nient. Tout autresi fist \persName{Josias} et \persName{Dorus} por porter compaignie a \persName{Peliarmenus} et douner \persName{Leum} l'onor de ceste emprisse. Dont il avint que li peres de lui li vint et li dist : <<Biaus fius, or n'avés vous nient fait se vous n'i ferés encore, car chascu\supplied{n}s dist que ce fu ausi come par la mescheance de la quintainne que vous i adierchastes. Mais faites le bien, que vous encore i puissiés ausi a droit ferir come vous avés fait. Dont dirons nous que vous en avés l'auwe copee\fnnote{La similitude avec l'expression bien connue \textit{couper l'eau} est trompeuse. Ici, l'\textit{auwe}, c'est l'oie. Il s'agit d'une expression proverbiale, très commentée bien plus tard chez Pierre Pathelin : en cotexte, il s'agit de "réussir de manière habile à qqc.". Il s'agit d'une pratique-jeu qu'on retrouver surtout dans le Nord (plusieurs témoignages dans des textes wallons). Voir Roques Mario. « Copper l'oe ». In: Comptes rendus des séances de l'Académie des Inscriptions et Belles-Lettres, 84ᵉ année, N. 5, 1940. pp. 395-401.}.>> -- <<Sire, dist \persName{Leus}, je ne sai que vous ne autres dirés, mais j'aim mius qu'il atant demeurt que jou encore i couruse, et pus se m'i covenist faillir, par coi chascuns de vous cuidast que jou n'i seuisse autre avantage que li uns d'iaus. Mais por çou que je ne veil mie escondire vostre proiere ne la lor, il s'i asaient li aucun qui drechier le fisent, par coi, s'il avient que il i faillent et jou apriés i fail, que tot soit trufe et que nul ne puist teil chose faire fors par enchanterie.>>

\pnum Quant \persName{Leus} eut ce dit, n'i eut celui qui ne l'en tenist a avisé, si ne vorent laissier qu'il n'esprovassent son sens. Lors se remissent au ferir en la quintainne et il n'i eut nul qui n'i fausissent plus vilainnement qu'il n'avoient fait devant. Dont quant ce vit li emperere, cui on avoit de sa chose contee ou il entendoit a autres esbatemens que li auquant faisoient par la campaigne ou il avoit si grant peule asamblé, que s'en tout l'empire n'euist plus de gent ne deuist on demander ou li remanans fust\fnnote{Cette phrase n'a pas de principale, mais je ne vois pas comment régler le problème : la digression qui suit est si longue que le rédacteur (ou le copiste) oublie que la temporelle n'a pas de principale...}. Por coi il me covient ciertefier que vous sachiés qu'en pluisor lius on faisoit de diviers esbatemens, par coi cis peules s'ensounioit li uns amont et li autres aval. Et por cesti raison que je ne puis ore mie faire de chascune chose mension, me covient traitier des grignors et venir a ce que \persName{Leus}, qui encore avoit le plus estragnement jué que tuit li autre, si come li escrit s'i acorde. Me covient\fnnote{Une nouvelle fois la syntaxe ne tient pas debout ! Il semble que le rédacteur (ou le copiste) oublie qu'il a écrit une première fois 'me convient ... venir a ce que', et l'écrit une deuxième fois, sans avoir terminé la première partie de la phrase.} a ce que li plus grans fais des barons et des dames furent venut a ceste merveilleuse quintainne, dont li mius entendant savoient que par proece n'i valoit rien sans avis. Et qu'en fist \persName{Leus} ? Il demanda le plus fort espiel que on peut trover, et li dus ses pere vint a lui et li dist :

\pnum <<Biaus fius, ne metés or mie si grant painne a la quintainne metre a tiere que vous i failliés.>> -- <<Ne vous doutés !>> dist il. Atant li fu uns frasnes aportés, qui par samblant li deuist iestre mout fors, mais cil ert entalentés de parfurnir ce qu'il avoit en propos. Lors l'a pris, voiant maint baron qui tuit missent lor ententes coment il se maintenroit. Mais tout ausi qu'il n'i acontast pau u nient, empuigna l'espiel et mist a sa volenté, et apriés s'est aprestés et se mella en ses armes et el cheval au plus afaitiement qu'il peut, come cil qui bien savoit que tuit et toutes l'esgardoient, et il s'i \surplus{s'i} maintint si noblement qu'il en eut la grasce de tous et de toutes. Au conbrer le cheval f\del{u}ist il merveille, car si le mist en un mont qu'il fu avis a chascun qu'il se deuist lanchier en l'air, \foliomar{4va}{https://gallica.bnf.fr/ark:/12148/btv1b10023851v/f14.image} \foliomarID{4v} et apriés ce s'est afichiés es estriefs de si grant forche que li diestriers en archoia desous lui, que mie n'ert petis ne foibles. Lors se mist au ferir des esporons petit et petit et li chevaus l'enportoit sor frain de si merveilleuse maniere que tuit cil qui l'esgardoient en eurent grant joie, et quant ce vi\supplied{n}t emi liu de la voie, si ne le hasta ore mie petit. Anchois li lasqua le frain et le comença a avencer des esporons, et il s'eslance et comença l'air a trespierchier, dont il ne fu nus qui mie peuist soushaidier ne penser\fnnote{Usage et construction de la négation à commenter}, sitost come il vint a la quintainne et feri de teil ravine et de si grant force en la clef que, vosist u non, covint la quintaine vierser et metre toute em pieches, si que a ce qu'il enpassa outre volerent li trons amont, et ce qu'il l'en demoura en la main jeta si haut que grant merveille fu dou veoir. Si que anchois que li trons fust cheois de l'air amont, sacha il sus le diestrier et retorna cele part, si le requelli de trop grant apierté, si repaira par l'estace qui rounpue estoit et i feri un cop si grant qu'i fist son trons voler \del{i} en \num{ii} pieches.

\pnum Ceste jouste virent mainte noble piersone qui mout en orent grant merveille. Meime li emperere dist que de toutes les apiertés qu'i\fnnote{Il faudra prendre une décision et faire une note en introduction pour cette forme de 'qui' équivalente de 'qu'il'...} avoit veut fait en armes, il n'avoit veüe sa pareille. -- <<Sire, dist li uns, ne cuidiés mie qu'il fust hom vivans qui ce peuist faire se il n'avoit cheval a sa volenté.>> -- <<Ne m'en chaut, dist li emperere, dou cheval. Bien puet on savoir que cheval covient avoir a ce faire. Mais jou aucune fois feru en quintainne et ai veut ferir, dont je vous aseur que, tout peuist on ferir de tous lés sans mesprendre, si avés vous la grignor proece faite si come hom de vostre eage et de vostre taille, que jou onques mais veisse, et ne mie\fnworknote{syntaxe de la phrase à revoir} sans plus de tout ce que je ne prise \uncertain{ps?} autant çou que vous revenistes au sachier sus a vostre trons si a point que vous le recuellietes en teil maniere.>> -- <<Sire, dist il, ço prenderoie je a faire tous les cos, mais a ce que \surplus{que} je ai la quintaine \add{^misse^} a tiere me sent jou un poi bleciés, si vous pri en gueredon de ce que jou ai fait un poi de chose que vous avés veü volentiers, que vous ne vous desaïriés\fnnote{Quel est ce verbe ? Visiblement un composé de 'aïrier', se mettre en colère. Mais je ne trouve nulle pas un composé avec le préfixe des-... On retrouve la rime 'desairier / repairier' d'un des des deux exemples cités par Tobbler-Lömmatsch... : abandonner / retourner dans sa patrie. Je ne pense pas que le sens signalé par Godefroy puisse convenir dans le contexte} mie de ce qu'il m'en covient repairier a l'osteil, car je me sent un poi bliciés.>> -- <<Ha ! \persName{Leum}, biaus fius, come je le cuit bien.>> Dont se traist priés de lui et le prist par le frain, si l'en mena hors de la et vierti li emperere viers \placeName{Rome} ausi come de ce ne fust riens. Li dus ses peres s'en pierçuit, tuit ausi ont fait aucun qui se sont apriés mis. <<Ha ! Sire, dist \persName{Leum} a l'empereour, por Diu merci, retornés ! I alairent lor joie et lor fieste, se il se pierçoivent que nos soiiens issi d'iaus partis.>> -- <<N'aiiés doute !>> dist li emperere. Atatant s'asambla li dus a aus, et dist : <<Coment, biaus ! Que vous est avenut ?>> -- <<Biens, se Diu plaist, dist li emperere, si est un poi estors a ce faire\del{re} que vous avés veut.>> -- <<Ha ! Sire, dist li dus, come jou li avoie bien dit a l'emprindre si fort espiel, coment sui je ore venus au chastiier.>> -- <<Nennil, biaus amis, dist li emperere, ne vous courciés mie. ``Car ci n'atient nus chatois.''>> -- <<En non Diu, sire, dist il, non, car il est atart. Je me muir.>> Dont le covint illuec pasmer \num{ii} fois sor le cheval.


\pnum \lettrine[lines=2]{\color{darkred}V}{ees} ici grant meschief et cruel anui estrait de si grant joie, car nos trovons el conte qu'il covint \persName{Leum} morir avant qu'il peuist iestre ramenés en la citei. De coi il \supplied{avint} que, quant li dus seut çou, il dist a l'empereour : <<Ha ! Sire, ci n'a point de recouvrier. Et qui d'un damage feroit \num{ii} ne \num{iii} ne \num{iiii} tant vauroit piis.>> -- <<Amis, dist li emperere, mout parlés sagement. \colmar{4vb}\colmar{b} Or wardés que vostre parole ait fruit par coi je vous tiengne a sage.>> -- <<Sire, dist li dus, pau voit on d'oumes qui doient iestre ausi iriés come je doi, qui puissent dire sens ne faire. Mais tous si iriés come je sui, je loeroie por le mius que nus autres que vous ne seuist huimais ne demain cest meschiés, car de cestui damage poriens nous grignor avoir qui ne le vorroit faire savoir en point et en liu, car je sui tous chiertains que se ma dame vostre fille le savoit, que je de mon fil n'ai que je n'atendisse de li\fnnote{'je ne ressens aucun sentiment au sujet de mon fils que je ne m'attende à trouver chez elle'}.>> -- <<Et par mon chief, dist li emperere, je vous tieng a sage.>>

\pnum Ensi avint de cesti chose que on desfendi sor cors et sor quanque nus povoit meffaire, qu'il ne fust nus si osés qui represist cestui afaire en liu ou plus de gent le seuissent. Lors fu mout soutiument ensevelis et apareilliés en un osteil dedens \placeName{Rome} por le mius celer, por coi raison il me covient ore venir a ce que, quant \persName{Leus}, ensi come jou ai dit desus, ot la quintainne misse \del{en} a tiere, ensi come vous avés oï, n'i eut nule des gentius dame dont j'ai desus fait mension qui ne fussent mout esbahies de cesti avenue. \persName{Nera}, qui sor toutes avoit le parler, en main enbracha \persName{Kassidoire}, qui feme fu \persName{Leum}, et li dist : <<Coment, suer ! Mout avés hui fait souple chiere. Or faites bone et si aiiés joie en vous, car de plus noble vasal ne sai dame honoree au jour d'ui.>> -- <<Ha ! Tres chiere dame, dist elle, merci. Or primes me samble que jou aie la pointe du coutiel qui au cuer me doie ferir.>> Dont n'i eut nule d'eles qui ne le tenist a grant folie, et dist chascune : <<Je ne sai si grant corous en moi que, s'il ert avenut a mon signor l'ounor qu'il est faite au vostre, que je ne feisse toute autre chiere que nous ne veons que vous faciés.>> Dont se reprist en soi et seut que ele mesprendoit selonc avis de chascun et de chascune. Mais li cuers, qui par nature ne le pot esjoïr, atendoit çou que Nostre Sire li eut proveu. Lors se cuida esforchier de joie faire, et dist : <<Ore cil que ceste proece a faite, ne venra il mie avant ?>> -- <<Par foi, dist chascune, de ce avons nos merveille mout grant, car il n'est mie usages que, quant on a fait teil chose c'on doit tenir a proece, que cil ne se doie mostrer devant les \del{m} dames et le damoisieles.>> A cest mot est venus li rois d'\placeName{Aragon} et autre baron qui fusent avisé, et disent : <<Dames, ceste quintainne est mise a mort. Or avant, une autre !>> De la endroit furent menees a une autre ou il avoit maint biel cop ferut, mais si forte estoit que nus ne le peuist metre a tiere. Quant \persName{Peliarmenus} s'aati de li vierser coi qu'il l'en deuist avenir, a cest cop fu li emperere repairiés et vit que \persName{Peliarmenus} s'aprestoit de ferir en la quintainne. Si saisi l'espiel ou il le tenoit, et dist : <<Biaus fius, n'est ore mie grans mestier d'esvoiturer chose vous ne autres viegn\add{^i^}és tart au repentir ! Mais huimais si hardi qui se painne de quintainne metre a tiere, car je ne veil mie qu'il i ait home qui soit paraus \del{s}a \persName{Leum}, mon chier fil, qui le los et le pris doit avoir des quintainnes a la jornee d'ui.>> Quant Peliarmenus oï çou, si fu trop dolans, car bien cuidast esvoiturer çou qu'il voloit entreprendre, si n'en osa plus aler avant.


\pnum \lettrine[lines=2]{\color{darkblue}D}{ont} s'en vint li emperere devant les dames et les damoisieles, qui toutes enquissent a un ton : <<Sire, c'avés vous fait de nostre millor chevalier, qui a çou fait que nus autres ne peut faire ?>> -- <<Par celi foi que je doi a vous toutes, je l'ai fait mener en teil liu ou nule de vous ne le verra huimais ne demain. Mar a ma quintainne \foliomar{5ra}{https://gallica.bnf.fr/ark:/12148/btv1b10023851v/f15.image} \foliomarID{5r} mise a tiere sans mon grei.>> Quant ce ont entendut, les dames si ont eut grant joie, car elles virent et cuidierent que çou fust toute joie et toute fieste, si come il lor douna a entendre. Et çou fu une trop biele voie por covrir çou qu'il avoient entrepris. Ensi coumença li fieste et li esbanois a demourer en sa viertut, et les dames furent comandees a mener en un rens qui fu avisés a jouster. Si avint \persName{Japhus} li Fris et \persName{Josias} d'\placeName{Espaigne} devoient par acort jouster li uns a l'autre. Lors vint li emperere a aus et lor dist : <<Biau signor, n'i ait nul de vous, s'il ne veut pierdre l'amor de moi, qui s'abandoinst a nule vilainne emprise ! Anchois chevachiés sagement, car trop doit on douter les perius.>> Dont n'i eut nul qui ne s'avisast a faire sa volenté. Et qu'en avint ? Il brisierent si bien et si biel que mout fu grans joie dou veoir et grans deduis dou raconter, qui arait tans ne liu dou faire. Mais ausi come chascuns si est mie disposés a oïr çou c'on puet et doit briement conter, m'estuet venir a ce qu'il m'est avis que on ne repuet mie si corrumpre sa matere, que cil qui volentiers l'oent me puissent blasmer, por coi raison je veil a çou venir que \persName{Dorus} vint a l'empereour, son pere, et li dist : <<Sire, il me samble que vous aiiés ausi come desfendut que nus ne joste a moi.>> -- <<Biaus fius, dist li emperere, sauve soit vostre grasce, anchois vous seit si plain de vostre volenté qu'il vous redoutent.>> -- <<Et coment ? dist il, ne demeure il por autre chose ?>> -- <<Ensi le cuit jou, dist li emperere.>> -- <<Et \del{m} par mon chief, dist il, je n'i ferai hui mais cop d'espiel ne de lance ?>> -- <<Vien avant, fol ! dist li emperere. Vieus tu faire chose dont on parot en bien et dont tu seras mius honorés que se tu avoies abatues toutes les quintainnes de ceste fieste ?>> -- <<Sire, dist il, or me laissiés dont oïr.>> -- <<Volentiers, dist li emperere. \persName{Leus}, cui serour \del{se} tu as, si est un poi bleciés, por coi il ne puet mie a la jornee de demain porter armes. Si loeroie que vous fussiesiés\fnnote{Curieuse terminaison d'une P5 subj. imp. de «estre» qu'on retrouve à «mostrer».} armés de ses armes et la mostresiés combien vostre se puet estendre.>> -- <<Sire, dist \del{\persName{Leus}} \persName{Dorus}, de ce que \persName{Leus}, mes frere, est bleciés me doit il anoiier, mais que je demain soie por lui a tornoi me ferai je a merveille joiant !>> Lors vint \persName{Helcanus} et li cuens de \placeName{Flandres} a l'empereour et dissent : <<Sire, qu'avés vous fait de \persName{Leum} ?>> -- <<Biau signor, dist il, jou en ai fait ce que vous povés oïr.>> -- <<Sire, dist dont \persName{Helcanus}, nous nos doutons qu'il ne soit blechiés.>> -- <<Biaus fius, dist li emperere, or ne vous en doutés ja, car il est autrement que vous ne cuidiés. Et si n'enquerés huimais de lui autrement que jou ai dit \persName{Dorus}, vostre frere.>> Dont \add{^se^} vint li cuens de \placeName{Flandres} vers l'empereour, et dist : <<Sire, de ceste jornee ne \fnfpz{Lire «por» ici pose un prb. «puis» ?}\corr{por}{pus} jou veoir qu'il ne soit bien eure del repairier en la citei, selonch çou que jou entenc, a demain ¶ le tornoi.>>


\pnum -- <<\lettrine[lines=2]{\color{darkred}S}{ire} cuens, dist li emperere, or avés vous voir dit.>> Lors fist li emperere souner le retraite en la citei, et il ausi ordeneement come il en issirent i sont rent\supplied{r}é\fntooltip{V3}. Il fu tans et eure que viespres durent soner, et \add{^on^} si fist, et puis les sont \fnfpz{à corriger en «alees» en suivant V3 ? Ou bien, si le fait se reproduit (participes passés en –er), le signaler en introduction linguistique.}aler oïr li emperere et li baron qui apriés vinrent tout a court ; si fu eure que on deut souper, si le fist on plus tempre, por ce que li aucun avoient pau mengiet en tout le jor. Qui dont veist coment li keu s'estoient pené del soper noblement entremis, dire peuist que autresi bien se penaissent d'iaus faire loer come li chevalier de chevalerie faire. Ensi avint que il fisent l'iauwe corner, et li emperere et li empereris laverent, et puis tuit li baron qui apriés l'empereour et l'empereris sont assis a lor droit. Cil qui de siervir se deurent entremetre le fissent si ordeneement que mout fu biele chose a veoir de ciaus qui a teil mestier se conissoient, et sans faille, qui qui le tiengne a trusfe, biele chose est en osteil de grant signor de siervir ordeneement, \colmar{5rb}\colmar{b} non mie selonc ce que chascuns vaut mais selonc ce qu'il apartient au signor.

\pnum La ne fu nus boutés hors de la court puis qu'il i fu entrés dedens. La ne fist on nului lever dou mengier puis qu'il i fu \del{entrés} asis. La peuist on savoir et aprendre toute \missing\fnnote{Il mqanque de toute évidence un mot ici.} et cortoisie d'oisteil a maintenir, si que li pluisor fissent qu'il en retenance le missent. Mout dura cis soupers et fu parle\del{t} sor table dou tournoiement a l'endemain. Mais atant me veil partir de cest souper car assés eurent selonc çou que j'en ai traitié, si que on apriés çou que ce fu fait li pluisor se sont mis et mises au caroler. En cestui point \persName{Nera} et \persName{Kassidoire} s'en sont venues a l'empereour et le misent a raison de ce qu'il avoit fait \add{^de^} \persName{Leum} \del{empereour}. Li emperere s'avisa a çou qu'il dist : <<Ma biele fille, li une et li autre, il est usages en osteil de grant signor, ausi come doit iestre a l'empereour de \placeName{Rome}, que, quant il est uns chevaliers, queus qu'il soit, et il a fait çou que \persName{Leus} avoit fait, que il doit iestre mis d'une part tant que raisons en soit faite, ausi come je vous avoie\del{e} jehui dit.>> -- <<Sire, dissent elles, nos ne cuidons mie que \persName{Leus} ait faite chose dont il doie blasmes rechevoir.>> -- <<Par mon chief, dist li emperere, que je ne cuit mie que se il l'avoit fait que je n'en fusse ausi iriés come nule de vous seroit. Mais vous savés bien que l'usage dou paiis covient tenir et aquerre.>> Atant se drecha li emperere, por ce qu'il ne volt mie que plus le tenissent au parler de cestui afaire. Si prist l'une a une main et l'autre a l'autre as charoles, si mist paine a ce que il et ses \num{ii} filles dissent cest rondet :
\begin{verse}[\versewidth]
Mout vaut mius amener joie q'iestre trop souplet.\fnfpz{le terme souplet, triste, apparaît déjà plus haut.}
\end{verse}


\pnum \lettrine[lines=2]{\color{darkblue}Q}{uant} li emperere et ses \num{ii} filles eurent lor chançon finee, mout en i ot qui puis se penerent de chançons dire et de joie faire. Meime li empereris s'i si\fnnote{doublon ? ou faut-il comprendre l'un des deux comme "s'i" : "si s'i fu mout corte tenue" : "fut alors ("s'i") tellement ("si") pressée de chanter", "tellement invitée à chanter"} fu mout corte tenue de chanter et elle ne se volt faire tenir a fole ne a nisce, anchois pria ses \num{ii} autres filles qui fure\supplied{n}t fames as \num{ii} damoisiaus de \placeName{Rome}, dont elle ert mere, que elles disent cest vier de chanchon mout joliement. 
\begin{verse}[\versewidth]
Ne se doit desconforter, qui en vraie amors erent. \\*
Rondet, ne te veil noter, d'autre chant que raconter. \\*
Coment amor s'amonte, veulent chiaus de lor covent. \\*
Ne se doit desconforter, qui en vraie amors erent.\fnnote{Le respect du mètre ne préoccupe pas beaucoup le scribe. Y a-t-il un message à tirer de ce rondeau chanté, à la demande de \persName{Fastige}, par ses deux belles-filles, les épouses des héritiers du trône de \placeName{Rome}, à la fête de ces noces renouvelées avec Cassidorus ?}
\end{verse}
Quant li \corr{emperere}{empereris} et ses \num{ii} filles eurent alé lor tour, si fu eure d'aler reposer selonc çou qu'il eurent a faire a l'endemain.


\pnum Atant se sont li baron trait as osteus et prisent repos, li uns plus et li autres mains. De çou il avint que \persName{Dorus} eut mout a penser qu'il peuist a l'e\supplied{n}demain faire chose dont li emperere se tenist apaiiet. Il avoit fait venir les chevalier \persName{Leum} devant lui et lor avoit dit coment il voloit armes porter en la samblance de \persName{Leum} lor signor, et qu'il ne s'abaubesisent mie, car il ne descroisteroit mie son pris, mais qu'il i tenist metre sa vie. Il li avoient dit que de çou ne s'esmaïst mie, il li seroient sovent et menut \supplied{et} priés, mais gardast qu'il fuist montés as guise. Il dist que de çou ne se doutoit il mie, que il avoit teil cheval que il ne cuidoit mie que il en tout le jor li deuist faillir, se il ne li failloit lui. Ceste parole oï uns Hainnuhiers, qui mout prisa le mot que \persName{Dorus} avoit dit, et dist : <<Sire, se jou estoie ausi riche hom et de la vostre valor come vous iestes, mout aroie chier un teil cheval \add{^et^} mout l'ameroie.>> Adont esgarda \persName{Dorus} celui, si le vit aorné de toute biauté et de jone aé, et dist : <<Et jou veil que vous l'aiiés, et puis me dirés qui vous iestes.>> Dont ne s'abaubi mie cil. Anchois se traist viers lui et li dist : <<Ha ! Sire, la vostre franchise ne se puet covrir, mais ma fole covoitise si m'a descouviert a ce que vous avés oït.>> -- <<Bien oï, dist \persName{Dorus}, a vostre raison \foliomar{5va}{https://gallica.bnf.fr/ark:/12148/btv1b10023851v/f16.image} \foliomarID{5v} que vous dou tout n'iestes mie a prendre. Mais vostre non veil jou savoir et en queil liu vous fustes nés.>> -- <<Sire, dist cil, je fui nés en la marche de Hainnau, fius a un vavasor de povre non et d'asés petit afaire, et je meime sui apielés \persName{Hainnaus}.>> -- <<\persName{Hainnau}, dist \persName{Dorus}, je vous pri que vous me soiiés priés a l'enconmencier dou tournoi.>> Dont l'en volt cil aler au piet quant \persName{Dorus} l'embracha et dist : <<Alé huimais a l'osteil, car ausi en est il tans.>> Adont prist chascuns congiet et s'en sont atant parti. A l'endemain eut grant bruit aval \placeName{Rome} ou chascuns s'aprestoit de çou qu'il peurent selonc çou qu'il furent. Si ne puis ore mie raconter son erement, car il n'est afaire ne raisons ne l'ensaigne mie. Anchois veil venir a l'empereour et au duc \persName{Borleum}, qui ne s'atirent mie de porter armes a celui jor. Lors fissent venir tous les grans signor qui armes devoient porter a celi jornee. Se n'i eut mout de ciaus de coi jou n'ai encore fait mension, car il vint noviele que uns damoisiaus de \placeName{Pulle} qui avoit esté fius au noble \persName{Sinador}, qui jadis avoit esté compains d'armes au boin seneschal de \placeName{Rome}, icil fu apielé \persName{Galiens}, et uns autres qui fius estoit au prince de \placeName{Calabre}, qui avoit non \persName{Pierchemons}, cil doi avoient esté nori ensamble, car il estoient cousin et s'entramoient mout, si come cil qui n'estoient encore mie chevalier, si avoient en covent li uns a l'autre qu'il ne seroient ja chevalier se dou millor non des millors, et il avoient enquis et demandé que on lor avoit fait a savoir \add{^ke li emperere^} de \placeName{Costantinoble} avoit le non de deseure dit. Dont il avint que li tornoimens fu respités de ci a l'endemain, que cil i porroient iestre plus covignablement.


\pnum Li emperere et li dus, qui ne peurent mie lor afaire et lor anui covrir come une fine merveille, eurent conseil qu'il au plus priveement qu'il porroient meteroient \persName{Leum} en tiere, dont il avint por la fille a l'empereour qui sa feme estoit, que on fist a entendant que çou ere uns povres chevaliers qui ere au conte de \placeName{Flandres}, et por lui honorer, on en fist la plus grant joie que on peut, si qu'il fu mis en tiere droit de devant l'auteil saint Michiel en la grant eglise de \placeName{Rome}, et la le puet on veoir chascuns qui garde s'en vient douner. Ensi avint que ceste chose fu celee, por ce qu' ``il covint covrir au plus c'on puet son anui ou il ne puet avoir recovrier nient autrement c'on fist a cestui.'' Et li plus grans raisons dou celer, ce fu que, quant on mue une tres grant joie en parfait duel, trop en puent de peril \add{^a^} venir et maiement d'arme et de cors. Si ne veil ore mie ci dire toutes les raisons, mais il atant vous en sosfisse, fors que de tant que il i eut encore une autre raison que je mie ne tieng a petite, car se tout seuisse\supplied{n}t le mechief quant il avint, n'euist rien valut la fieste ne la joie come avoit emprise a l'onor de l'empereour et de l'empereris, car joie faite en tristece n'\corr{estre}{est} autre chose que blankes chauces noires\fnnote{Cette expression, qu'on rencontre encore au § 250, n'a pas été trouvée. Sur le substantif, nous hésitons entre \textit{calx} ou \textit{calceus} ; ce dernier est entré en composition de beaucoup d'expressions en français (FEW, 2.70b). La deuxième occurrence est : \textit{en celui val si avoit une cité qui seoit en une montaigne, et por ce que qui ne feroit autre devise, si sambleroit que ce fussent unes chauches blanches fussent noires, et ce ne seroit or mie une mout grant merveille}. On retrouve dans les deux cas l'idée d'un dévoilement grâce auquel le narrateur, dans une invitation faite au lecteur de ne pas se borner aux apparences et à la surface des choses, (la joie ressentie dans un moment de tristesse est feinte, une cité qui ne paye pas de mine au premier regard est trompeuse), en dénonce le caractère limité ou trompeur.}, et por ces raisons et por autres fist li emperere et li dus cest afaire, issi com vous poés oïr.


\pnum Li damoisiaus de \placeName{Puille} et cil de \placeName{Calabre} chevaucierent tant et a si noble compaignie qu'il sambloit a tous ciaus qui les virent que tous li mons deuist iestre enclins a aus, car je truis escrit qu'il furent en lor compaignie \num{xxx} chevalier, dont li \num{x} portoient baniere si que tiaus menerent teil harnois q'a l'entree en \placeName{Rome} lor veut on veoir l'entree. Quant novieles en vinrent a \placeName{Rome} a \persName{Gasum} le seneschal, \colmar{5vb}\colmar{b} cil en fu trop iriés et coumanda et dist que li chief de \placeName{Rome} n'ert mie abaubis de membrer qu'il deuist avoir. Dont entrerent en \placeName{Rome}\fnnote{Ici, amalgame d'un e et d'un a avec barre de nasalisation. Le scribe a corrigé «à» par «en», on garde «en».} a grant harnois, come cil qui faisoient traire les plus biaus chevaus apriés eus dou \del{ci} monde. Si qu'il avint que Romain\del{s}, qui mout sont covoiteus, disent : <<De ciaus ci nos covient avoir !>> Li damoisiel, qui a grant bruit venoient par arriere, n'ont finet tant qu'il encontrerent les \num{iiii} fius a l'empereour et maint autre baron en lor compaignie, qui lor fissent mout grant joie et merveilleuse honor. Il entrerent en \placeName{Rome} et furent mout esgardé de tous, come cil qui bien et biel se maintinrent, si ne descendirent de ci au peron dou Palais Majour. Puis sont monté amont tout \num{vi}, tenant mains a mains, et issi vinrent devant l'empereour, qui mie ne s'abaubi des damoisiaus bienvignier quant il l'eurent salué. Que vous iroie ore faisant un conte de chose que li emperere lor demandast ne qu'il li respondirent ? Asés entendanment vous \del{i} ai dit por coi il venoient. Si ne furent mie nice de lor besoigne mostrer, ne li emperere dou respondre. Lors fu eure et tans c'on dut l'iauwe corner, si l'ont fait cil qui s'en durent entremetre, si que li emperere et li empereris et tuit en apriés li autre mout ordeneement et puis \fnfpz{ou 'plus' ? Mais de toute façon ça ne convient pas très bien, le dernier 'et' pose problème}s'asisent. Si ne truis mie ou conte que, se on avoit de devant nul jor siervi noblement que encor le fist on plus de grant signorie, si qu'il vint a mout grant merveille les mius entendans coment on peut si grant peule\fnnote{&gt;populus, vocalisation de cons. labiale [b] dans le groupe [bl] issu de [pl] : nord/nord-est.} siervir si paisulement ne a si grant aise. Cil qui b\supplied{ie}n couneurent\fnnote{Prb avec l'abréviation conueͬrẽt} l'iestre de ciaus qui çou avoie\supplied{n}t ordené respondirent : <<Biau signor, par un seul home est uns osteus retenus en honor, et par un autre est il deshonorés. Vees ici \persName{Gasum}, le seneschal, qui ne chace mie a metre les siens parens ne ciaus qui par dons aquerent les siervices des grans signor, qui ne sont dingne de sierv\supplied{ir}\fnnote{nous faisons l'hypothèse d'une absence de tilde vertical sur «u», ce qui permet d'avoir une forme infinitive, alors que «servi aus» n'a aucun sens.} iaus ne autrui. Mais tous les millors et tous les mius esprovés qu'il puet avoir ne tenir, ciaus acoint\del{e} il et a acointiés tous jors el siervice de son signor. Et par ce poés vos veoir le noblece de la court, ensi come vos le demandés.>> En cesti chose se deduisoient li aucun, non mie sans plus en boire ne en mengier, car il est a savoir as plus gentius homes qu'il sacent au jor d'ui\fnnote{On garde la forme analytique pour une oeuvre de la fin du 13e siècle : Marchello-Nizia, Christiane. « La sémantique des démonstratifs en ancien français : une neutralisation en progrès ? », Langue française, vol. 141, no. 1, 2004, pp. 69-84.} c'uns hom destruit un osteil, et uns autres le retient.


\pnum \lettrine[lines=2]{\color{darkred}E}{nsi} fu ceste chose deseure nomee siervie et honoree de tous, et avint que, quant on \del{d}eut siervit et les napes furent traites, \num{ii} menestreil s'abandounerent de viler sons prouvenciaus, et en la fin dissent une chaçon de court en chantant en lor vieles et de si noble maniere et de si mervilleusement boin estrument que tuit cil qui les entendirent n'avoient oï si bien chanter. A ce que il eurent lor chançon finee et il disoient lor issue, esteme vos\fnnote{La locution \textit{esteme vos}, de même que \textit{(atant) eme vos}, est fréquente dans les chansons de geste. Sur cette locution, qui signifie \textit{voici}, voir la Syntaxe.} un liuon qui estoit desloiiés du travail ou il ert mis por çou qu'il ne feist mal a chascun. Lors se sont tuit desairiet por le paour de lui. Li emperere, qui amoit le liuon et qui l'avoit fait amener de \placeName{Costantinoble} a \placeName{Rome}, ausi come cil est en l'istoire, sailli em piés et vint en la court aval ou il avoit un chevalier qui ne savoit mie que li emperere l'amast ensi come il faisoit. Cil avoit une espee et jete son \del{brac} mantiel entor son brac, et voloit movoir au liuon et li liuons a lui quant li emperere s'escria : <<Compains, compains, mar vous avenroit !>> Lors se mist li liuons a merci et envint viers l'empereour les menus saus, et puis se mist a ses piés ausi humlement come il peut mius. \foliomar{6ra}{https://gallica.bnf.fr/ark:/12148/btv1b10023851v/f17.image} \foliomarID{6r} Ceste chose virent li baron trop volentiers. Li emperere se baissa et le prist par le chief et li frota les oreilles, et dist ausi come il euist dit a un siergant : <<Je vous comant que vous a nului ne faites vilounie dont vous me puissiés courecier.>> Dont se dreça li liuons ausi conme il l'euist entendut et vosist faire sa volenté. Li emperere fist crier sor toute la court que nus ne fust si hardis que le liuon deist ne feist vilounie, s'il ne voloit pierdre la vie. De cesti chose se mervillierent li auquant, que dou tout ne seurent mie l'aventure dont li histoire fait mension. Si avi\supplied{n}t que li liuons sivoit l'empereour ausi come uns levriers suit son soverain signor. Si veil ore atant laissier a parler de ce liuon a çou qu'il covint ordener coment cil tornoimens peuist venir par acort, et que li une partie n'en euist tant le millor que se li autre se vosist desfendre qu'il ne peuissent iestre folé par nule male covoitise. Dont so\supplied{n}t mis ensanble cil ami il en atennoit le plus. Si trovons que li quatre \add{^frere^} furent a l'un lés, et li rois d'\placeName{Aragon}, \persName{Japhus} li Friis, \persName{Josias} d'\placeName{Espaigne} et li cuens de \placeName{Flandres} a l'autre lés. Et chascune partie si eurent chevaliers de grant non qui ci endroit ne seront mie nomei. Mais quant ce venra es batailles dou tournoi, adont porés oïr les mius faissans, si come çou iert raisons et drois, por coi nos vous volons ciertefiier le nombre de la chevalerie qui a cest tornoi s'armerent. Si t\supplied{r}ovons qu'il en i ot, que riches que povres, \num{xviii} cens et \num{l} et \num{vi} a l'un lés, en eut \num{xxvi} plus que il n'eut a l'autre lés, si que cil que le plus en eurent, ce fu li rois d'\placeName{Aragon} et cil de sa partie.


\pnum Cis otrois et ceste partie fu faite et confremee. Lors aprocha la nuis que li damoisiel qui chevalier devoient iestre s'amonstrent de devant l'empereour. \persName{Gasus}, dont nos avons de devant traitié en asés boine maniere, avoit un fil, que il a merveil amoit, por ce qu'il li sambloit que il avoit mout biel coumencement, et sans faille que si avoit il, car il ert biaus et bien tailliés de tous menbres, et jones de l'eage de \num{xxv} ans, et sages por un empire a maintenir, et gratieus sor tous. Icestui amena \persName{Gassus} devant l'empereour et li dist : <<Sire, vees ici un mien fil que je covoite mout qu'il vous peuist faire siervice, et autresi a vostre enfans que vous peuist plaire. Et por itant vous requier je que vous en faciés chevalier a l'honor de Diu et de Sainte Eglise.>> Li emperere esgarda le damoisiel, si vit q\supplied{u}'a grant biauté n'avoit il mie failli, et il li enquist son non. Il li dist qu'il avoit non \persName{Ganor}. <<\persName{Ganor}, biaus fius, mout sui liés de la requeste que vostre boins pere m'a faite.>> -- <<Dont, dist cil sire, ja Diu ne place que je ne puisse chose faire ains que je muire \fnfpz{Godefroy donne pour 'valoir' un sens de 'tirer avantage' qui pourrait convenir : 'dont vous puissiez tirer avantage' (ce 'que' marque une nouvelle fois un rapport lâche, une relation entre le verbe et le complément).}que vous puissiés valoir, si voireme\supplied{n}t que jou ja tous jors m'en veil metre en painne.>> Dont l'en vot aler au piet quant li emperere l'en leva, et dist : <<Amis, or soiiés priés au matin aveuc les autres.>> Apriés cestui en revinrent maint autre, dont li contes ne fait mie mension. Aprés çou fu eure de viespres, que li emperere et li baron les ont oïes, et puis vint li soupers et refu fais ausi come il covint. Li damoisiaus de \placeName{Puille} et cil de \placeName{Chalabre} et pluisor autres aveuc aus por lor amor alerent villier avec maint fil de prince a l'eglise de \placeName{Rome}, qui l'endemain furent apresté, quant li emperere eut oï mese en sa chapiele, qui a chascun douna l'ordon de chevalerie.


\pnum \lettrine[lines=2]{\color{darkblue}Q}{uant} ce avint que li damoisiaus de \placeName{Puille} et cil de \placeName{Calabre} et de pluisor mai\supplied{n}te autre region, si lor douna li emperere \colmar{6rb}\colmar{b} a tous garnimens, selonc çou que chascuns estoit. Gent noble compaignie quant il furent \num{iiii} vins et \num{x} noviel chevalier entre \num{xxx} ans et \num{xx}, dont li mains puissans porta baniere, ceste biele compaignie retinrent tuit li damoisiel de \placeName{Puille} et de \placeName{Chalabre}. Qi dont veist coment chascuns se traist a sa partie, dire peuist cest provierbe qui est escris ou latin, et dist en teil maniere : ``similis similem cuerit.''\fnnote{On retrouve l'idée structurante à de nombreux égards selon laquelle la ressemblance de comportement, de classe, de propriété, de talent est un facteur de cohésion voire d'émulation ici.} Ce fu a dire que chascuns frans hom se mist a sa franche labour et li autre, qui de çou cure n'avoient, se missent a l'autre lés. De lor mengier ne de lor boire ne me covient ore mie que je \fnfpz{C'est une graphie analogique des P1 de présent de l'indicatif habituellement : canto &gt; cant, chant. Pour computo, c'est un proparoxyton de réduction tardive : le -o final a reçu un accent secondaire qui l'a protégé fin VIIe au moment de la chute des voyelles finales, donc &gt; compte, conte et raconte. Par analogie donc, 'racont'.}racont, car je cuit que asés legier fussent asiervit, car chascuns ne chaçoit mie le ventre a emplir, mais a faire chose dont on peuist savoir que il, li siergant de Sainte Eglise, seroient contre ciaus qui adont le vausissent foler\fnnote{Comprendre avec V3 «fouler» > fullare et non «foler» > follis }. Teus estoit lor ordenemens au tans de dont. Je ne sai mie queus il est au jor d'ui. Por cesti chose asavoir, \persName{Helcanus}, \persName{Fastidorus}, \persName{Dorus} et \persName{Peliarmenus} se fissent mout richement et a g\supplied{ra}nt merveille seurement aprester et lor batailles d'autre part, ou il ert \persName{Daphus} li Gris, a \num{ii} banieres de ses armes et \num{x} chevaliers de sa maisnie, \persName{Mirus} li Fiers en teil point, \persName{Gasus} de \placeName{Rome} en auteil point, \persName{Cliodorus} en auteil maniere. Ensi avoit chascuns de ces \num{iiii} freres une bataille aveuc la siue qui li venist au secours se mestiers li fust. Et por ce ne demouroit mie qu'il n'euissent chascuns des quatre freres autres chevaliers et escuhiers sans cui il ne peuissent longes durer, et tout en auteil maniere \fnfpz{Ce verbe est-il à rattacher à 'rencier', retomber, récidiver (mais le sens négatif de 'retomber dans une faute' semble ne pas convenir), ou à 'rencier, ranchier', rançonner, mettre à rançon (ou racheter, délivrer) ? Le contexte du tournoi inciterait à choisir le 2ème sens, mais dans la phrase, j'ai plutôt l'impression qu'il s'agit d'une répétition de la même attitude de l'autre côté}rancient cil encontre qui il avoient afaire et en chief chascuns quadrubles banieres teles come de lor armés.


\pnum \persName{Dorus}, ausi come jou avoie dit desus, avoit fait doubler \persName{Karum} de \placeName{Nisse} le bon chevalier, de ses armes, et il meismes s'estoit mis es armes \persName{Leum} son frere, et avoit une bataille de \num{x} chevaliers esleus, dont li vaillans chevaliers dist une fiere parole, car il dist : <<Biau signor, je vous ai esleus a l'ounor d'un mort chevalier por sauver son non, et vous savés bien que, quant li hom muert, que li siens nons ne puet et ne doit morir. Et por ce qu'il ne puet morir, vous pri jou a tous ensamble que vous m'aidiés huimais le sien non en retenance, tout en porisse li siens cors en tiere, dont que je sai qu'Envie est si joiause de çou qu'il est trespassés et mors, qu'elle nos pardonra tous mautalens, ja soit çou chose que nous veillons le sien non ensauchier.>> -- <<Sire, sire, dist cil \persName{Hainnuiers} dont jou avoie de devant parlé, ne nos preciés mie, car nous le savomes bien tout.>> -- <<\persName{Hainnuiers}, dist \persName{Dorus}, je ne le di por el que vos sachiés que \persName{Leus} si doit avoir au jour d'ui chevaliers esleus.>> -- <<Sire, dist chascuns, bien nos entendons a ce que vous nos avés dit, et nous soions li premier.>> Dont il avint que il fu eure que on dut issir as chans, par coi cil furent li premier issu.


\pnum \lettrine[lines=2]{\color{darkred}A}{priés} ne demoura mie que \persName{Helcanus} et lor partie issirent mout ordeneeme\supplied{n}t. Li rois d'\placeName{Aragon} a l'autre lés et li sien ne furent mie a aprendre. Si ne vous puis or mie faire atendre dou tout liqueil furent a l'un lés ne liqueil a l'autre autrement que je des\supplied{sus}\fntooltip{V3} avoie touchié, et vous qui auques savés coment teus afaire si doit iestre raconteis, si vous en sousfisse çou que jou vous en sarai raconter selon l'escrit.


\foliomar{6va}{https://gallica.bnf.fr/ark:/12148/btv1b10023851v/f18.image} \foliomarID{6v}

\section*{[C4]}


\pnum \lettrine[lines=4]{\color{darkblue}C}{i} endroit nos dist li contes que, quant li prince et li baron se furent mis tout hors de \placeName{Rome}, ensi come jou vous avoie touchié, li emperere et li dus alerent de l'un lés a l'autre por savoir l'acort d'iaus. Si ni troverent chose qui les enpechast fors tant que mout lor demoroit ce qu'il n'erent ensanble, dont il i eut de teus qui ja asés atant ne le cuidierent faire, qui puis lor sambla que li jornee fust trop longe. Dont ne demoura mie que lor batailles ne fussent ordenees ; et a qui chascuns se devoit asambler, si m'estuet ore dire liqueil furent li premier.


\pnum \persName{Dorus}, qui avoit le cuer de tous, ausi come de celui qui ne se vausist partir de nului se li miudres n'en fust siens, fu contre le partie de ses freres. Fist ses banieres conduire el premier chief de lor gent et cil qui a l'autre lés vint el premier chief, si fu li damoisiaus de \placeName{Puille} qui ja asés a tans n'i cuida venir as coous douner. Cil avoit o lui des millors chevaliers dou monde, si come vos m'orés traitié avant que jou ai mie mout parlé d'autrui. Lors ont cil doi prince avisé li un l'autre et vinrent au ferir des esporons, si ne le fissent ore mie mal ordeneement, mais tout ausi come g\del{r}ierfaus descent a sa proie, se sont entraprochiet ferant des esporons. Qui dont euist veü la contenance d'iaus \num{ii} dire peuist : <<Cist n'ont mie le repos encoumenchiet !>> Car je truis escrit que si ataignanment feri li uns sor l'autre que longement n'euist mie duré li Grius quant uns chevalier de \placeName{Puille}, que li escris apiele \persName{Fenor}, jeta les mains a \persName{Dorus} et l'euist do cheval a tiere, quant il hurta des esporons et li vola des puins, ``come anguille au povre pescheor''\fnnote{Proverbe «a grand pescheur eschappe anguille» chez Cotgrave. Ou encore «anguille peschie, n'iert ja ampoignie» pour rester dans le thème du «poing».}. La endroit recoura a celui de l'espee et le mist a l'escremie mout anguiseuseme\supplied{n}t. Si me covient ore de ciaus laissier atant et venir a tous emsanble, que ja estoient pelle melle qu'il ne fust nus qui peuist conoistre liqueil en euissent le millor, car se li une partie pierdoit en un liu, il regaignoient en un autre. Si avint selonch çou que li contes tiesmoigne que si grans force de chevalerie ne fu onques en une piece de tiere veüe come la peuist on veoir, et bien i parut em pluisors tamai\supplied{n}te maniere, car je truis escrit que, quant \persName{Helcanus} eut asamblé au roi d'\placeName{Aragon}, et \persName{Fastidorus} a \persName{Japhus} le Fris, et \persName{Karus} de \placeName{Nisse} a \persName{Josias} d'\placeName{Espaigne}, et \persName{Peliarmenus} au conte de \placeName{Flandres}, que d'une grant demie liue en peuist on oïr le fereis qui painne vausist metre a l'entendre. Si ne puis ore mie raconter ne dire de chascun : <<Cil feri a ce liu ne cil a cel autre>>, car trop durroit li \fnfpz{au sens de 'plaisanterie, farce' ? ou au sens de 'difficulté, peine' ?}ruse et d'autre part qui le verité en vorroit dire ne saroit. Il covenroit que li aucun de coi li contes fait mension euissent confusion d'aucuns povres chevaliers qui si les aloient destraingnant, que se il ne fussent soucouru d'autrui que d'iaus, il n'euissent plain piet de \fnfpz{Sens ? Le sens de 'lien, dépendance' convient-il ? on retrouve l'Expression plus bas}loiien. Mais \persName{Helcanus}, de coi li contes taire ne se puet, avoit es premieres venues si le roi d'\placeName{Aragon} desbareté qu'il n'avoit povoir qu'il peuist metre desfense a lui. En cest liu avoit \persName{Fastidorus} ses f\supplied{re}res le tout pierdut contre son cousin \persName{Japhus} le Fris quant \persName{Helcanus} guenci a lui et si laissa que li rois peuist recouvrer a sa maisnie. Qui dont euist veü \persName{Fastidorus} recouvrer quant il eut le soucors de son frere, dire peuist q\supplied{u}'a lui ne seuist povoir dou prendre.


\pnum Li cuens de \placeName{Flandres}, qui a l'autre lés avoit le tornoi a \persName{Peliarmenus}, n'estoit mie oiseus, car li escris dist que tant avoient ferut li uns et li autres sor lor avierse partie, \colmar{6vb}\colmar{b} que mal de celui qui tous n'en fust \fnfpz{«rensés» -&gt; «rincer» &gt; lat. recentare ; mis plaisamment pour «charger de coups» ?? ; une origine picarde donnerait plutôt «*rinchier»... Hypothèse plus satisfaisante : «reuser» &gt;recusare : 'Reculer, se retirer de qq. part' rëusés peut signifier « confondu », « troublé » (voir FEW = torner a rëus). Mais l'hypothèse de « rensés » = rossé est séduisante aussi (sauf si l'absence de traits picards vous semble rhédibitoire).}reusés, car ausi come jou ai dit desus, li cue\supplied{n}s, qui toute s'entente metoit a ce qu'il peuist metre \persName{Peliarmenus} au fianchier\fnnote{au sens de "se reconnaître prisonnier". Infinitif substantivé.}, ne le peut faire por le \fnfpz{Déverbal de 'peceer, peçoier' = mettre en pièces : 'à cause des coups répétés d'une pauvre chevalier qui l'attaqua tellement qu'il lui détruisait tout ce qu'il est possible de faire, au point que li comte vint trouver Pélyarménus (ou le pauvre chevalier ?) et lui rendit son épée en disant...'}peecement d'un povre chevalier qui si le contredist qu'il li desfaisoit quanqu'il peut faire, que li cuens vint a celui et li rendi s'espee, et dist : <<Sire chevalier, de moi povés vous faire vostre plainne volenté, car vous ne me laissiés avenir a mon droit.>> -- <<Sire, dist cil, je n'iere mie niches dou rechevoir, coment vous soiiés cortois dou faire.>>


\pnum \lettrine[lines=2]{\color{darkblue}E}{nsi} esploita li cuens au chevalier, et quant \persName{Peliarmenus} oï çou, si fu mout joians et cuida bien avoir l'auwe \surplus{avoir} copee. Mais en pau d'eure tourna la roe autrement quant uns povres chevaliers vint au conte et li dist qu'il n'avoit riens fait se il n'amenoit a ce qu'il avoit tout le jor chacié et sans faille, que si fist il, car il ne demoura mie que li cuens a l'aide de celui qu'il ramena por \persName{Peliarmenus}, teil que, mal grei de tous ciaus qui aidier li vorrent, fu pris dou \fnfpz{Nouvelle rupture de construction : 'il ne demoura mie que li cuens .... fu pris dou conte' (le copiste - ou l'auteur ? - oublie qu'il a commencé la subordonnée par le sujet 'li cuens', et tourne la phrase au passif en remettant 'comte' comme complément d'agent. Reste à savoir quel est le sujet de 'fu pris'... Est-ce toujours le pauvre chevalier qui l'a vaincu plus haut ? Ou un autre 'pauvre chevalier', celui qui vient de le défier ?}conte et li covint fianchier et faire fin. Ensi covint l'un pierdre et l'autre gaingnier. \persName{Karus} de \placeName{Nisse}, qui a \persName{Josiam} avoit l'estrit, avoit ja tant fait qu'il avoit mis lui et les siens ausi come au sofrir dou tout quant novieles vinrent a \persName{Helcanus}, qui ne se tenoit ausi come nient a sa ba\del{i}taille, anchois chiercoit l'un et puis l'autre por savoir liqueil en avoient le millour ne le piour, et tant qu'il s'enbati ausi come par aventure sor le damoisiel de \placeName{Calabre}, que \persName{Mirus} avoit mout mis entrepiés, q\supplied{ua}nt il vint a ce point sour lui et feri a diestre et a seniestre en teil maniere que vausist \persName{Mirus} u non, li a fait guerpir le damoisiel, qui mout avoit mis grant painne a lui desfendre isi conme \persName{Helcanus}. Lors quant \persName{Mirus} vit \persName{Helcanus}, fu si avirounés qu'il covint que il se meist dou tout a sousfrir de ci a dont que li damoisiaus de \placeName{Chalabre} se fu recovrés, et vint l'espee ou poi\supplied{n}t de si parfaite viertu que il a ciaus lor doubla sousfrance. Quant \persName{Helcanus} se senti desempeechiet, si s'aficha es estriers, par coi li diestrier \fnfpz{Pourquoi un infinitif ? sorte d'infinitif de narration ? ou d'infinitif substantivé ? ou sorte de proposition infinitive ? On n'en est plus à cela près avec ce copiste / cet auteur.... La syntaxe n'est pas son fort... Sens « au point de courber le destrier sous lui »}archoiier desous lui, et apriés tint l'espee toute nue, si poinst le cheval que li despecha la priesse en teil maniere que mal de celui qui voie ne li ait faite. Dont s'en vint u \persName{Dorus} avoit encore le caple au damoisiel de \placeName{Puille}, qui metoit grant painne a çou qu'il se peuist partir a hounor de \persName{Dorus}, qui teil paine metoit a çou que il le peuist faire fianchier come une fine merveille. Et qu'en avint ? Illuech se guenci \persName{Helcanus} en la priesse, si le desrompi en auteil maniere come fait li faucons le fouch des anes apriés le mallart. Qui lors euist veut le maintien coment il se prist a \persName{Dorus} son frere qui le damoisiel avoit fiancié qu'il de lui ne se partiroit devant çou qu'il l'aroit mis a merci, u il lui ! Adont i eut sachié, bouté et hurté et feri en tamainte biele pluisor maniere, si qu'en la fin covint que \persName{Dorus} venist au sou\corr{f}{f}rir\fnnote{Est-ce cette lettre qui est exponctuée ? ou le -r- qui suit ? Et est-ce un -s- ou un -f- ?} et le covint reuser entre ciaus qui le missent arriere de son cop, et en fu ausi come a l'issir dou sens. Lors s'escria s'enseingne en teil maniere qu'il dist : <<Or li aiuwe a \persName{Leum}, or li aiuwe \del{l} a \persName{Leum} !>> \persName{Hainnaus}, dont jou avoie devant parlé, n'avoit mie en oubli la parole que vous oïstes desus. Anchois eut tost entendut la parole quant il dist <<or li aiuwe a \persName{Leum}.>> Car il en teil maniere achainst le \add{^bon^} diestrier ou il sist, que vosissent il u non, remist son signor au deseure de çou que \persName{Helcanus} l'avoit mis au desous. Et quant \persName{Helcanus} vit qu'il avoit failli au damoisiel faire aiie, si hurta li diestrier de mervillous aïr et muet a \foliomar{7ra}{https://gallica.bnf.fr/ark:/12148/btv1b10023851v/f19.image} \foliomarID{7r} \persName{Hainnau}, qui le recuelli entre ses bras si douchement que, vausist u non, le covint fianchier a \persName{Leum} de millor volenté qu'il n'euist fait a nul autre qu'il seuist en tout le tournoiement. apriés ceste chose ne peut durer la reconse dou damoisiel qu'il ne li covenist venir a merci et fiancier tuit a \persName{Leum}. <<A \persName{Leum} ! dist \persName{Hainnaus}, a \persName{Leum} ! Qui en vieust se vingne \add{^a son^} encontre.>> Lors retorna tous li tornois sor \persName{Leum}. Qui dont veist ses banieres droitement en aïrement tenir \del{droitement} \corr{l}{s}our lui, bien dire peuist on : <<Cil \persName{Leus} est esleus sor tous les autres, qui ne trueve son pareil ne qui a lui puist durer.>> Ceste noviele vint a l'empereour et au duch, qui trespasoient les rens et veoient les mius faisans a l'un lés et a l'autre, et <<vraiement, dist li emperere, que on puet au jour d'ui veoir en ceste piece de tiere des mius faisans, que on ne peuist veoir devant Troie quant \persName{Hector} et \persName{Achile} si fissent onques plus de chevalerie.>> Li dus respondi : <<Sire, il est voirs, selonch çou que jou ai etendut que cil \persName{Hector} et cil \persName{Achilles} furent mout preu a lour tans. Mais tant come a ore je ne pus \del{vivre} veoir que nus au tans de dont peuist faire ce que jo ai hui veü fer \persName{Helcanus}.>> -- <<Sire dus, dist li emperere, mais que çou soit consaus\fnnote{ graphie «au» produit de [ẹ] + [l] antéconsonantique : nord/nord-est} çou que jou vos dirai, et nel di por chose que li uns ne li autres me soit riens, je ne sai home de son tans ne de sa forche de la chevalerie, ne de la proeche de \persName{Dorus} ne se vaut \fnfpz{La syntaxe est-elle de nouveau fautive ('ne se vaut nus' reprendrait 'je ne sais home', que le copiste ou l'auteur aurait oublié entre temps...) ou peut-on trouver une ponctuation qui permette à la phrase de tenir debout ? Début de la phrase : 'Seigneur duc, dit l'empereur, bien que ce que je vais vous dire soive reste entre nous - et je ne le dis pas du tout parce que je porterais de l'intérêt à l'un ou à l'autre - je ne connais pas d'homme de son âge et de sa force qui ait les qualités chevaleresques ni la prouesse de <persName key=}nus, car je ne \corr{s}{l}e saroie trover. Et si n'en vauroie avoir dit tant en audienche por mout grant chose, car lor fais si les moustrer queil il \fnfpz{Comment comprendre la construction et le sens : 'je leur fais ainsi (ou ici) se montrer quels ils sont' ? Il faut alors comprendre le personnel non réfléchi 'les' comme employé à la place d'un réfléchi 'se' : c'est possible (voir syntaxe de Ménard, p. 64 : emploi du pronom anaphorique à la place du réfléchi.)}sont, et a tant je m'en doi tenir.>> -- <<Ensi, dist li dus, covient dire a la fois çou que li cuers pense.>> En çou qu'il se deparoloient, estemevous une chace de chevaliers qui venoit a grant bruit et amenoient un chevalier qui a merveille metoit grant defense a çou que on le voloit jus metre dou cheval, et uns autres venoit apriés de mout grant ravinne, criant : <<Or li aiue ! Or li aiuwe a \persName{Daphus} !>> Qui dont euist veut \add{^celui^} coment il se feri entr'iaus qui son signor enmenoient, dire peuist merveille, car mal de chelui qu'il ne covenist laissier celui \persName{Daphus} et entendre a celui qui si les frapoit de l'espee qu'il ne consivoit un seul, qu'il ne l'envoiast aval sour le cheval devant ou derier ! Et quant il a lui, si povés savoir que mout eut a soufrir, car si de tous sens si l'avirounerent, que mal de celui qui ne ferist sor lui, bien de \num{x} chevalier qu'il estoient. Si que, quant qui il enmenoient avant se retorna a aus, si ne vit onques mais \num{ii} chevaliers metre si grant painne ne si grant forche a aus faire valoir come cil fissent. Li emperere et li dus, qui çou esgardoient, furent tout abaubi coment cil poreist durer contre ciaus qui les aloient \corr{cu}{in}juriant, si qu'il ne failli se çou non qu'il vit venir ses banieres de \persName{Daphus} o eus \num{iii} chevaliers qui ne venoient mie a gabeles. Mais en auteil maniere se ferirent ou tas que se il se missent en un gues. La endroit recovrerent li \num{ii} chevalier lor forche, si qu'il avint que par arramie dura tant li estris que li \num{v} des \num{x} pierdirent lor chevaus, et li autre \num{v} s'en partirent par anui.


\pnum \lettrine[lines=2]{\color{darkred}E}{nsi} pierdoit li uns et li autres recouvroit, si qu'il ne fu nus qui peuit mie cuidier les aventures qu'il i avint, car tout ausi come jou vous avoie dit coment \persName{Dorus} s'estoit partis dou damoisiel de \placeName{Puille}, li avint c'une bataille li \fnfpz{Je me demande si cela ne peut pas être une forme du verbe 'sourdre'. Pour le prétérit Gdf donne des exemples de 'surst' 'sourdist' ou 'sordi' (P3), 'sustrent' ou 'sostrent' (P6)}souist de chevaliers qui mie n'ere\supplied{n}t a signor, et furent bien cil \num{xv} qui erent ausi famich come teil qui avoient esté em prison. Cil vinrent a uns fais sour lui et l'aqueillierent a l'un lés et a l'autre come cil qui mout s'abandonoit \colmar{7rb}\colmar{b} et achainst le diestrier des esporons, et se feri entr'iaus, ausi conme cil qui soi voloit essaiier, et il si fist en maniere que je vous dirai. Il vit et si seut que cil ne li feroient nule raison, et il feri a l'un lés et a l'autre de l'espee, que li pluisour ne \fnfpz{l'oisoient (n'osaient l'approcher) ?}l'oisoient aprochier. Mais ne demoura mie li un muet a lui qu'il le cuida enbrachier, si li desfendi un coos si grans sor l'un des bras qu'il i rompi le mohoistre. Et uns autres l'enbracha de si grant forche c'a poi qu'il ne le torna a travers jus do cheval. Et quant il se senti de celui si entrepris, si le cuida rahierdre, mais il n'en eut povoir car trop fu couvriiés des autres qui sor lui feroient sans pitié. Quant \persName{Dorus} vit ce, si hurta et poinst le cheval, mais tout ausi come li senglers qui est de toute pars \fnfpz{PP attesté en -ER}ahiers des levriers, et il se \fnfpz{Le tilde semble gratté ; la forme singulière peut être préférable : accord avec le possessif P3 : aller SA voie =/ P6 aller LEUR voie}muet por aler sa voie, si enporte apriés par sa forche \del{apriés a} ciaus qui lui tiennent ausiment par le force dou diestrier et de lui. Les mena il grant pieche criant : <<Or li aiuwe a \persName{Leum} !>> -- <<Ne vous vaut, dist li uns.>> -- <<Rendre vous covient a cui ?>> dist \persName{Dorus}. -- <<Au povre chevalier, dist li uns.>> -- <<Et par Diu, dist il, ja cil povres n'iere a cui je me veil rendre.>> Illuech peut bien \persName{Dorus} mostrer sa forche, quant cil le demenoient sor le cheval de si males aleures qu'il le voloient tout defroissier por la raison de lor conpaingnon qu'il avoit issi blecié, come je ai dit desus. Sa maisnie a l'autre lés ravoit tant a faire que cil dont il se devoit mius aidier erent entrepris de \num{ii} batailles ou de trois. Meime cil \fnfpz{exemple de 'que' pour 'qui' (introduction linguistique : morphologie -ou morpho-syntaxe- du pronom-personnel}que ses banieres portoient ne seurent qu'il fu devenus. Anchois se tenoient en une mervilleuse tourbe ou il veoient de lor chevaliers entrepris, et pierdoient de lor chevaus vaussisent ou non.


\pnum \lettrine[lines=2]{\color{darkblue}E}{n} çou que vous avés oï desus, rescrie \persName{Dorus} : <<Or li aiuwe ! \persName{Hainnau} a \persName{Leum} !>> Lors li dist uns chevaliers : <<Ja \persName{Hainnau} ne autres ne vous i ara mestier que ren\del{chce}dre ne vous estuece au povre chevalier.>> -- <<Ja par Diu, dist il, au povre ne me renderai.>> -- <<Si ferés, dit cil, au povre d'avoir et riche de chevalerie.>> -- <<Et par Diu, dist \persName{Dorus}, ja enviers celui ne meterai desfensse quant je le troverai.>> -- <<Trové l'avés, dist cil, aler ne vous covient mie lonch a i \fnfpz{Comment comprenez-vous ? Est-ce une graphie du verbe être ? Je n'en trouve pas d'exemple. Faut-il corriger en 'errer' ?}erre.>> Atant le fiert uns autres un cop si dur et si pesant sor son hiaume qui le fist acoler le col de son diestrier, et a çou dist cil : <<Au povre chevalier vous covient rendre !>> A ce fu \persName{Dorus} abaubi, si car il n'ert mie encore a ce menés qu'il n'euist recovré de sa forche. Si le coureurent seure de si grant ravine, que se cil n'euist esté soucorus, tout se fust partis de celui sans autre doumage. Mais cil le racuellent devant et deriere, si l'euisse\supplied{n}t en la fin ochis quant cil \persName{Hainnaus} se feri entr'iaus conme gierfaus a la grue.\fnworknote{Image stéréotypée, comparant le combattant à l'oiseau chasseur. Emprunt à Yvain, ou le chevalier au Lion : si con girfauz grue randone. Cf. Baudouin Van den Abeele De l’épervier à l’émerillon : images de la chasse au vol dans les romans de Chrétien de Troyes } La endroit reconmecha une escremie mervillouse, que il covint que li gius tornast au meschief, car issi come je le truist escrit, il covint a ciaus de lor malisse coper le cengles as chevaus des \num{ii} chevaliers avant qui les peuissent metre a tiere, ne seuist l'eure de mot. Quant il se troverent jus sans autre pierte, q\supplied{ua}nt \persName{Helcanus} vit son signor a tiere dejouste lui, si le prist par le main, et dist : <<Sire, traions nous hors de ceste chace qui vient ci, car trop me doute que vous ne soiiés bleciés de ces maleois robeour.>> -- <<N'en doutés ja, dist il, mais de ce ai je grant merveille ou vostre chevalier sont.>> -- <<En non Diu, sire, dist \del{c}il, \foliomar{7va}{https://gallica.bnf.fr/ark:/12148/btv1b10023851v/f20.image} \foliomarID{7v} si \fnfpz{La P3 est bizarre, on attendrait la P1, 'dui' (à Dorus qui lui demande où sont ses chevaliers, Helcanus répond 'j'ai dû les perdre tous'. ?)}dut avoir le tout pierdut.>> Atant vinrent lor banieres waucrer parmi le tornoi, si se sont trait cele part, et escuhier comencent a cuerre a l'un lés et a l'autre. Si ne furent onques gent \fnfpz{??? Comment comprenez-vous cette forme ? Ne peut-on résoudre autrement ? Est-ce un participe dialectal ? Je n'en trouve pas la trace... Cette forme devrait être un participe passé au cas régime masculin pluriel, mais est sans contexte graphiée 'destrua' (quel est d'ailleurs le sens de la phrase 'il n'y eut personne de détruit sinon lui' ou 'sinon eux' = les chevaliers d'Helcanus ?).... Cela me fait penser à la forme 'esta', du verbe 'ester', mais c'est plutôt une sorte d'impératif. Le FEW parle d'une 'interjection commandant l'arrêt'. Bref, ce n'est pas ce qui convient dans la phrase en question...}destrua s'il non, en ce que \persName{Dorus} et ses compains aloient apriés lor banieres. Lor avint qu'il ont veü \persName{Helcanus} qu'il \num{iiii} le menoient a piet entr'iaus mout vilainnement. Et li avoient le hiaume dou chief osté par mal avis, si qu'il ert ausi come tous sanglens el vis des hurteures que cil li avoient fait. Et quant ce vit \persName{Dorus}, tout ausi come lupars qui saut a sa proie, vint a l'encontre de ciaus, l'espee ou puing, et en fiert si l'un que il l'enporta dou cheval a tiere jus, a teil meschief qu'il li brisa la canole. Dont saissi le diestrier et son frere par la main, et dist : <<Montés, sire chevaliers, si irés plus covignablement ou il vous vauront mener, car mie n'est chose covingnable que il voisent a cheval et vous a piet.>> Il n'eut mie le mot pardit quant ses compains en eut un autre mis a tiere et sailli ou cheval, si se prist as autres \num{ii} qui a merveille eurent grant ire de ce que avenut lor estoit. Si qu'a ce qu'il se desfendoient se mist \persName{Helcanus} ou cheval et sans hiaume qu'il euist en chief, si cuida departir \persName{Hainau} de ciaus, mais il ainch n'i eut entendre. Anchois en covint l'un trebuchier, de coi \persName{Dorus} ne fist dangier dou cheval s\del{i}aisier en lui metre en la siele. Quant cil eurent ensi recovré çou que vous povés oïr, si ne vorrent illuec le quart metre jus de son cheval. Anchois sont a lor gent revierti qui mout avoient grant dolor de çou qu'il avoient tant esté ¶ sans aus.


\pnum \lettrine[lines=2]{\color{darkred}M}{out} longement dura cis tornois d'une part et d'autre, et tant que li solaus avoit ja tant alé qu'il fu venus d'orient en ochident. Et li un et li autre avoient tant feru que mal de celui qui volentiers ne vosist que chascuns fust retrais. Li emperere et li dus firent souner le retraite et se coumencierent a retraire petit et petit. Mais quant ce seut \persName{Dorus}, dont primes fu il raloii\surplus{i}és a sa bataille et mis en fors et entais chevaus, si se feri es plus fors batailles qu'il veoit, si qu'il avint a celi fois que si chevalier le tinrent plus priés qu'il n'avoient en tout le jour, si qu'il avint qu'en la fin les for\surplus{\add{^li^}}ma\fnworknote{Ne peut-on ignorer la correction, et garder « forma » au sens de « mettre au pas », « éduquer », avec un sens ironique, comme quand on dit « je vais lui apprendre... » ?.}, si qu'il n'eut a cui avoir l'estrit, fors au conte de \placeName{Flandre}, qui de sa partie fu. A celui se prist il, mais en pau d'eure euist cousté li aillié. Quant li emperere et li dus les de\del{s}partirent, si se retraisent a merveille tart en la citei.


\pnum Que vous iroie desoremais plus autre chose disant ? Tout et comunalement qui faire le volt vint a court souper. Ja peuist on veoir des nés et des surchius froissiés et autre mainte pluisor bleceure qui puis ne fu si sanee qu'il n'i parust toute saine. A l'autre lés, mout i eut de ciaus qui furent si bleciet et navret qu'il lor covint demorer as osteus et iaus faire garder qu'il ne chaïssent en piour point. Mais de ce ne me covient mie faire lonch conte, for que venir a ce que, quant li souspers fu fais, il se missent d'une part. Li grant signor si encherchierent le fait de chascun, si troverent tout de comun asens que \persName{Leus} avoit le pris dou mius faisant des quintainnes et del tornoiement. Ceste chose fu partout seue et disent entr'iaus que ce n'ert mie sans raison, anchois en estoient li pluisor sage et ciertain. Avint de ce que il covint a l'empereour cuere maniere et coulour de faire entendre et savoir a tous ensamble la venue del bon chevalier \persName{Leum}, dont il n'i eut nul d'iaus qui mout n'en euist grant pitié et en fissent chiere marie meis\colmar{7vb}\colmar{b}mes. Issi avint que li emperere vint a sa fille, qui feme avoit esté au noble chevalier \persName{Leum}, et li dist tant d'un et d'el que il li fist a savoir au plus amiaublement qu'il peut coment Nostre Sire avoit porveu dou comencement dou monde la fin de chascun, por coi il dist que entre le fin de chevalerie et le mort de pluisors preudomes ses sire avoit eue noble fin et biele repentance. <<Ha ! Pere, dist la dame, come ceste fins et ceste mors me doit iestre mise en retenance quant onques jor de ma vie n'eut un tout seul jour de joie !>> -- <<Ensi, dist li emperere, covient vivre en cest siecle. Mais atant me covient ore metre fin a ceste chose et venir a çou qu'il covient muer cest duel en joie de çou qu'on puet.>> Mais en la fin covint a autre chose entendre, car li prince et li baron qui tuit en furent anoiié de fiestoiier et de joie faire, vinrent a lui et li disse\supplied{n}t tout ausi come ensamble : <<Ha ! Sire emperere, come Nostre Sire Dius nos\fnnote{parti intérieur/extérieur dans une quintaine/tournoi} a fait grant hounour en cest siecle quant il vous a mis en la haute roe de Fortune, car vous avés sourmonté tous ciaus que on puet savoir qui contre nos ont esté, et encore plus, a ce que on puet veoir. Vous avés fait de vos anemis nos millours amis, et a daerain mis en cest siecle en parfaite ¶ honor.>>


\pnum \lettrine[lines=2]{\color{darkblue}Q}{uant} li baron eurent çou dit, si parla li emperere humlement a aus en disant : <<Ha ! Biau signor, de çou vous doi jou mout hautement merchiier, si come cil qui ne sui c'uns seul hom d'asés povre sens et non mie de mout grant valour. N'euist esté la vostre grans proece qui s'est estendue a parfaite chevalerie, por laquele je sui venus a ce que vous povés veoir.>> -- <<Sire, dissent il, autant i a de l'un come de l'autre, car se la vostre chevalerie et la vostre valour n'euist esté, ja ne vous en fussions mis a ce que nus\fnworknote{sic} soumes fait, por coi on ne voit au jour d'ui mie sovent de povre chief venir a bone besoigne faire.>> -- <<Beneïçon aiie de Diu, dist li emperere, come ceste parole doit iestre covingnable as pluisors qui les honors doivent avoir a maintenir !>> Ensi ont mout grant piece parlé sor cesti chose dont je ne me doi ore mie ariester fors que de venir a ce que tuit li prince de cui j'ai fait mension devant s'apresterent de congié prendre et de raler chascun en so paiis et en sa tiere. Si le fissent mout covingnablement. En avint que li emperere dist : <<Biaus signor, mout me part a envis de vous. S'il peuist iestre que nos tous jors puissons iestre ensamble, mais il ne puet avenir. Et por itant proï jou a celui a cui nos devomes iestre tuit sierf que je puisse encore lui faire siervice, a coi vous puissiés encore tuit partir en maniere qui vos sache grei del siervice que vous li avés fait.>> Il respondirent tuit que il lor ert bien meri quant la chose estoit venue a ce que chascuns povoit veoir. Dont il avint que apriés cest congié que li emperere lor fist douner mout riches dons selonc ce que chascuns ere, et avint que mout sont esmerveillié li pluisor, dont si riche juiel porrent venir, qui sousfirent a si grant barounie qui la estoit asamblé, por coi li aucun disent que nule chose ne povoit tant pourfiter come sagement g\supplied{ar}der ce qui en tans et en liu povoit avoir mestier. Et sans faille, que ceste chose avoit li empereris fait, qui b\supplied{ie}n avoit en liu et tans mis ensamble ce qui puis li torna a pris et a loenge. Si vieng ore a ce que tuit issi loant se sont mis chascuns vers son paiis et n'ont finé l'un jor plus l'autre mains tant qu'il sont venut en lor contres sain et sauf et liié de que\fnnote{"que" = "ce que" (relatif neutre sans antécédent). Cf. Syntaxe de Ménard, p. 80 : "le relatif neutre au sens du FM “qui, ce qui”."} avenu lor estoit. Si me veil ore atant d'iaus taire et venir a l'emperere et a l'empereris coment il se maintinrent apriés cesti \foliomar{8ra}{https://gallica.bnf.fr/ark:/12148/btv1b10023851v/f21.image} \foliomarID{8r} avenue.


\section*{[C5]}


\pnum \lettrine[lines=4]{\color{darkred}O}{r} nos dist ici endroit li contes que, apriés çou que li baron et li prince de tamai\supplied{n}te region se furent parti de \placeName{Rome} de l'empereour et de l'empereris, il demourent en lor bone pais et en lor grant joie a \placeName{Rome}, si come cil qui avoient grans prosperités et mout de lor desiriés acomplis. Et ert ensi que pluisor fois il aloient de l'un empire a l'autre et menerent bien ceste vie parmi \num{x} ans. apriés ice avint un jour que li emperere et li empereris se joient as eschas, et avint ausi come par aventure que li gius se torna a ce que li emperere en eut le piour, par coi li empereris en amor et en joie dist : <<Sire, or sachiés que vous ne chevacerés mie tous jors a lorains que il ne vous covigne pierdre en aucune maniere.>> Li emperere entendi l'empereris et nota ce que elle avoit dit en autre maniere que elle n'avoit fait. Lors li dist en sourriant : <<Dame, dame, n'est mie merveille, s'il n'est \add{^ensi^} que vous avés dit, car toute sifaite chose est çou de ce siecle, car quant il avient que on s'i aseure le plus, c'est quant ``on est plus tost torné çou desous deseure.''>>

\pnum Quant l'empereris eut oï l'empereour si mua a mervelle grant coulour et dist : <<Avoi ! Sire, or amaisse je mius que je me fusse teute.>> -- <<Dama, dist il, ja, se je puis, ne vous en repentirés, car tout li tans et les eures ont lor saisons. Et je ne cuit mie que ce que vous avés dit ne doiie porter fruit fructefiant en liu et en tans, dont il nos ert mius apriés ce que nous seroumes mort et trespasé.>> -- <<En non Diu, sire, dont veil jou que vous me dites çou dont je ne sui mie sage, coment ne a coi vous avés notee la parole que j'ai dite, quant je ne le dis fors en amour et en fieste, et me vola hors de la bouche. Por coi vous pri en amor et en gueredon que, se je ai dite chose qui vous ait anoiié, que vous le me pardounés, et, s'il vous a esté biel, que vous me fachiés sage por coi vous l'avés issi repris.>> -- <<En non de moi, dame chiere et bone amie, or m'avés vous demandé une chose dont je veil que vous en sachiés la verité, et je le vous dirai, dist il, mais que çou n'iert mie ore, anchois iert une autre fois quant je verai que poins sera, et çou iert quant je verai et sarai en vous autre volentés que jou encore n'i voie.>> -- <<Ha ! Sire, par Diu merci, ce dist la dame, or soit quant il vous plaist, car je ne cuit mie que vous le me doiiés bien dire avant qu'il m'en fust auques de mestiers, ne je ja ci endroit n'arai la maniere d'au\del{ques}\add{^cunes^} qui ja a tans ne quideront \add{^savoir^} ce qui ausi leur puet torner a anui come a joie.>> -- <<Dame, dist il, sauve soit vostre grasce, je ne cuit mie que ceste chose vous doie torner a anui. Et por ce que je ne veil mie qu'il vous puist torner a anui s'a grant porfit non, je vos en dirai une partie de ce que vostre parole m'a dounee a entendre. Il \add{^est^} voirs que on dist en provierbe que ``tandis que li gius est, biaus est un poins dou laissier'', ausi come il avient que cil qui est en vertut d'astinnence que, quant il mengue d'une viande volentiers, qu'il n'en prent mie tant que par abomination li covigne laissier, anchois se refraint et le met arriere por çou que elle li face millor digestion d'aquere natureil apetit. Tout autre\add{^si est^} il, ma tres chiere dame et bone amie, m'avés vous mis en voie de ce qui me puet faire millor disgestion en liu et en tans et vous dirai coment ne por coi.>>

\pnum <<Come il soit ensi que N\supplied{ost}re Sires m'a prestee tant d'ounor qu'il m'a mis el plus haut estage \add{^de la roe^} de Fortune, ausi come il me fu dit un jor qui passés \surplus{es} est, il ne puet mie avenir par droit ne par nature que tous \colmar{8rb}\colmar{b} jors je puisse vivre en l'ounor de l'empire de \placeName{Rome} ne de \placeName{Costantinoble}, anchois sont nei cil qui ceste honor atendent. S\supplied{i} me sambleroit raiso\supplied{n}s et drois que, tandis que li gius est et que je l'ai d'apeti, que jou le laisse, car je sai de voir que je mie ne porai \fnfpz{déjà rencontré aussi cette répétition de 'mie', avant et après le verbe : laisser ou corriger Mais il convient de le signaler en note, et peut-être de faire un § en introduction linguistique, pour signaler ces redites, ces reprises de termes (la conjonction 'que', cet auxiliaire de négation 'mie'...)}mie tous jors chevachier a lorains, ceste a dire vivre en ceste signorie. Et por ce di je, tres douche dame, que se je ne laisse cest daintié en coi je me sui par tante fois delités et enorguilliés, que mout priveement pus faillir a grignor daintiés que cist ne soient, c'est a plus haute honor que ceste ne soit en laquele je ne porroie mie tous jors demourer. Et por itant que je mie tous jors demorer ne porroie \add{^ne^} chevachier a lorrains, si me covient acuere parfaite honor, cest a dire que, se jou ai eut honor en cest siecle, que apriés ma mort en soit aucune chose ramenteue en l'autre.>> Quant li emperere eut ce dit, si parla l'empereris en teil maniere, et dist : <<Ha ! Sire, come de bone eure je dis la parole que vous avés si bien mise a moralité. Se il avient que Nostre Sire vous en doinst metre en la voie dou perseverer, ausi come vous le m'avés doné a entendre, par coi jou en peuisse iestre perçouniere ausi come jou en aroie le volenté dou desiervir !>> -- <<Dame, dist li emperere, la desierte ne tenra s'a vous non.>> -- <<Sire, dist elle, mout grans mercis.>>


\pnum \lettrine[lines=2]{\color{darkred}C}{este} chose ne misent mie en oubli li emperere et li empereris. Anchois av\corr{tnt}{int}t en apriés en asés pau de tierme \supplied{qu'i}l s'aviserent d'une chose dont il furent a merveille puis prisié, et dont li pluisors s'esmerveillierent mout, car li emperere vint en l'empire de \placeName{Costatinoble}, et a a ce menés les baron de l'empire qu'il aseurerent \persName{Helcanus}, son fil, et le reçuirent a signor, et douna \persName{Dorus}, son mainsné frere, le roiaume de \placeName{Gresce}, si que il dou tout se demist de l'empire et prist congiet a iaus et revint arriere a l'empereris, qui tout en auteil maniere come il ert demis de l'empire de \placeName{Costantinoble} se demist il de l'empire de \placeName{Rome} et de toute la tiere que il onques li uns ne li autres maillie\fnnote{Möhren, Frankwalt. Le renforcement affectif de la négation par l’expression d’une valeur minimale en ancien français. Niemeyer, 1980.} n'en detinrent. Anchois dounerent a tous les Roumai\supplied{n}s a entendre qui selonch les avenues de devant cil ne voloient mie que dissension eui\supplied{s}t entr'aus, por coi en lor plainne vie, il seuissent dont elle poroit naistre por plus covignablement metre la concorde. Et sans faille ne fu mie mout grans merveille, car maintes fois est il ensi avenut d'autrui que d'iaus, que plus crient on I home que M et sans faille que ceste colour de raison eurent li une partie et li autre. Mais dou tout ne fu mie la raison de l'empereour et de l'empereris. Anchois furent a ce mis, dont jou ai traitié desus, que il si soutiument atornerent lor afaire que sans le seü de nului se missent hors de \placeName{Rome} mout descounuement, par coi nus ne peuist conoistre lor afaire ne lor iestre. Et ne sai mie ne ne truis en l'istoire se il nule arme n'emenerent aveuc aus, en cui il se fiassent tant come a cestui afaire. Mais d'une mervilleuse chose fait li contes mension, car ausi come je avoie mon conte fait touchier d'un liuon que li emperere avoit porté compaignie, ausi conme il est dit de devant, se gisoit en un liu dispoisé dou palais l'empereour, si que, quant ce venoit au matin ausi acoustumeement, il s'en venoit en la sale et sans nului mal faire et por l'empereour a atendre ausi come di me tu : <<Je doi mon signor garder \foliomar{8va}{https://gallica.bnf.fr/ark:/12148/btv1b10023851v/f22.image} \foliomarID{8v} et lui porter compaignie.>> Il mie ne s'oublia a celui jour dont l'emperere et li empereris s'estoient parti la nuit devant. Dont il avint que, quant li chanberlench se furent apierciut de cesti chose, si furent a merveille abaubi et ne sorent que dire ne que faire. La noviele en ala par tout l'osteil, come cele que mie ne peut iestre cellé legierement et sans faille, que a celi eure li cours avoit esté wid\del{i}é des grans signors et a mout privee maisnie. Il avint que li maistre de l'osteil se misent ensanble et por avoir conseil coment il porroient esploitier de cestui afaire. Dont parla uns chevaliers, qui fu sages et avisés, et dist en teil maniere : <<Je ne puis veoir en cestui afaire c'une chose qui mie \fnfpz{Comment comprendre la P6 ? Noter aussi l'emploi de 'mie' sans 'ne', au sens de 'quelque peu'.}\corr{facent}{face} a douter, et vous dirai quele. Li aucun si senvent bien que autre fois li emperere s'est partis de son empire sans le seu de nului : l'une fois sans le seu de ses barons et enmena son fil aveuc lui, et li autre fu quant il ala es desiers faire penitance, dont il amena ce liuon qui encore l'atent en cele sale. Or est la tierce fois a ce que je puis veoir. Et toutes ces choses avenues, je n'en i voi c'une qui tant face a douter que de nostre empereour noviel qu'il ne nos veille demander son pere, en cui garde il le laissa, car, sans faille, de si grant chose que çou doit iestre de l'empereour de \placeName{Rome}, bien avenist qu'il fust si sagement wardés qu'il ne peuist iestre issus de ses chambres sans le seu d'aucuns, dont il me samble que, se il n'i a nul de vous qui parler en sache, que je me doute que vous n'en aiiés a sousfrir.>> N'i eut nul d'iaus qui bien ne deist que nulement il n'en avoient nient seu.


\pnum \lettrine[lines=2]{\color{darkblue}E}{n} ce que cil estoient a ce conseil, li liuons, qui mie n'avoit apris que li emperere demourast tant qu'il n'issist des chambres, comença mout fort a gromir et a ruignier, si qu'il n'eut si hardi en la sale qui i osast demourer, anchois s'enfui qui mius pot. Lors vint li liuons a l'uis de la chambre ou li emperere avoit geu, si le trova fremee mout fort et coumença a l'uis a grater mout roit. Cil dont j'ai desus parlé, \supplied{qui} tenoient lor conseil de l'empereour, ont oï le liuon qui mout merveilleusement gromisoit et metoit grant paine a çou qu'il peuist en la chambre entrer. Por coi il n'i eut nul d'iaus qui paour n'euist, car il esgarderent a ce que, se on ens le laissoit, et il ne trovast l'empereour, que il ne lor dounast a sousfrir. Dont il n'i eut nul qui seust que faire, fors li chevaliers qui devant avoit fait mention de l'empereour, liqueus vint a l'uis de la chambre et li a ouvierte. Quant il fu dedens, si chiercha aval et amont, isi ne trova mie ce qu'il queroit. Et que fist la mue bieste ? Tout en auteil maniere come li vrais loievriers porsuit le sengler quant il a trové lor voies, en auteil maniere ala il porsivant l'empereour, et trova qu'il ert issus par une feniestre qui ert en une warde reube, et de la ert issus par un faus huis et passés a une naciele un large fossé qui avoit \num{cc} piés, et de la se missent en la citei, si issirent hors de \placeName{Rome} a la porte devers orient, laquele on apiele porte oriental. Et tout issi come je di se mist li liuons apreirs aus que il onques n'espargna rien qui tenir le peuist.


\pnum Quant ce ont veü et seü la maisnie a l'empereour, si eurent mout grant doute del noviel empereour que il ne les vosist ochoisoner de \corr{lor}{son} pere et de \corr{lor}{sa} mere. Dont il avint qu'il i eut aucuns qui veurent aler apriés le liuon, mais desfendu lor fu des plus sousfissant, et disent que qui apriés iaus se meteroit, que ja deuist \colmar{8vb}\colmar{b} rechevoir le gré de l'un ne de l'autre, car chascuns povoit bien savoir que, por ce qu'il ne voloient mie que on seuist queil part il viertiroient, s'estoie\supplied{n}t il ensi de \placeName{Rome} et de lor honor parti. Lors n'i eut nul qui i osast viertir, anchois ont fait savoir a tous lour enfans en queil liu qu'il fussent, long ne priés, ensi qu'il fu avenut de l'empereour et de l'empereris.


\pnum Q\supplied{ua}nt il en oïrent la noviele, mout en furent esbahi et n'en seurent autre chose que il seurent ausi come par avis que il erent alé en exil ausi come li emperere avoit autrefois fait. Li autres dist : <<Or covient que on die que nos avomes esté fil a un hiermite qui ne seit vivre ne morir fors en hiermitage non.>> Li autre disent que il ne cuidoient mie que s'il se fust en son bon sens que ja se fust partis de l'empire ensi qu'il est, ne fust ore por autre raison que tout autre (tant peuist il faire de bien en ce que il tenist l'empire en loiauté faire et en droiture maintenir come de vivre en povreté, et metre son cors a exil ains tans que la mors natureus preist) \fnfpz{Je me demande si je ne mettrais pas entre parenthèses le groupe de mots de 'tant peuist' à 'natureus preist', puis une virgule et pas de majuscule à 'ne (fust ore por autre rison' : ce dernier groupe reprend les mots qui se trouvent avant ce qui pour moi est une sorte de parenthèse. Sens : 'Les autres de dire qu'ils ne croyaient pas que, s'il avait été saint d'esprit, l'empereur aurait jamais quitté l'empire de cette façon, si ce n'était pour n'importe quelle autre raison (car il aurait pu faire autant de bien en gardant l'empire dans la loyauté et en le maintenant dans la justice qu'en vivant dans la pauvreté et en s'exilant jusqu'à ce que la mort naturelle l'emporte), n'importe quelle autre raison que, dans le cas où une dissension naissait dans l'empire de <placeName key=}, ne fust ore por autre raison que, \del{ce} se aucune disencions movoit en l'empire de \placeName{Rome} ou de \placeName{Constantinoble}, que il euist millor povoir de metre a point que nus autres.\fnnote{Ajout par rapport au Pelyarmenus #621.}


\pnum Ensi ont le preudome atourné a mal çou qu'il avoit fait por bien. Dont il avint c'un damoisiel avoit en la court, dont li contes ci devant fait mension, et avoit non \persName{Celidus}. Icelui avoit li emperere \persName{Kassidorus} engenré en une puciele el roiaume d'\placeName{Espaigne}, isi come il a esté dit. Cil ert si biaus et si gens qu'en tout Europ de biauté n'avoit son pareil, et aveuch tout ce il fu sages de son tans plus que on n'en seuist nul en tout l'empire. Dont il avint que, quant il vit que li emperere s'estoit issi partis de \placeName{Rome}, il ne fina si vint a son frere \persName{Fastidorus}, qui ert emperere de \placeName{Rome}. Quant il vi\supplied{n}t deva\supplied{n}t lui, se li filerent aval sa biele face larmes des yex. Lors dist : <<Sire, il me covient a vous pre\del{r}ndre congiet en maniere que aventure est se je jamais vous voi.>> Quant \persName{Fastidorus} l'entendi, si esgarda le damoisiel et vit qu'il larmoit mout tenrement, et en eut mout grant pitiet, et dist : <<Coment ! Biaus dous chiers frere, ja Dius ne place que je jamais puisse jou un jor vivre en pais quant par ma volenté de moi vous departirés, si vous arai autre bien fait que je ne vous aie encore.>> Li damoisiaus li respondi : <<Sire, la vostre merci de tous vos biens, car la merci Diu asés sui riches hom en mon paiis. Mais une chose me destorbe, que jou ai mon signor mon pere pierdu, douqueil jou n'atendoie el que de lui a rechevoir l'ordene de chevalerie. Et il me samble que li atente soit mout longe avant que ce soit mais fait.>> -- <<Mes amis, dist il, ne cuidiés que vous encor n'i puissiés bien venir, car vous iestes encore asés jones por l'atendre, de ci a dont que nous aucunes novieles en poromes oïr.>> -- <<Sire, dist cil, sauve soit vostre grasce, ja ce n'atenderai, car je sai de voir que trop longe seroit ceste\del{st} atente. Anchois ne me laist jamais Dius armes porter come hom qui soit chevaliers, de ci a dont que je li ere de lui.>> Dont ne valut rien a \persName{Fastidorum} chose qu'il peuist dire a \persName{Celidum} qu'il ne se partist de lui, vosist u non, et se mist ariere en son paiis, o lui un chevalier et sa maisnie. Si \surplus{re}retourne ore ci endroit li contes a l'empereour et a l'empereris et se taist de \persName{Celidum}.


\foliomar{9ra}{https://gallica.bnf.fr/ark:/12148/btv1b10023851v/f23.image} \foliomarID{9r}

\section*{[C6]}


\pnum \lettrine[lines=4]{\color{darkred}O}{r} nos dist ici endroit li contes que, quant li emperere et l'empereris se furent de \placeName{Rome} parti, il n'eurent mie alé c'une jornee asés petite quant ce vint au matin qu'il eurent giut a un viliel, et issoient de lor osteil ou il avoient ostelé. Si troverent au matin, a l'issir de l'huis, le liuon, qui se gisoit au dehors de la porte. Et quant li emperere le vit, et il lui, onques ne fu faite joie de bieste a home ne d'oume a bieste si grant come il se firent. <<Beneïçon aiiou de Diu ! dist li empereris. Sire, c'avés vous enpens\supplied{é} de cele bieste a faire, qui en teil point nos a sivi et veut sivre ?>> -- <<Dame, dist il, mout est noble compaignie de lui ! Et puisqu'il plest a Nostre Signor sosfrir, le me covient et vous d'autre part.>> -- <<Sire, dist elle, je me doute mout de lui qu'il ne nos courece en aucune maniere ou destorbe.>> -- <<Dame, dist il, n'en doutés, car il covient, ce me samble, qu'il nous porte compaignie, puisqu'il nous a trovés, et je sai qu'il plaist a Nostre Signor.>> Ensi avi\supplied{n}t que li emperere et l'empereris se misent a la voie ensi come vous oés, et avint que ce lor tornoit a anui qu'il ne porrent venir en liu ou il euist gent que il n'euissent mout grant paour de cel liuon. Si qu'il vinrent en asés de lius ou il ne peurent avoir osteil par le paor qu'il avoient de lui. Anchois lor covenoit gesir as chans ou en aucuns lius hors de la voie.


\pnum Dont il avint qu'il eurent tant alé qu'il vi\supplied{n}rent en la tiere de Patras, ou il avoit mout grant foriest. La, en cel paiis, avoit un home anciien et de sainte vie, dont alerent tant et vinrent qu'il le trouverent. Et quant il les vit, si eut grant merveille qui il fure\supplied{n}t, que une tel bieste menoient aveuch aus si amiablement, quant li emperere, qui en habit estoit d'un saint home qui a merveille sambloit iestre de bon liu venus, salua le preudome et li dist : <<Biaus dous amis, cil Dieus en cui vous avés mis vostre cuer et vos esperance, vous aiit a perseverer en bones ouevres !>> -- <<Biaus frere, dist il, et vous aiiés bone aventure et vostre biele compaignie. Par Diu, queil gent iestes vous qui teil bieste menes aveuch vous si paisiublement ?>> Dont li dist li emperere : <<Sire, por ce soumes nous ci venut que vous sachiés qui nos soumes.>> Lors li conta tout de fil en aguille : qui il ert ne qu'il cueroit, et quel volenté il avoit de demourer en aucun liu, il et sa feme, ou il peuist le cors traveillier et metre a exil a çou qu'il peuist faire sa penitance, come cil qui trop avoit pris et usé des biens Nostre Signor en vanités et en autre maniere. <<Sire, par Diu merci ! Or sachiés que lonch tierme jou ai conviersé en cest desier et fait asés pau de bien. Et quant plus i ai vescu et mains conseillier me sai, por coi je vous di a bries paroles que selonch çou que jou ai entendut de vos, vos avés mains a faire enviers Nostre Signor que jou n'aie. Et d'autre part je vous voi de millor et de grignor sens que je ne sui, et non mie par ce que je ne vous doie dire mon povre avis selonc ce que je sent et voi de vos. Il m'est avis que vous avés ci endroit ceste dame que je voi qui est aparellié a faire tout plainneme\corr{n}{n}t vostre volenté, coment ses povoir se puist estendre, et bien savés que Nostre Sire si ne coumande mie que nus soit omecides del cors metre a destruction, se ce non qu'il en puet souffrir selonc çou que Nature se puet estendre a çou que li chars n'ait mie son delit, par coi on n'oblie mie son creatour. Et par icest raison que jou voi en la dame, qui mie n'est conplexionee de faire penitance en droit de vos et de moi, il covenroit que elle fust en liu qui fust en droit foi et ou ele ne \colmar{9rb}\colmar{b} vous empeechast a faire chose qui a Nostre Creator deuist desplaire.>>


\pnum \lettrine[lines=2]{\color{darkblue}Q}{uant} li emperere eut le preudome entendut, si vit et seut qu'il ert sages. Se li respondi a ce qu'il avoit dit, come cil qui en fu bien apensés, et dist : <<Dous amis, encore m'avés vos dite une raison et pluisors que je voloie de vous oïr et savoir. Et par toutes sifaites choses, la dame, qui ma feme est, ne se partiroit por rien qui fust de moi. Anchois dist que elle o moi vieut morir et vivre. Et d'autre part, ma conscience ne me porroit adoner que sans son gré peuisse ja bone besoigne faire ne asouvir. Pour queile raison je me douteroie que se elle ert sans moi et je sans li, que li uns de nos \num{ii} ne forvoiast d'aucune chose plus legierement. Et por ce qu'il plaist a Nostre Signor que jou ai ceste mue bieste a compaignie de grant piece, je ne porroie mie convierser entre comunitei de gens por la paour qu'il aroient de lui. Me vorroie je demourer en cest desiert et vivre de ma labour en aucune maniere, ausi come il font aucun preudonme qui mie n'ont de coi vivre sans le cors traveillier.>> -- <<Beneïçon aiie de Diu ! dist li preudome. Coment poriés vous endurer le travail de labour, qui onques riens n'en feistes ?>> -- <<Amis, dist li emperere, or ne savés vous que vous avés dit. Ja ne savés vos que je ne sui fais d'autre matere ne de plus noble come nostre premiers pere \persName{Adam}, qui vescui de labor, ausi come les escritures dient\fnnote{"Tu mangeras ton pain à la sueur de ton front" Genèse 3, 19.}. Et savoir le doit chascuns ausi come par nature. Et d'autre part, jou ai entendut que Nostre Sire dist que qui ne labourra ja, es chieus n'entera. Et por cesti raison, je n'en viel mie iestre \add{^hors^} mis, car asés sui grans et fors por porter un fais ou autre labour faire, dont li aucun me donront aucune chose qui le cors soustenra a penitance faire.>> -- <<Sire, dist cil, bon cuer et bone volenté aves. Or vous doinst Dius avenir a vraie labour qui vous maint al regne de paradis.>> -- <<Or me dites dont, dist li emperere, en queil liu je poroie mius demourer por teil vie mener, come vos ci avés oï.>> -- <<Sire, dist cil, a \num{ii} liues de ci a un chastel ou il a une bone vile et sai de voir que on \surplus{n} i fait maintenant une noviele eglise de \persName{saint Nicholas} ou vous arés a faire s'il est ens\del{i com il coveroit} que vous fussiés en liu ou vos peussiés repairier a vostre osteil, ensi come il covenroit.>> -- <<En non Diu, dist li emperere, bien ira la besoigne, mais qu'il plaise a Nostre Signor.>> Lors avint que cil preudom s'avisa d'un liu qui ert priés d'iauwe\fnworknote{"proche de l'eau" ou "proche d'eux" ? Dans ce dernier cas, il faudrait corriger.}, mais que mout i avoit sauvage liu. Et ce demandoit li emperere por ce qu'il n'avoit mie mout a faire que nus s'i enbatist qui lui peuist destorber.


\pnum Li emperere et li empereris, ausi come je vous ai dit, quant il eurent le liu choisi, mout lor pleut et abielist, car c'estoit auques priés d'une liue de \placeName{Gomor} (ce fu li chastiaus dont li preudom lor avoit fait mension devant) et ert auques sor la voie qui venoit del rechiet dou preudonme \surplus{qui venoit del rechiet} au chastiel de \placeName{Gomor} desus dis, liqueus estoit a \num{xii} li\del{e}ues de Patras ou li beneois cors \persName{saint Nicholas} repose, et est ausi come sor la marine \fnfpz{Comment comprendre cette forme ? C'est le participe passé d' 'espartir', mais il faut ici un cas sujet masculin singulier : peut-être 'espartiés' ? Mais il faut noter cette forme dans l'introduction linguistique (morphologie verbale) et voir si on a d'autres exemples dans le texte...}espartiés deviers soleil levant. La isi, come je vous ai desus avisé, fist li emperere faire asés une petite chiele por lui demourer et l'empereris ausi. Dont il avint que, quant il eut sa chosete arré ausi come il li pleut mius, que tout ausiment come uns hom diroit mius a un autre, il coumanda le liuon a warder l'empereris, par coi il ne se meuist devant ce qu'il seroit repairiés. Et sans doute que tout ausi sagement come il l'entendist, il s'ala gesir a l'uis de \foliomar{9va}{https://gallica.bnf.fr/ark:/12148/btv1b10023851v/f24.image} \foliomarID{9v} la chiele, et li emperere torna sa voie et vinrent entre lui et le preudome desus dit au chastiel de \placeName{Gomor}. Et \placeName{Gomor} avoit non li sire a cui li chastiaus ere, qui asés novielement l'avoit fait fermer. Li preudom dont li emperere estoit acointiés ert ouvrriers et mout bons maistre de filés faire et de rois a peschier en douce aige. Et quant il les avoit fais, si les aportoit au chastiel por vendre, si que de ce vivoit li preudome mout saintement, et dou sorplus qu'il ne despendoit mie, dounoit a l'eglise de \persName{saint Nicholas}. Li emperere, par celui preudome, s'acointa de chiaus qui avoient l'ueuvre entre mains et lor dist que, por le \persName{saint Nicholai}, il le meissent en l'uevre, et il se travilleroit a ce que de porter piere, meutier, çaus et savelon. Quant cil ont veü l'empereour, si esgarderent en lui toute biauté parfaite et disent : <<Biau sire, nos ne veons mie en vos que vos teus soiiés que por faire teil manuevre que vos dites. Anchois iestes mius dispo\del{s}sés a tenir un empire et un roiaume por tenir loiauté de justice, dont li pluisor sont au jour d'ui mout deciut.>> Quant li emperere eut celui entendut, si mua une coulor mout merveilleuse, et dist : <<Ha ! Biau signor, merci, ne puent mie tout iestre emperere ne roi, ne doit on mie avoir en despit les laboureurs ne les gaingneours de mestier. Mius aim a gaignier mon vivre a le suor de mes menbres et l'ame sauver, que je fusse empereour ne rois et puis vivre es delisses et mengier les morsiaus dou sanc de la suour de ciaus que je voi chascun jor traveillier por le cors soustenir en santé, ausi come font maint en i a.>> Dont quant li emperere eut ce dit, si seurent vraiement que ce fu un sains home, et l'ont detenut et mis en l'uevre a faire ce qu'il li pleut. Lors entendi li emperere a siervir machons de piere porter et de meutier faire, si come cil qui avoit en lui grant sens natureil. Si veoit pau de choses faire c'uns autres feist que ne seuist faire. Et d'a\supplied{u}tre part, nus ne peuist soustenir le fais qu'il portoit, si que maintenant de ce fu grans renomee. Dont il avint que cil qui les ouvriers paioient esgarderent a çou et l'apielerent si com il se fist apieler. <<Biaus amis desconeus, nos vos voloumes paiier selonc çou que vos ouvrés. Bien nos samble que vos autant faciés tous seus come ceus \num{iii} ouvriés avons nos. Et por ce, si volons nos que vos aiiés treble louier.>> -- <<Ha ! Biau signor, dist li emperere, merci. S'il est ensi que Nostre Sire si m'a fat plus grant grasce qu'il n'ait fait un autre, por ce ne veil jou mie avoir grignor leuier c'uns autres, car nient autrement ne sui jou c'uns seus hom come est uns autres. Et s'il est ensi que je mius desierve, si me consaut li beneois confiés \persName{sains Nicholas}.>> Quant li maistre de l'eglise oïrent ce, si l'ont mout prisié entr'iaus et disent que çou ert uns sains hom.


\pnum \lettrine[lines=2]{\color{darkred}E}{nsi} demoura li emperere grant piece en ceste labeur et revenoit chascune nuit en sa chiele aveuch l'empereris, qui il anoia, et la matinee a la jornee se levoit et ala a l'uevre, et li lions gardoit le jour l'empereris et la nuit alait em proie et revenoit a la journee issi come li emperere devoit aler a l'uevre. Que vos iroie ore plus delaiant ? Avint ensi come il pleut a Nostre Signor. Se senti enchainte et ne le veut mie celer a son signor por toutes aventures. \colmar{9vb}\colmar{b} Quant li emperere eut entendue l'empereris, si coumença a sourirre, et dist : <<Dame, loués en soit c'un qui nos fist et forma et qui aioie, vos en otroie a faire delivrance, par coi ce puist iestre fruis qui puist acroistre le sien non.>> -- <<Sire, ce dist la dame, Dius vos en oïe.>> Lors dist li emperere : <<Dame, il me plaist que vos mout soigneusement vos maintenes, car li cuers me dist que cis fi\del{r}us que vos portés venra encore a grant perfection d'ounor. Et por ce me plaist par honesté que nos puissons trever aucune bone puciele qui vos puist porter compaignie, qui tout le jour si seule ieste chaiens.>> -- <<Sire, dist elle, mout me plairoit, mais qu'il ne vos deuist anoiier.>> Ensi avint ke li emperere s'en vint un jor a li ermite deseure dit, et li a jehi çou qu'il avoit mestier d'une puciele qui sa feme compaignie bone peuist porter. Cil s'avisa d'une povre puciele qui mout estoit de bone vie et de sainte, a son avis, et celi li fist avoir en asés brief tierme. Quant l'empereris le conut, mout li pleut sa maniere, et avint ausi come asés nue chose est de feme qui tous jors\fnnote{Ici, il faut comprendre "tout le jour"...} est oiseuse, que elles ont fait que elles eurent un travile, en coi l'empereris coumença a faire juiaus d'or et de soie les plus riches, que onques en nule tiere peuissent iestre fait ne trové. De coi il avint que, quant elles les avoit fais, la puciele les portoit vendre a \placeName{Gomor}, et n'eles faisoit chose que ele ne les vendist.


\pnum Dont fu par le chastiel une renomee merveilleuse de cest ouvrage, qu'il ne fust nus ouvriers de pointure ne de taille qui peuist ataindre chose que l'empereris ne feist de coulours de soie si au vif, que tuit li ouvrier qui ce veoient disoient : <<Nous ne cuidons mie que çou peuist iestre fait de mains d'ome ne de feme, anchois samble mius chose soushaidié.>> De coi il avint que ceste noviele envint a la chastelainne dou chastiel, qui vit unes parures d'une aube. <<Beneïçon aiiou de Diu ! dist la chastelainne, qui est cele qui teil chose fait ?>> Dont li fu dit : <<Dame, \add{^une^} puciele les aporte de hors vile, mais nous ne savons qui çou a fait.>> -- <<Par Sainte Crois, dist elle, savoir le veil.>> Dont avint un jor apriés que la puciele aporta autre juiaus en la vile, par coi elle fu espiie et menee devant la chastelainne, et li fu enquis do\supplied{n}t teil juiel venoient ne qui les avoit fais. Cele cui il fu desfendut ne l'osa dire, et dist : <<Dame, por rien ne le diroie, car il m'est desfendut.>> Quant la chastelainne eut çou oït, dont fu plus engrant dou savoir çou que vos poves oïr, mais onques par chose ne par manace\fnnote{Passage du «e» initial à «a» : trait picard, voir Gossen #29 et Claude Régnier p. 264.} que on li seuist faire iehir ne le volt ce que on li avoit desfendut. <<En non Diu, dist la chastelainne, or savons nos de voir que sifaite chose ne vos vient nient de bon liu.>> -- <<Maisement le savés ore>>, dist la puciele.


\pnum \lettrine[lines=2]{\color{darkblue}E}{nsi} avint que cele fu isi detenue de ci a dont que li chastelains, qui mie n'ert en la vile, deut revenir. Et quant li emperere, qui ert a l'uevre, vit que la puciele ne revenoit a lui ausi come elle soloit faire, si en fu mout merveilleus. Si fu li eure que il deut \del{ent} repairier en son rechiet et il si fist. Et quant il ne trova celi, dont demanda a l'empereris se elle ert nient repairié. <<Sire, dist la dame, saciés que je puis ne le vi que elle de chaiens se parti aveuch vos.>> Dont ne seurent que penser, si furent mout dolant que elle n'euist aucunne chose qui li anuiast. Et d'au\foliomar{10ra}{https://gallica.bnf.fr/ark:/12148/btv1b10023851v/f25.image} \foliomarID{10r}tre part, il l'amoient mout come cele qui ert de grant siervice a merveille et amiable. <<Dame, dist li emperere, ne vos en destorbés mie, car il ne puet avenir que nos n'en oïons prochainement aucune noviele.>> Ensi demoura de ci au matin que l'emperere vint a so\supplied{n} ouvrage. Et li chastelains revint d'autre part, cui la chastelainne eut mout tost acointié de la puciele çou que vos avés oï. Quant li sires le seut, si fist celi venir devant lui et li dist : <<Damoisiele, por coi ne dites vos dont teus ouvrages vos vient, que vos alés vendant par cest chastiel, car il atient a nos que nos de sifaites choses sachons la verité.>> -- <<Sire, dist la puciele\surplus{s}, tout est verités çou que vos dites. Et d'autre part est il ausi usages que, quant que aucun prodome ou aucune bone dame a aucun vallet ou meschine en cui il se fie et en celi fiance ne puet avoir se tous biens non, et puis li coumande sa privauté a faire, dont il ne veut mie que parole avigne dont il puist avoir destorbance ausi come il poroit avenir de cesti chose, et puis le diroie mal sagement.>> -- <<Amie, dist li chastelains, jou aucune fois veut ai de plus foles que je ne cuit que vos soiiés, mais aucune couleur autre que vos n'aiiés encore dite me covient savoir, par coi jou en soie autrement en me pai\del{i}s que je ne soie.>> -- <<Sire, dist elle, et je vos en dirai par raison que je porrai savoir :>>

\pnum <<\lettrine[lines=2]{\color{darkred}V}{oirs} est qu'i a en cest paiis une dame de grant relegion plainne qui mie ne veut que li conmu\supplied{n}s sace ou ele demeure, et i a auques boinne raison, cele de que ele puet espargnier deseure sovivre\fnnote{"elle cache une partie de ce qu'elle peut épargner sur ses besoins vitaux"}, si m'en fait achater l'estosfe a faire teil ouvrage come vos avés veu. Por coi il m'est desfendu que je n'en die chose qui destorber le puist, et si vos en sousfisse ore ce que je vos en ai dit car, por rien qui vive, je ne vos en diroie el que vos poves oïr.>> Li chastelains, quant il ot ce oït que cele dist, si l'en seut mout bon gré et en prisa mius son afaire. Il li dist : <<Amie, alés a Diu et vees ici un besant que je vos doins, por le raison de ce que vos avés si sagement escusee vostre dame.>> -- <<Sire, ce dit la puciele, se vos saviés come ja le doi bien faire, vos i ariés bien vostre pais.>> -- <<Amie, dist il, por ce que je ne le soi mie, si m'en anoie, car ja por moi ne cuit que elle en deuist mains valoir.>> Atant s'est cele partie dou chastelain et s'en revint a son signor, qui a merveile fu en grant doute de li, si come cil qui mout fu joians quant il le vit. Dont li conta cele qu'il li fu avenut, que de rien ne li menti. Et li emperere l'en seut a merveille bon grei et mout le prisa en son cuer, et dist : <<Amie, ensi doit faire bons siergans envers son maistre, dont il bee a avoir bon leuier, et je grant doute avoie de vostre anui.>> Ensi demoura la puciele en la vile de ci au viespres que li emperere s'en ala en son rechiet et enmena la puciele, dont l'empereris fist mout grant joie q\supplied{ua}nt elle le vit. <<Beneïçon aiiou de Diu ! \persName{Nichole}, ou avés vos demouré ?>> ce dist l'empereris. Dont li conta cele tout de fil en aguille ce qu'il li fu avenu, et quant elle l'eut entendut, si dist : <<Sire, se vos ne voles que on sace que nos soiomes ci, je ne bee mie que on plus porte de mon ouvrage en la ville.>> -- <<Dame, dist li emperere, il me plaist que vos plus n'i envoiiés, car je me douteroie d'empeecement.>> Dont demoura atant \colmar{10rb}\colmar{b} ceste chose de ci a dont que li chastelains manda un jour l'empereour come cil qui avoit envoiiet apriés la puciele, ausi come de çou ne fust nient. Et cil qui çou avoit enquis fu envoiiés, ensi come je vos ai dit, a l'empereour, ki li dist de par le chastelain que il venist a lui parler. Li emperere respondi au mesage : <<Amis, ja ne vees vos que je ne sui mie a moi, et qu'il anoieroit ces bones gens, de cui je sui tous ensouniiés, se je me departoie ore d'iaus en teil maniere.>> -- <<Biau sire, dist li siergans, ne vos escusés ore mie de çou, car mesire se puet bien tant fïer de ciaus vos \fnfpz{sens : 'avoir confiance au sujet de ceux à qui vous appartenez' (ciaus : cas régime absolu, COI sans préposition)}iestes que vos hardiement i povés venir sans mesfaiture.>> -- <<Amis, dist li emperere, b\supplied{ie}n cuit que cudiés voir. Mais il avient bien que qui ne fait volentiers la chose, avisés doit cil iestre de lui escuser.>> Quant cil entendi l'empereour, si coumença a sourrire et li respondi : <<Biau sire, je vos responderai ci encontre : ``autreteil est il au chaitif mesage cui on envoie en aucun liu besoignier, que, quant il avient que on ne li respont mie a sa volenté, qu'il raporte bien tous novieles par son povrement besougnier, dont il lui ou autrui empeece anieusement.'' Por coi je ne veil mie iestre de ciaus por ce que je veille que vos i pierdés, et mesire se chourechast a \fnfpz{Sens ? Il me semble qu'il manque une négation dans la proposition introduite par 'por ce que'... 'je ne veux pas faire partie de ces messagers, pour vouloir que vous y perdiez, et monseigneur s'en prendrait à moi'. Je comprendrais mieux : 'je ne veux pas être de ces messagers: pour peu que je ne veuille pas que vous y soyez lésé, mon seigneur se mettrait en colère contre moi'... Mais il n'y a pas de négation... A moins que ce ne soit pas 'de ciaus', mais 'decraus' = decreüs, 'mésestimé' : C'est pourquoi je ne veux pas être méprisé pour vouloir que vous y perdiez. Mais on ne comprend plus le 'et', on attendrait plutôt 'mais' : mais mon seigneur se courroucerait contre moi...}moi.>>


\pnum \lettrine[lines=3]{\color{darkblue}L}{i} emperere, quant il oï celui ens\supplied{i} respondre, si vit qu'il n'i eut point d'ecusement, car cil en vint a celui qui bien avoit povoir de lui delivrer et li a dit : <<Alés a mon signor parler qui vos mande.>> Li emperere s'est apresté et vint devant le chastelain en mout noble contenance, et il s'en apiercuit mout bien a lui saluer, et il li rendi son salut et li dit mout anbleme\supplied{n}t por lui tempter : <<Qui ies tu, qui n'ies mie disposés a cest mestier que tu as empris a faire ?>> Li emperere ne s'abaubi mie, si come une fine merveille anchois dist : <<Biau sire, par Diu merci ! Ja ne vees vos dont jou sui uns hom gaingnans ausi come uns autres ?>> -- <<Autresi com uns autres n'iestes vos mie, ce puet chascuns veoir !>> -- <<Par Diu, sire, voirs est que on pau troveroit de teus come je sui, mais une maniere est de parler a coumune raison.>> -- <<Ce sai je auques, dist li chastelains. Il ne covient mie que vos m'aprendés a parler, car je voi bien a vostre maniere que vos iestes chevaliers, et sai de voir que je vos ai asés veu, mais je ne sai en queil liu.>> Dont s'abaubi un poi l'emperere et li respondi : <<Chiers sire, mout a de gent par le paiis qui ne cacent a nului se tous biens non qui mie ne vorroient que tous li mons seuist qui il fussent.>> -- <<N'est mie merveille, dist li chastelains, mais que sui sire de ceste ville : puis et doi bien savoir qui cil sont qui en ma tiere repairent, por tant que je le veille savoir.>> -- <<En non de moi, dist li emperere, je ne cuer ja que vos ne autres sace que jou sui autres que vos avés oï.>> -- <<Par mon chief, dist li chastelain, savoir le m'estuet u vos demorés devers moi.>> -- <<Et coment, dist li emperere, le porriés vos savoir se autres que jou \add{^ne^} vos en faisoit sage.>> -- <<Bien vos en \fnfpz{Croire : P1 ind. fut. Graphie picarde. Confusion possible non seulement entre les tiroirs du futur et du cond, mais aussi entre les verbes «croire» et «querre». Voir Gaston Zink Morphologie «métathèse dans les séquences -rer-»}kerrai sor vostre parole, dist li chastelains, mais qu'il i ait coulor de raison.>>

\pnum <<Sire \fnfpz{N'est-il pas étrange que l'empereur appelle 'empereur' celui qui n'est que châtelain ? Ne faudrait-il pas corriger ? Ou du moins mettre une note pour commenter... (proposer une explication... : simple erreur - du copiste ou de l'empereur qui parle ? flatterie ?)}emperere, dist il, mout avés grant tort et ne vos anoit se je le di, et vos dirai por coi il est voirs que vos me metés en \foliomar{10va}{https://gallica.bnf.fr/ark:/12148/btv1b10023851v/f26.image} \foliomarID{10v} voie de mentir, car se je vos disoie qui jou sui, espoir vos ne le kerriés mie, por ce qu'il n'i aroit mie boine color de verité. Et d'autre part, s'il avenoit que je vos en deisse le contraire et i euist coulour de raison, vos le kerriés.>> -- <<Tout çou est verités, dist li chastelains, et por ce que je ne veil que vos ne diiés se verité non, vos conjure jou que se vos vorriés que ja Dius vos pardounast mesfait que vos feissiés, que vos me dites en queil tiere vos fustes nés de mere ne por queil raison vos iestes en teil point. Et je vos jur sor l'ordene de chevalerie que, se vos ne le mesfaites de ce jour en avant, que ja ne vos en iert piis se mius non.>> -- <<Sire, dist dont li emperere, pus que vos m'avés ensi aseuré, et d'autre part conjuré, je vos dirai auque qui jou sui, pus que je n'en puis par autre tour eschaper. Voirs est que je sui nés en la citei de Coustatinoble et sui fius a l'empereour \persName{Lauron} et de ce meime liu. Et a pleut a Nostre Signor que empres men pere ai tenu l'empire tant qu'il me souffist que je m'en sui partis, isi come vos poés voir, par la raison de ce que il ne puet mie avenir legierement que je trop ne me soie abandounés a pechié. Por coi il plaise a Nostre Signor que je, en ces parties, soie venus por faire ma penitance, et il vos plaisse que je puisse demourer a l'honor de Diu et de ciaus entor cui je veil convierser.>> Quant li emperere eut çou dit, li chastelains failli em piés et se laissa de si haut come il fu chaoïr as siens, et dist : <<Ha ! Hom de tres grant vertut ! Voirement ne povoie mais que \add{^se^} je disoie que je vos avoie veü aillors quant je fui en piece de tiere ou vos outrastes \persName{Askaron} devant \placeName{Thiberiadis}, qui ne fu hom en toute l'ost mon chier cousin \persName{Hedipum} qui contre lui s'osast armer, se vos cors non.>> Adont releva li emperere le chastelain, et dist : <<Biau sire, or soiiés sages de çou que je vos ai ci dit que je me die voir, car de celui \persName{Askaron} dont vos parlés ne veil je mie tenir mout lonc conte. Mais s'il a en vos sens, honor ne cortoisie, je vos prie que ceste chose soit celee, et faites ausi que nient ne tenés de moi chose dont nus sace que que jou soie autres que je me monstre.>> -- <<Sire, dist cil, por Nostre Signor, sousfrés moi une parole a dire. N'aiiés doute que ja nus sace qui vos soiiés, fors une chose me sousfrés qu'il me covient que je face et ce ne vos doit desplaire a ce que vos avés a faire, et vos dirai quele : en couleur et en espesce que chascuns prudome si doit honourer tous chiaus qui au dehors et au dedens mostrent qu'il soient siergant de Diu, on doit honorer \del{e} tous ciaus d'autre part qui sont estrange, et ce veil je que vos saciés que je veil que vos o moi soiiés au plus sovent que vos porrés. Et d'autre part je vos pri en amor que vos me dites v\supplied{ost}re rechiet, por savoir mon\fnnote{Adverbe «certainement», construction savoir mon + interr. hypo.} se il vos fauroit rien chose que je vos peuisse faire par vostre pais, car je cuit bien que vos ne vorriés mie prendre tous le avantages que je vos vorroie faire.>> -- <<Ha ! Sire, dist li emperere, par Diu merci ! Je vos pri par celui qui nasqui de la beneoite \persName{Virgene Marie} que vos de moi fachiés ausi que vos ne sachiés qui je sui, car autrement me covenroit partir de ci, car vos povés bien savoir que se je vausise demourer en liu que on me coneuist, je ne fusse mie demourer\fnnote{Un pp en -er} ci endroit ne venus se uns jors me deuist porter de joie autant come jou onques en ai. Et si vos sousfisse ore par vostre humilité çou que je vous ai dit, de ci a do\supplied{n}t que je porroie avoir mestier de vos, fust por moi u por autrui.>> -- <<Sire, dist li chastelains, et jou l'otroi.>>

\pnum Atant prist congiet \colmar{10vb}\colmar{b} li emperere au chastelain et envint a sa manuevre. La chastelaine envint a son signor et li dist : <<Or me dirés vos que cil bons hom est.>> -- <<Dame, dist il, boins et hom est il ne je ne cuit mie que en tout l'empire de \placeName{Rome} ne de \placeName{Coustantinoble} ait si saint home que je cuit qu'il est.>> -- <<Coment, dist elle, le savés vos ?>> -- <<Dame, dist il, ne me demandés autre chose de lui que je vos \supplied{di} ja. Autre chose ne vos en dirai.>> -- <<Coment ! dist elle, avés vos covena\supplied{n}ces a lui que vos ne volés mie que je sace ?>> -- <<Dame, dist il, n'ai jou.>> -- <<Par mon signor \persName{saint Nicholas}, si avés ! Et s'il est ensi que je n'en sace la verité, mal ira li afaires !>> -- <<Dame, dame, dist il, sousfrés de ce dire, que foi que je doi a l'ame mon bon pere, que se jou savoie que vos en feissiés chose \supplied{que vos} ne deissiés, je vos courecheroie dou cors en teil maniere que il vos dorroit a tous jours ! Coment diable covient il que un prudom ne une prudefeme ne porra mie vivre en ceste ville que vos ne veilliés savoir sa confiesse !>> Quant la dame oï que se sire se courecha de çou qu'il avoit dit, si se pensa qu'il ne faisoit ore mie bon de lui esmouvoir, si dist : <<Mout avés ore une estrange maniere que on ne puet a vos parler, que vos ne vos courechiés maintenant. Ja avés vos esté en ausi grans savoir come je sui ore.>> -- <<Et por ce, dist il, que je sui engrans dou savoir le savoir et n'est mestiers que vos le sachiés.>> -- <<Sire, dist ele, puisqu'il m'en couvenra sosfrir, si le fe¶rai.>>


\pnum \lettrine[lines=2]{\color{darkred}E}{nsi} demourerent adont lor paroles. Mais li diables qui, par le malvois, cunchie maint preudome, mist en la chastelaine une pensee qui mai\supplied{n}te anui et pluisor destorbier avenut sont au monde, car cele, qui mie ne creoit bien son signor, eut une pensee qui tele fu que elle cuida vraiement que cil hom qui ert issi la mandés euist une acointe qu'il vosist veoir avant que nus le seuist qui elle fust. Et de ce ne la meist nus hors, si estoit elle mal pe\supplied{n}sans\fntooltip{V3}. Et qu'en avint il ? Apriés ce \num{ii} jors mist elle le chastelain a parole de çou que elle li demanda : <<Et coment, sire ! Ne demandés vos mie a ce saint home qui cel ovrage avoit fait ?>> -- <<Dame, dist il, or sachiés que je non, car si d'autre chose l'ensenniai\fnnote{&gt; *insignare }, que de cestui afaire ne li osai demander. Mais bien i cuit recovrer.>> -- <<Sire, dist ele, bien vos en croi.>> Lors demoura chose ensi et avint que li chastelains aloit aucune fois u li emperere se travilloit, ensi come desus a esté dit, por savoir se il li vausist aucune rien dire. Et sans faille que mout couviertement li uns parloit a l'autre, por çou qu'il ne voloient mie que nus se percheust que çou fust si grans chose de lui come il ert. Et de çou li savoit li emperere mout grant grei, car il veoit apiertement qu'il ne fust chose, se il li quesist que il apareilliés ne li fust dou faire. Et qu'en avint ? La dame, feme au chastelain, cui li diaubles avoit pris en cure, et vit qu'il ne povoit mie venir a chief legierement par autrui que, par celi qui en jalousie estoit si grant que bien i parut pus, vint a un sien escuier, et li a dit : <<Il covient que tu me faces une chose que je te dirai, par maniere qu'il n'iert chose nés de mon cors viergonder que je ne fache por toi, et te dirai quele. Je sui toute chiertainne que cil païsans veut mon signor atraire, liques mesire tient por si saint home, et ai doute qu'il n'ait une ne sai quele chanlande enchiés lui, qui cel ouvrage a fait que nos avomes veu, qui si parfaitement est biens fais, que tous li mons s'en doit avoir mervelle. Et s'il est ensi que tu i puisse\supplied{s} chose savoir ne voir que tu cuides qu'il me doit iestre celé, si le me di, et je ferai do tot ta vo\foliomar{11ra}{https://gallica.bnf.fr/ark:/12148/btv1b10023851v/f27.image} \foliomarID{11r}lenté.>>


\pnum Cil qui mie ne fu loiaus siergans, et fu covoiteus de faire la dame sa volentei, et qui plus engrant estoit d'oïr male novile que de bone, li dist dou coumenceme\supplied{n}t : <<Dame, s'il estoit ensi que je traeïsse mon signor por vostre volenté acomplir; mout seroit estrange chose, car bien savés que messire et li vostres n'a siergant en son osteil cui il croie autant com il fait moi. Et ne cuit mie que en aucun maniere il ne l'ait trové en moi. Por coi, ma tres chiere dame, je vos dirai une chose a coi je cuit que vos vos i acorderés, et tout par coulor de raison. On dist en provierbe, et c'est voirs, que ``qui ainme mius de mere, cel est fole norrice''\fnnote{Peut-être par référence à Amour de mère est plus grande que de nourrice», Perceforest, III.}. Por coi je di que nus ne doit amer tant \del{a} mon signor que vos meime, et par ce que je cuit que vos plus l'amés de moi, si gardés la soie honor et la moie, car de ce que vos me querés, je ferai mon povoir por la vostre volenté aconplir, et por ceste chose mius celer de vos a lui et de lui a moi. Par coi nule parole n'en peuist issir hors.>> -- <<\persName{Fastre}, ce dist la dame, de çou ne sui jou mie a aprendre, que ce ne covenist faire.>> -- <<En non Diu, dame, dist cil, de ce me plaist mout que vos en iestes avisee, a ce verrai je volentiers avant que jou gaires vos en die chose dont je soie blasmés de mon signor.>> Quant la dame eut celui entendut, si visa et seut a coi cil beoit, mais il dou tout ne li osoit dire son corage. Si pensa que se elle voloit esvoiturer sa fole pensee, que li covenoit faire un vilain meschief de son cors, çou dont ele n'euist cure, n'euist e\surplus{n}sté la dolante jalousie, dont je desus ai fait mension.


\pnum Lors avint de ceste chose une laide aventure, car la dame estoit mout anieuse que ele ne povoit avoir nul oir de son signor. Si avint de çou que li diaubles, ausi come jou ai dit deseure, ne beoit mie sans plus un seul a cu\supplied{n}chiier. Anchois avint a celui \persName{Fastre} et li dist : <<Amis, pluisors raisons me mainnent a çou qu'il me covient que je soie plus privee de vos que je ne soie de mon signor, de coi je ne pus avoir oir de lui, qui la tiere doie tenir empriés lui. Por coi je sui et ai esté mout destorbee, por ce que je cuit vraiement qu'il ne m'ainme mie ne prise come il deveroit\fnnote{Exemple de «e svarabhaktique» dans un cond.} et sai tout vraiement c'autres femes il aime mius de moi. Et por toute sifaites choses et autres que je vos ai dites, me covient que je m'abandoins a faire vostre plaine volenté et la moie.>> -- <<Dame, dist cil, ne voi en ceste chose nule rien dont vos puissiés iestre blasmé d'ome qui s'entendist, et ma vole\supplied{n}tés n'iert autre que por metre, et tors ne vos fauroi a mort ne a vie.>>


\pnum \lettrine[lines=2]{\color{darkblue}Q}{ue} \fnnote{visage sévère dans la lettre} vos iroie ore celant lor afaire ? Ne regarda li uns ne li autres droiture ne honor que li \fnfpz{Exemple d’une résolution d’abréviation casse-tête : on lit livꝰ &gt; livus &gt; li vus &gt; li uns. Le ꝰ aurait ici une valeur graphique «ns» ???}uns et li autres n'euist a faire ensamble en teil maniere que mout plaissoit a l'un çou que li autres faisoit. Dont il avint que mout volentiers li escuhiers meist son signor en voie, par coi il peuist dire a sa dame chose qui euist aucune coulor de verité, qui a ce touchast, dont li dame avoit fait mension. Por coi il avint un jour asés tost apriés ces avenues que li chastelains tenoit l'empereour mout coviertement de parole, et avint qu'il dist : <<Sire, se il ne vos devoit anoiier, a merveille volentiers iroie en vostre rechiet por savoir se il vos i fauroit chose qui bone vos i fust que je faire i peuïsse.>> -- <<Sire, dist li emperere, et je veil, puisqu'il vos plaist que vos i vigniés, mais que nus o nos n'i vigne qui ne soit vostre g\supplied{ra}ns secrés.>> -- <<N'en doutés, sire, fait cil.>> Lors \del{de} ne demoura que li empereour et li \fnfpz{cas}chastelain meime, cil de coi j'ai desus dit. Si n'ont finé tant que li emperere les amena en son rechiet, ou il faisoit a merveille neït isi come en teil \colmar{11rb}\colmar{b} liu. Quant l'empereris vit le chastelain, si mua une grant color si come ce ne fu mie merveille, nonporquant le fist elle mout noblement bienvignant et il li, mout humlement l'embraça, et dist : <<Ma tres chiere dame, loués en soit li glorieus Dieus de paradis, qui tele honor m'a faite que je poroie en ces paiis ne dire chose qui vos peuist plaire !>> -- <<Sire, dist ele, de la vostre grant humileté vos sace Dius grei.>> Lors sont asis tout \num{iii} li un asés\fnnote{Signe abréviatif unique, résolu naturellement par "s".} priés de l'autre, et avint que li emperere fist signe au chastelain qui il fust arriere traite son escuier por plus priveement dire ce qu'il lor plarroit, et il si fut. Quant cil fu hors de la chiele et li liuon l'eut senti, si conmença a grondir et l'euist mout tost devoré come cil ki par nature savoit qu'il ne chaçoit nul bien, quant cil sailli arrier en la chiele, come cil qui bien cuida iestre mors, et cria : <<Ha ! Biau signor ! Tuit soumes mort !>> Quant li emperere oï ce, si sailli em piés ausi come en \add{^sour^}riant, et vint au liuon et prist une vergielle, et dist : <<Que c'est, sire compains ? Traiés vos arrier, et gardés que vous ne faciés nului mal en liu ou je soie !>> Et por ce ne laissa il mie a rugier ne a grondir ausi come di me tu : <<Vos n'iestes mie venus por bien, et me poise que je ne vos puis paiier vostre desierte !>> Et li emperere s'en apierchuit, et hucha le chastelain qui \surplus{avoit} en avoit teil paor qu'il a poi n'issoit dou sens. Et li emperere dist : <<Sire, alés hardiement a mon signor qui vos apiele.>> Il i ala \del{u ele} envis, mais il de honte ne l'ossa laissier, et quant il i fu venus, li emperere dist : <<Orés gardes de nature et n'avés doute !>> -- <<Ha ! Por Diu, merci, dist li chastelains. Que vos plaist que je face ? Jou ai si grant paour de cest liuon que je me muir.>> -- <<Il ne vos fera mal, dist \del{dist} li emperere. Prindés le hastivement par la grigne et si l'aplanoiiés.>> Et il si fist. Maitennant il s'umilia lui et se mist a ses piés, et li emperere dist : <<Or vos traiiés arriere, chastelain, et me faites avant venir nostre siergant.>> Et il si fist. Et quant li liuons le \surplus{le} vit, si fist samblant de lui devorer, et li emperere le maneça, et \del{i} il se tint ausi chois com uns aignaus.


\pnum \lettrine[lines=2]{\color{darkred}Q}{uant} ce vit li chastelains, si eut teil merveille de cel liuon dont il venoit qu'il ne seut s'il fu en ciel ou en tiere, et dist : <<Ha ! Ge\supplied{n}tius sire, merci ! Dites moi dont cil liuons vient, et s'il est vostre.>> -- <<Par mon chief, sire chastelain, vos m'avés fait une mout corte demande, mais a merveille i aroit a dire avant que vos en seuïssiés la verité dont il vient ne cui il est. Mais avant deuissiés demandé çou qui greignor mestier vos puet avoir.>> -- <<En non Diu, sire, si esbahis sui que jou auqueil ne sai entendre.>> Lors prist l'emperere le chastelain par la main et l'emmena arriere en sa chiele, et sont derechief assis li uns jouste l'autre. Et dont apiela li emperere l'escuhier et li dist : <<Mon ami, n'euïstes vous ore mie mout grant paour ?>> Cil respondi que ce n'avoit mie esté mout grans mierueille. <<Verité dites, dist l'emperere. Je vos lo bien que vous ne autre qui n'ainme l'ounour et le preu de chiaus de chaiens, qu'il ne si enbatent mie, car il porroeent asés tost pierdre.>> Quant li esc\del{h}uhiers oï çou, si en fu mout abaubis, et dist : <<Sire, se Diu plaist, je ne sai home \foliomar{11va}{https://gallica.bnf.fr/ark:/12148/btv1b10023851v/f28.image} \foliomarID{11v} çaiens qui vos veille se bien non, ce m'est avis.>> -- <<Amis, dist li emperere, ce gardés vos, car se vos ne le savés ausi ne fai jou.>> -- <<Ha ! Sire, dist li chastelains, ne cuidiés mie, ne je ne le vauroie por rien que je vos amenasce arme chaiens, que je vausisse que piis nos avenist que a mon cors meime.>> -- <<Or le laissomes atant, dist l'emperere.>> -- <<Dame, dist il a l'empereris, je vos pri que se vos avés aucun jueleit, que vos en dounés aucun mon signor le chastelain.>> -- <<En non Diu, sire, je n'en sui mie senuech\fnnote{G. Roques, R. Ling. rom. 60, 1996, 611, pour l'aspect régional.} que je n'en aie aucuns. Mais il est aucune fois avenut que j'en avoie plus et de plus biaus.>> -- <<Dame, dist l'emperere, or ne vos roue je mie de dire.>> -- <<Ha ! Sire, ce dist li chastelains, ma dame se puet bien voir dire.>> Atant aporta l'empereris \num{ii} au\del{s}mosnieres mout riches qui bielles fussent aveuch son signor, quant il onques avoit esté en la roue plus grant de properité\fnnote{Absence de \textit{s} aprés \textit{pro-}. La graphie existe, bien qu'elle ne soit pas fréquente (TL, 7.1999.34/36, repris par GodefroyC, 10.437a ; FEW, 9.467b ; Matsumura regroupe \textit{prospreté}, \textit{propreté}, \textit{prosperité} dans la même entrée.) L'absence de \textit{s} peut s'expliquer par analogie à \textit{propreté} du latin \textit{proprius}. Au § 280, on lit \textit{a son prope liu}, dont la graphie existe par ailleurs.}, et dist : <<Sire, vees ici de mon ouvrage dou queil ma dame la chastelaine fu si engrant dou savoir qui jou estoie. Si le me salués atoutes ces enseignes que je li envoie ceste aumosniere, et mesire qui ci est vos doune ceste autre, et cil autres damoisiaus si avra cest aguillier en amende de çou que li liuons mon signour li a faite si povre chiere.>>


\pnum \lettrine[lines=2]{\color{darkblue}Q}{uant} li chastelains vit ce, si fu ausi come tous abaubis des riches juiaus que l'empereris li avoit douneis, et ne seut autre chose que dire qu'il respondi : <<Dame, nient plus come li povres hom ne doit escondire le don dou riche, li riche\supplied{s} ne doit escondire le don dou povre quant il a cuer et volenté dou merir. Et por ce nel di jou mie que cis dons soit de povre home a riche, anchois est dons de riche a povre.>> -- <<Castelains, chastelains ! dist l'emperere, encor a a amende\del{r} a ce que vos dites et ne vos anuit se je vos en reprenc.>> -- <<Ciertes, sire, non, fait il, et sans faille que vos verité dites. Mais une chose le me fait dire que encore ne me porroie je tenir que je ne le deïsse que je di et si me di verité que ce n'est mie dons de povre home a riche que vos m'avés fait, anchois di que se li emperere de \placeName{Rome} m'avoit douné ce que vos et ma dame avés fait, se diroie jou que çou seroit dons d'empereour a roi.>> -- <<Or soit ensi qu'il vos plaist, biaus dous sire !>> dist l'emperere au chastelain.

\pnum Ne vos puis mie tout recorder toutes lour paroles, car li contes n'en fait or de plus mension. Anchois covint le chastelain penre congiet, qui mout volentiers fust encore demourés se il osast, mais nennil, car il lor cuidast a anuiier, et sans faille non feist il. Mais li emperere par avoit une conscience si estroite que il ne li plaisoit a prendre nule recreation, se dou mains non qu'il povoit, et ce veoit bien li chastelains, qui a merveilles avoit bien sa pais. Si prist congiet, et dist : <<Sire, partir m'estuet ore de vos tant come a ore, et s'il peuist iestre qu'il vos pleust autant enchiés moi ou je vos feisse faire une chambre tele come il vos plaisist, toute ausi porriés vos faire la ce que vos faites ci.>> -- <<Chastelains, dist li emperere, sauve soit vostre grasce\fnnote{Locution toujours (?) en climat négatif signifiant : « avec tout le respect dû à votre bonté», donc sous-entendu «non merci».}. >> Et quant li chastelains vit ce qu'il ne porroit a chief venir, si dist : <<Au mains sire, ferés vos aucune chose por moi de mon conseil.>> -- <<Quele ? dist l'emperere.>> -- <<Je veil, dist li chastelains, qu'il ait aucu\colmar{11vb}\colmar{b}ne forterece entour vostre chiele, par coi on n'i puist mie venir fors que par une entree por toutes aventures, et cele, ce me samble, sera auques bien gardee tant que vos aiiés le liuon dont je sui mout temptés de savoir coment vos l'avés isi a vostre volenté, come jou ai veu.>> -- <<Chastelains, dist li emperere, n'est mie mout grans merveille, se vos en voliés savoir la verité, mais n'est ore mie poins dou savoir, anchois le sarés tout a tans. Mais de ce que vos avés dit vauroie je bien qu'il fust fait.>> -- <<En non Diu, sire, demain i ferai venir les ouvriers. Mais i covenra aillors metre vostre liuon tant que ce soit fait, si vos vorroie proiier que vos enchiés nous vausissiés venir, car la dame de maison vos veroit a merveille volentiers.>> -- <<Sire, dist l'emperere, ce puis jou savoir mout bien et mout grans merci. Mais ausi come jou vos ai dit, jou redoute mout renomee de gent, que, quant il voient un home ensi come je sui entr'iaus, il i a tant de murmure qu'il ne puet avenir que li aucun n'i facent a la fois povreme\supplied{n}t lour preu.>> -- <<Ha ! Sire, dist li chastelains, ja de çou ne doutés, car puisqu'il ne vos plairait, ja hom ne feme ne le sara, se çou n'est jou et la chastelainne.>> -- <<Sire, dist l'emperere, je ferai vostre volenté et a çou ai ma chose. Et demain, quant il ert a viespri, iroumes cele part.>> -- <<Chiers sire, dist li chastelains, sachiés que çou iert dou millor.>> Atant s'est li chastelains partis de l'empereour et de l'empereris, et se misent au chemin lui et son escuhier, qui avoit mout grant despit dou liuon, qui ensi si ert hierechiés a lui et a son signor nient, si ne se puet tenir qu'il n'en dist une parole dont il fu blasmés de son signor. <<Il me samble, dist li cuviers\fnnote{Le \textit{culvert} appartient au paradigme sémantique du serf, mot qui provient du latin \textit{servus} et qui signifie \textit{esclave}. Il y a loin toutefois de l'esclavagisme latin à la servitude médiévale. }, que ce soit uns enchantemens de ce liuon.>> -- <<Coment ! dist li chastelains, \persName{Fastre}, qu'en dirés vos ?>> -- <<Je n'en dirai autre choise, dist il, que il me samble ausi come uns enchantemans, que cil liuons est ausi come afaitiés a faire ce que se\add{^s^} sire vieut.>> -- <<Et coment ? dist li chastelains, vos samble çou encha\supplied{n}terie ?>> -- <<Par foi, dist cil, oïl.>> -- <<Dont dites vos que cil est uns enchanteres qui l'a entre mains ?>> -- <<Par foi, dist il, jou ne sai que dire.>> -- <<Or en dites dou piis que vos povés, car vos n'avés povoir dou bien dire ne dou mal, fors çou qu'il vos plaist.>> -- <<Voire, en non Diu, dist cil. Or ai jou dit paumoire\fnnote{La transcription semble bonne. Le mot a le sens de "digne d'éloges"} se jou ai dit que il me samble un enchateis. Encore dirai el, car je ne cuit mie que teus hom se partist onques por bien de son paiis, qui lui ait fait aucun vilisse, par coi il ne puet arriester soit de cele dame ravie qu'il a aveuch lui, ou de son signor traïr d'aucuns chas qui mie ne soit covignables. \fnfpz{à reprendre}>> -- <<Voire, dist li chastelains, come la bone dame a ore bien esploitié, qui nos a dounés de ses juiaus.>> -- <<Par Diu, sire, dist cil, voir avés dit, car se je li povoie chose faire ne dire que biele et bonne li fust, mout en seroie joians, car a merveille me samble bone dame et de boin liu venue.>> -- <<Et coment, dist li chastelains ! Ne te samble mie de son signor ?>> -- <<Sire, dist il, si fait. Mais de ce m'anoia il mout quant il nos douna a entendre que li aucuns de nos estoit la venus por son destorbier.>> -- <<En non Diu, dist li chastelains, car ses liuons li douna a entendre. Ja ne vois tu coment il fu meli a devant moi et \surplus{et} toi voloit corre seure.>> -- <<Ja Dieus \foliomar{12ra}{https://gallica.bnf.fr/ark:/12148/btv1b10023851v/f29.image} \foliomarID{12r} ne m'ait, dist cil, se por ce ne di je ore que ce me sambla une maniere de flaterie, tout ausi come di me tu : ``vos iestes\fnnote{Même signe que précédemment, qui ressemble à un 9 mais qui vaut «s».} gentius hom et de noble afaire et cil est vilai\supplied{n}s et mal acoustumés, pensans de mal afaire.''>> -- <<Et jou ai dehé, dist li chastelains, que je ne cuit mie que li liuons sournoiast mie de mout.>> -- <<En\fnnote{Pronom adverbial, COI de "cuidier"} ne cuidiés vos bien, çou dist li escuhiers.>> -- <<Par mon chief, dist il, pus que tu ce as pensé\fnnote{ou 'aspense' ?}, je le cuit.>>


\pnum Ensi sont venut tot deparlant de ci a l'oisteil, et li chastelaine estoit en agait por oïr novieles, si fist ausi come ele ne seuist mie dont il venoient, si dist : <<Sire, il me samble que vos soiiés ausi come tous lassés. Dont venés vos issi a piet ?>> -- <<Dame, dist il, il covient que vos le sachiés.>> Lors le prist par la main et le mena en sa chambre, se li dist : <<Dame, on vos salue par moi et si vos envoi on ceste aumosniere une dame que vos arés ici endroit dedens \num{ii} jors.>> La chastelaine prist l'aumosniere, si le vit et seut que çou estoit de l'ovrage dont ele covoitoit a savoir qui fait l'avoit, et si dist : <<En non Diu, sire, por cui amor on le m'envoie ait boine aventure et la dame ausi.>> -- <<Dame, dame, dist li chastelains, tous jours dites vos des vostres, et sachié que se je cuidasse que vos en deuïssiés tant dire encore, l'euissiés vos a avoir.>> -- <<Et coment, sire ! Si ne me porrai ja mile fois nier a vos que ce ne soit uns courous.>> -- <<Dame, dist il, en cestui giu ne en vostre pensee ne puet avoir se destorbier non.>> -- <<Sire, dist elle, or tenés v\supplied{ost}re aumosniere, car jou n'en ai mie mout a faire puis que vos vos courechiés, et bien voi que se il n'i avoit aucune chose que vos ne volés mie que je sace, que ja ensi ne parriés\fnnote{«parler» au cond. présent P5. On conserve cette graphie car elle est connue chez Froissart : Mais anchois vous parrés ["vous parlerez"] a Foi (FROISS., Dits Débats F., 1363-1393, 125).} come vos faites.>> Adont se courecha li chastelains et fu si hors dou sens qu'il hauça la paume et douna la chastelaine une paumee si grans que bien en peust on tenir une marchie de \num{x} mars. <<Sire, dist elle, fait avés vostre volenté et mout me doit desplaire ce que par autrui mesfait me deshonorés.>> Dont li redit \uncertain{toure} seule li chastelains quant une puciele vint a tel point, et dist : <<Ha ! Sire, refraingniés vostre ire, car mout est malordenee chose de lui issi courechier.>> -- <<Damoisiele, dist il, je n'en puis mie de mout, car ensi li plaist que je le face.>>


\pnum \lettrine[lines=2]{\color{darkred}E}{nsi} avint de la chastelaine qui onques ne queroit fors ochoison de son signor destorber et se pensa come cele en cui li diaubles s'estoit mis, que mar l'avoit se si rebatue, il le chourecheroit a chiertes. Dont ne demoura que li emperere, ensi come jou avoie dit desus, atorna sa chosete et vint enchiés le chastelain et le chastelaine si priveement que nus ne le seut, fors il et ses escuhiers qui en avoit seü l'afaire. Quant li emperere fu venus, une chambre li fu consecré en un gardin ensus de la sale et des chambres a le chastelaine et au chastelain. Illuech furent mout priveement et fu deffendu a l'esc\del{h}uhier et a un grant om cui il le covint savoir que nus n'en feist mension que il fussent laiens. Li emperere, qui mie ne volt sousfrir que il fust un jor sans labeur, prist congiet a ses maistres en cui ouvrage il avoit esté et dist qu'il li covenoit enchiés lui un poi labourer. Mout anieusement et a envis l'ont fait et toutes eures ne li vorrent il escondire, et il en vint a son rechiet o lui pluisors ouvriers et comencierent a faire une fortereche de fosse et de palis si fors que nus n'i \colmar{12rb}\colmar{b} passast jamais. Tandis que on ouvroit a ce faire, la chast\supplied{elaine} avoit mis son adru a raison ou il avoient esté le jour qu'il fu\supplied{st} enchiés l'empereour. <<Ma tres chiere dame, merci. Je me doute que se jo vos di verité, que vos ne vos desroiés si que on ne truist en vos \uncertain{sos ne riue}.>> -- <<Ja de ce ne doutés, mais que je puisse avoir vraie colour de ce que je sache, je ne veil autre chose.>> -- <<Je ne sai, dist cil, qu'el a autre color vos volés avoir, car chaiens est cele que mesire i a fait venir por mius acomplir sa volenté. Or i parra coment vos en porés mius ouvrer por plus malitiosement savoir la pure verité.>>

\pnum <<Coment, \persName{Fastre}, dist la chastelaine, dis tu que la chanlande a mon signor est chaiens, por cui jou ai esté batue ?>> -- <<Ma dame, por Diu merci. Je ne di mie que ele soit chanlade. Anchois vos jur sor la foi que jou ai a vos, qui ma chiere dame iestes, que jou, ainch jour de ma vie, \fnfpz{dislocation du sujet}je ne vi dame qui mius samblast iestre de boin liu \surplus{iestre} venue conme elle fait.>> -- <<Et de biauté, coment va ses afaires ?>> -- <<De la biauté, dist cil, ne vos sarés que dire, fors ce que de son ae il\fnnote{À la P3, «il» peut être utilisé comme féminin dans le Nord. G. Joly, Précis, p. 54. Le «il» désigne donc bien l’impératrice ici.} a passé toutes les dames de tes paiis come cele qui n'a mie plus de \num{xl} ans.>> -- <<Çou n'est mie de grant ae, dist la chastelaine, et or me dites en queil liu ele est chaiens.>> -- <<Dame, dist cil, d'une chose soiiés ausi chiertainne que de la mort, qu'il n'est hom qui vive ne feme, s'il venoit en liu ou la dame fust et il li voloit autre chose que toute honor, que maintenant ne fust devorés d'un liuon qui est aveuc la dame.>> -- <<Or me samble, dist la chastelaine, que tu me trusfes.>> -- <<Dame, dist cil, je les ai \supplied{vus} par vrai esperiment.>> Lors li conta coment il euist esté devorés emchiés l'empereour q\supplied{ua}nt il i fu, isi come il est conté de devant. Quant la chastelaine eut çou entendu, si eut mout grant doute dou liuon que se elle venoit en liu ou il fust, dont seroit elle bien devoree selonc çou que elle ne pouroit amer feme que ele seuist que ses sire amast nuilli. Nonporquant a ele tant enquis et fait que elle seut ou l'emperere estoit, si s'avisa d'une chose que forche ne guiers au roi diervet ni vauroit rien, mais par mallisse et ouvrer de traïson pouroit elle venir a çou que elle covoitoit. Atant vint o li chastelains estoit, se li jeta un faus ris et puis li jeta les bras au col, et dist : <<Metés ça m'aumosniere, car je cuit vraiement que jou euc hier soir la paumé de men droit !>> -- <<Ha ! Loués en soit li Dius rois de paradis, quant vos covissiés vostre folie.>> -- <<Ha ! Sire, dist la chastelaine, je vos aime tant qu'il me samble que je vos piert toutes les fois que je ne sai ou vos alés ne ou vos venés !>> -- <<Dame, dame, dist li chastelains, çou n'est mie amant, anchois la doit on tenir a haine.>> -- <<Par Diu, sire, dist elle, je le counois. Mais ensi m'en a Dius dounee la grasce que je n'en ai pau autre chose faire, si m'en repont et l'amenderai d'ore en avant.>> -- <<Si me rendés m'amosniere se vos ne le volés detenir por l'amor de celi qui çou fu.>> -- <<Et encore, sire, dist elle. ne vos courechiés mie, car je le di por l'amor de celui que je veil que vos le me rendés.>> -- <<Dame, dist il, mout volentiers je ferai mais il covenra que vos li renvoiés de vos juiaus arriere.>> -- <<Coment, sire ! Ja ne doit elle chaiens venir.>> -- <<Par ma foi, dame, bien euist aferut que sor teis paroles l'euïsse fat venir.>> -- <<Sire, dist elle, merci de çou que j'ai mespris et j'en ferai l'amende a vostre devis.>> \foliomar{12va}{https://gallica.bnf.fr/ark:/12148/btv1b10023851v/f30.image} \foliomarID{12v} -- <<L'amende, dist il, en iert grande de q\supplied{ue}le eure que vos i rechees.>> -- <<Et je l'oitroi, dist elle. Mais il covient que ele vigne chaiens, ou il me covenra aler ou ele est.>>


\pnum <<Dame, dist li chastelains, il covient que on vos acointe un poi de leur afaire avant que vos n'en sachiés el que je cuit que vos n'en sachiés.>> Dont l'en dist une partie ausi come il l'avoit acointié a son siergant. Mais onques tant n'i seut nient de coulor de boinne raison que ele ne quidast qui tout fust couverture. Mais par son atrait en mist a pais le chastelain. Que vos iroie ore plus atargant ? Conoistre li covint que la dame estoit laiens venue, ausi come jou ai dit deseus. Mais ne povoit nus parler a la dame se ses sire n'estoit presens, et por le liuon qui en garde l'avoit, car il ne fust nus qui, se il euist nule male volenté enviers li, que maintenant il ne devorroit s'il ert ensi que ses sire n'aloit au devant. <<En non Diu, sire, dont ne sai jou que dire se je la voise, por ce que jou ai eue une fole pensee, par coi il l'en soi remist, issi come vos savés.>> -- <<Dame, dist il, se vos i avés eu male pensee, si en metés vostre cuer a pais, car je tieng a miracle de Diu cesti chose conme la plus noble chose dont jou onques mais ai oï parler.>> -- <<Sire, voire, dist la dame, mais une chose i puet avoir que je vos dirai. Il ne s'en suit mie que se li liuons se muet contre moi por ce qu'il ne me conoist mie, que je por çou vausisse nul mal a la dame ne a son signor.>> -- <<Dame, dist il, je le cuit avoir prové par \persName{Fastre}, vostre siergant, que par ce que je sai et croi qu'il avoit male pensee enviers ces gens, il l'euist devoré se on li euist laissié, et moi n'euist il malmis ne fait, anchois se mist a mes piés quant je vieng a lui.>> -- <<En non Diu, sire, je vos oï conter merveille. Nonporquant je vos aseur que je n'aroie jamais bien si en sarai le covenant. Mais il seroit plus grans honors a moi et a vos qu'il venissent a moi que je a aus.>> -- <<Dame, sauve soit vostre grasce, je ne cuit mie que se vos saviés que il sont bien, que vos ne fussiés mout lïe d'iaus faire honor et cortoisie.>> -- <<Sire, por ce que je cuit et sai que vos counissiés iaus et lor lignié, si feriés vos bien se vos me volliés counoistre qui il sont.>> -- <<Ha ! Dame, merci. Je vos jur sor la foi que je vos ai plevie que je ne vos em puis el dire que vos avés di, fors que de tant que jadis fui jovene en la tiere de \placeName{Galylee} ou mes sires mes oncles me mena en une guerre qui la fu d'un prince de \placeName{Bethsaïde} et d'un autre de \placeName{Thyberiadis}. La ne peut on concorde trover, fors que par le bataille de \num{ii} chevaliers, dont je vos aseur que cis hom qui ci est venus s'arma encontre le plus redouté sarrasin qui fust en toute la susperior \placeName{Galylee}, et la fu cil outrés d'armes par le cors de cestui, que vos ore si poi prisiés.>> -- <<Sire, dist elle, or le veil je tant prisier que je me doute qu'il ne vos doie anuier, puis que je sai qu'il est de teil valour. Et si ne cuit mie qu'il ne soit de grant lignage et aveuc toute agrant renomee puet on sa¶voir qu'il n'a mie failli.>>


\pnum <<\lettrine[lines=2]{\color{darkblue}D}{ame}, dist dont li chastelains, en cesti louenge ne voi ge nule mal raison, et bien porés savoir encore la verité mius que vos encore ne faites.>> Atant en ont la parole escorcié, si ne demoura mie que en pau d'eure vint li emperere et li chastelains vint a lui et li dist : <<Sire, il covient que ma dame la chastelaine voie ma dam\supplied{e} l'empereris avant que elle se departe de chaiens.>> -- <<Chastelains, dist li emperere, puisqu'il le covient, nus ne puet \colmar{12vb}\colmar{b} aler au devant.>> -- <<Sire, dist il, je nel di por el que vos sacés bien que plus priveement ne puent iestre ensamble.>> -- <<Mout avés grant tort, chastelains, dist li emperere, sauve soie v\supplied{ost}re grasce, il vauroit ore mius aillors, car femes si sont si merveilleuses que tant i aroit ja de chieres qui nos empecheroient come une fine merveille. Et je ne voroie sousfrir por rien que li liuons feist nului mal.>> -- <<Sire, dist il, a vostre volonté.>> Atant repara li chastelains et la chastelaine li vint a l'encontre, et dist : <<Com il me samble que vos soiiés esfraés !>> -- <<Dame, dist il, jou n'en pus nient, car je doute que cil liuons ne vos face mal, qui a ce est acharnés come vos avés oï.>> -- <<Voire, dist ele, mout font ore grant train de cele dame voir. Mal ai jou se je le veil nient voir.>> -- <<Dame, dist il, droit avés, car ausi vos cousteroit elle de vos juiaus aucuns dont elle ne vos costera nul por que vos ne le veroi\uncertain{s}\fnnote{Ici, le scribe corrige veroie en verois}.>> -- <<Sire, dist elle, vos avés voir dit.>>


\pnum Mout cuida li chastelains avoir mise a pais le chastelainne, mais non fust, car